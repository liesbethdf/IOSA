\documentclass[12pt,english]{article}
\usepackage[T1]{fontenc}
%\usepackage[latin9]{inputenc}
\usepackage[utf8]{inputenc}
%\usepackage[french]{babel}
%\usepackage{pdfpages}
\usepackage{geometry}
\geometry{verbose,tmargin=2.2cm,bmargin=2.2cm,lmargin=2.2cm,rmargin=2.2cm}
\usepackage[parfill]{parskip} % Activate to begin paragraphs with an empty line rather than an indent
\usepackage{babel}
\usepackage{varioref}
\usepackage{float}
%\usepackage[caption = false]{subfig}
\usepackage{amstext}
\usepackage{amsmath, amsthm} 
\usepackage{mathtools}
\usepackage{amsfonts}
\usepackage{mathrsfs}
\usepackage{amssymb,graphics,psfrag}
%\usepackage{dsfont}
\usepackage{hyperref}
\usepackage{gensymb}
\usepackage{graphicx}
%\usepackage{subfig}
\usepackage{setspace}
%\usepackage{subcaption,siunitx,booktabs}
\usepackage{booktabs}
%\usepackage{caption}
\usepackage{subcaption}
\usepackage{pdflscape}
\usepackage{adjustbox}
\usepackage{esint}
%\onehalfspacing
\usepackage{cite}
\usepackage{natbib}
\usepackage{textcomp}
\usepackage{multirow}
\usepackage{array}
\usepackage{longtable}
\usepackage[dvipsnames,table]{xcolor}
\hypersetup{colorlinks=true,
			linkcolor=black,
			citecolor=Blue,
			urlcolor=Blue}
%\usepackage{tikz}
%\usetikzlibrary{positioning}
\usepackage{makecell}
\usepackage{tabularx}
\usepackage{array}

\newcommand{\mli}[1]{\mathit{#1}}


\PassOptionsToPackage{normalem}{ulem}
\usepackage{ulem}
\usepackage{babel}
% \usepackage[version=3]{mhchem}
 
\usepackage[textsize=small,textwidth=2cm,shadow]{todonotes}
\usepackage[]{todonotes}
\newcommand\lies[2][]{\todo[color=orange!50,#1]{ldf: #2}}
\newcommand\li[1]{\lies[inline]{#1}}
%\newcommand\etienne[2][]{\todo[color=yellow!50,#1]{es: #2}}
%\newcommand\es[1]{\etienne[inline]{#1}}
%\newcommand\antonin[2][]{\todo[color=green!50,#1]{ap: #2}}
%\newcommand\ap[1]{\antonin[inline]{#1}}
%\newcommand\rb[1]{\textcolor{green!80!black}{#1}}
%\newcommand\ru[1]{\textcolor{green!80!black}{\sout{#1}}}


%\title{Effect on employment from a low-carbon transition on South Africa} %On the non-neutrality of climate impacts and policies \ \\ \Large Working paper, not for distribution.

%\author{Liesbeth De Fossé}

\begin{document}
\defcitealias{united2002isic}{ISIC rev. 3.1}
\defcitealias{IOT2014}{StatSA, IOT 2014}
\defcitealias{MineralsCouncil2018Facts}{MinCo, 2018}
\defcitealias{MineralsCouncil2010Facts}{MinCo, 2010}
\defcitealias{british2018bpcoal}{BP, 2018}
\defcitealias{ChamberofMines2018Strategy}{Chamber of Mines, 2018}

\newcommand{\STAB}[1]{\begin{tabular}{@{}c@{}}#1\end{tabular}}

\tableofcontents

\begin{abstract}
The low-carbon transition constitutes both risks and opportunities for the South African economy.
The CPI study puts the possible loss in revenues from coal exports under a 2DS scenario as the major source of risk. 
Because of the procurement relationships in the economy, this loss of export-revenue does not only affect the coal sector, but affects many sectors throughout the economy. To estimate the distribution of risk over investors and workers over all sectors, the present study employs input-output tables, separated for domestic output and for imported goods. This is appropriate because the use of imported inputs does not create jobs in the domestic economy. In addition, the reduced value of imported inputs when export demand is lower is an attenuating factor for the value risk.

The uncertainty regarding how and at what speed productive systems will evolve to mitigate and adapt to climate change entails climate change-related risks for economies worldwide. This paper makes a macro-economic evaluation of the external transition risk --- outside of the control of the government --- faced by South Africa resulting from the possible transition of the world towards a socio-economic systems compatible with 2°C warming (a 2DS scenario). This study is a complement to the study on the transition risk faced by South Africa conducted by Climate Policy Initiative (CPI) using a sectoral and asset-level approach. CPI investigates the total value at risk and which actors bare this risk. The industrial sectors identified by CPI that are mainly effected by external factors under 2DS versus under BAU %face the majority of external risks and opportunities 
are coal mining and oil refining. The external risk under a 2DS is constituted of lower demand for coal exports and a lower price for coal on the international market. Lower price for crude oil under 2DS constitutes an opportunity for South Africa which is an oil importer. The present study uses a simple input-output model to offer an additional perspective on the analysis by CPI, by systematically investigating the way risk is transferred over the different sectors through procurement relationships. %and over workers, investors and the government. An estimate is also made of the transition risk to employment throughout the economy.  The transition risk is estimated by comparing 
An estimate is also made of what type of jobs, over different sectors, in the economy are at risk. %The biggest part of the risk faced by South Africa is external and stems from reduced exports under a 
This paper thus offers both additional estimates of the explicit distribution of the transition risk faced by South Africa with respect to the CPI study, as well as methodological insights into how the finance-based asset-level and the IO approaches can be combined to offer a more complete and detailed risk distribution than can be obtained by both approaches separately.
The IO-model estimates that in addition to the about xxxx number of jobs lost in the coal-mining sector due to reduced export volume, an additional xxxx jobs are at risk through the economy. xxx \% of these jobs are jobs with higher than average remuneration and xxx\% lower than average mainly because impacted sectors are capital intensive. Of the total value of exports lost, xxx\% falls on the mining sector, xxx\% on other sectors of the economy and x\% of the lost export value would have been spent on imported production inputs. The explicit risk distribution according to the input-output model puts xx\% on workers, xx\% on investors and xx\% on the government. Detailed sector and asset-level analyses of the firms within sectors and the specific investors in assets and firms, as performed by CPI, can then provide information on the financial risk to specific firms and investors due to a possible 2°C compatible wordlwide development trajectory as stipulated in the Paris Agreement. %distribute risk to investors over specific actors and analyze how the explicit risk distribution is further transferred in the economy.

The IO study estimates that about 20.000 jobs are at risk in the coal sector in South Africa, due to %the risk of reduced export demand under a 2DS scenario.
the lower expected coal exports under a 2DS scenario. In addition, for every job at risk in the coal sector, another job is at risk elsewhere in the economy. About xx\% of all the jobs at risk are jobs that pay wages higher than average and xx\% are lower-than average paying jobs. Of the total revenue export-income at risk of xx m Rand (USD) in NPV terms, xx\% is attenuated because of avoided costs of imported inputs, xx\% falls on workers and investors in the coal sector and xx\% in the rest of the economy.


\end{abstract}

\section{Introduction}

The aim of this study is to evaluate %from a macroeconomic perspective 
the risk to the South African economy that would result from a global transition to a low-carbon economy. This risk --- the transition risk --- forms together with the physical risk\footnote{The physical risk is the risk to assets and production systems stemming from changing climate patterns, induced by human activity.} the main categories of climate-related risks %defined by the Task Force on Climate-related Financial disclosures (TCFD) 
\citep{board2017recommendations}. The low-carbon transition risk is the risk that would result from the world following a development trajectory compatible with global warming limited to 2°C (a 2DS scenario) versus development according to a business as usual (BAU) scenario. Indeed, under both scenarios, regulations, reporting obligations, technological choices, costs of inputs and consumer demand will differ. For example, the reorientation of energy production towards renewables under a 2DS scenario would globally result in financial risk to investors in the fossil-fuel industry : lower demand for fossil fuels would result in lower valuations, early write-offs and retirements of fossil fuel-related assets. In addition, many jobs in the fossil-fuels sector would be lost. In South Africa, the main affected industries by a low-carbon transition are coal mining and its related activities as well as the petroleum sector.

This study is a complement to the study on transition risk in South Africa conducted by Climate Policy Initiative (CPI) \citep{CPI2019SA}.
The CPI study takes a sectoral, firm and, when necessary, asset-level approach. The present analysis adopts a macroeconomic perspective with a simple input-output model. The macroeconomic analysis is complementary in that it takes into account, in a systematic manner, the way risk spreads throughout the economy via the interdependencies between all industrial sectors via buyer and seller relationships. % and, to some extent, between actors. %analysis stems from the difference in the methodologies adopted.
The present study builds on results from the CPI-study. The methodology used by CPI starts with determining the sectors that are affected by the low-carbon transition. These are the sectors that are both important to the South African economy and are clearly subject to transition risk. 
%The identified sectors are the coal sector, oil refinement, the platinum metals group (check).
The CPI analysis covers both external risk, due to factors outside of the influence of South-Africa, as well as risk within the control of the South-African government, stemming from energy and industrial policies. External risk and opportunities arise from lower volume and export prices for coal, lower oil import prices, changes in metals and minerals demand and lower adaptation costs under a 2DS.
The risks within the influence of the South African government arise from choices for the development of the domestic power sector, which majorly impacts the coal mining sector, and also for the coal-to-liquids and oil refinery industries. 

Some of these risks were assessed by CPI with a detailed quantitative analysis, and some were qualitatively analyzed. Both the total value of the risk was estimated, as the allocation of the risk over the economy's actors, i.e. over investors, government and workers. The measure of risk used is the difference between the net present value To assess how risk is ultimately distributed, CPI proceeds in three stages. First the explicit risk is calculated. The explicit risk is as determined by demand addressed to each sector, regulation, policy and contracts and assuming companies are allowed to fail. Secondly, CPI models how this risk could be transferred within the economy by companies and government trying to mitigate their risk. Finally, when the residual risk cannot be borne by a company (leading for example to default), the risk will be passed on to investors or the government. 

CPI estimates the total value at risk at 123 billion USD, of which the explicit risk allocation puts close to 80\% on the shoulders of investors. However, after taking into account implicit risk transfers and transfers arising through contingent liabilities, more than half the risk would fall on the South African government. Finally, the CPI study makes recommendations and suggests policies to mitigate the low-carbon transition risk.

 
Overall, both methodologies allow to make assessments of different aspects of the transition risk. However, in some cases, the different methodologies assess the same aspects. %That the types of assessments obtained are largely different, make
%The type of results obtained from the sectoral- and firm-based analysis by CPI and from the present macroeconomic IOT analysis of the external transition risk overlap in part.
Where there is overlap, the assessments can be compared. Differences can be traced to the type of information present in the data and models used and the assumptions behind both methodologies. This is the case for the risk %--- arising from global oil and coal export/import trajectories under BAU and 2DS scenarios --- 
to employment and to the total value of risk. The explicit risk allocation over economic actors can also be compared to some extent between both studies. %for estimates, under BAU and 2DS scenarios, of employment in the coal sector --- both due to foreign and domestic demand --- and for the revenue different sectors and agents obtain stemming from coal exports. The difference between the estimates under both scenarios is a measure, in each methodology, for the number of jobs and the revenue at risk.
Before discussing the comparable results from both studies, a recap is given of the scope and methodology of the CPI study. This makes clear how the input-output study is related to CPI's. 


The present IO study is more restricted in scope. It estimates the total value of external risk arising from the difference in international demand for coal and oil under BAU and 2DS scenarios. In terms of risk allocation, it allows to asses the explicit distribution over industrial sectors. The distribution of the risk over actors is mostly limited to a distribution over workers on the one hand and investors, as a homogeneous group, on the other. Distribution of risk to government can be assessed to some extent via estimations of tax receipts. To further distribute risk over actors, information would need to be incorporated on who are the invested actors in the firms that make up the affected sectors. 

The coal and oil sectors are selected because for these, CPI conducts a detailed quantitative analysis. %, calculating the value at risk for key economic actors impacted : South African firms (SOE's, international majors, state-owned banks, BEE's and other SA firms), local and national government and workers. 
The starting points for the analyses are export/import volume and price trajectories for coal and oil. In the CPI analysis these are used as input to the sectoral and firm-level models to yield income projections for ... For the coal sector, total demand (export and domestic) is allocated over different mines according to contracts, location, type of coal mined etc.\lies{à compléter}. In the present study, the same volume and price trajectories are used in conjunction with a demand-driven IO-model to assess how the difference in demand between BAU and 2DS scenarios impacts not only the primarily affected importing and exporting sectors, but the whole economy. 





%In addition,  just as the CPI analysis yields many results 
The IOT analysis %is not equipped for, the latter 
allows to make a systematic assessment of how the export/import trajectories under both scenarios spread to possibly all sectors through the procurement relationships linking the economy.	% throughout the South-African economy. 
The CPI analysis makes detailed assessments of how the transition risk is spread over individual companies, mines and assets and on how this risk might be ultimately transferred to other agents, in particular the State. However, whereas it does provide an assessment of the risk to some infrastructure that supports export/import activity --- such as the Richard's Bay Coal line, asset owned by Transnet --- the methodology used is not equipped to make an estimate of how %systematically assess how 
different export-demand trajectories impact the economy as a whole in a systematic manner. In addition, the simple macroeconomic IO model allows to make projections of how risk to final demand might impact jobs, household income, consumption, investment and GDP.

The IO analysis projects that under a BAU scenario, the coal industry would sustain close to 80.000 jobs in 2035 and close to 61.000 under a 2DS scenario (table \ref{coalExport_BAUvs2DS_2035}). The difference stems from the loss of the volume of exported coal, which would put about 19.000 jobs at risk. The essential assumptions underlying these estimates are 1) that the number of workers is dictated by the number of tons mined, and not by the type of coal extracted (thermal coal or higher CV) or the price obtained per ton, and 2) that the productivity of all mines, whether they are multi-product mines, old or new, is the same. CPI \lies{Why the very similar projections under BAU and 2DS ?} projects 25,000-35,000 jobs lost under both scenarios. Under 2DS, there would be an earlier reduction of about 7,000 jobs (coal-related?) with respect to BAU. Towards the end of the period, the sector would sustain about 50,000 jobs. 

In addition, the IOT analysis projects close to 21,000 additional jobs at risk in other sectors of the economy, induced by lower export demand under 2DS versus BAU. A third of these is situated in the Transport sector (table \ref{coalExport_BAUvs2DS_2035_table_employment}).

The revenue at risk from coal exports represents about 1.93\% of 2017 GDP. In addition, the lost jobs represent lost income for households and thus diminished consumption, which further impacts GDP. This effect would lower 2014 GDP with another 0.25\%. %in addition to the 2.36\% directly lost due to coal exports. This effect is to be taken as a general indication. Under 2DS, the 39674 jobs would disappear over a period of 18 years. Aided by accompanying policies, new economic activity and jobs can be created throughout this period.
Less demand throughout the economy also lowers investment needs, by about 0.03\% of 2017 GDP. Overall, the value lost to reduced coal exports under 2DS and the indirect effects of diminished consumption and investment represent 2.2\% of 2017 GDP.



A low-carbon transition also implies lower global demand for petroleum products. South Africa imports all of its domestic demand in crude oil and refines domestically. %Figure \ref{Coal-Export_CPI_VolPrice_BAU_2DS} presents CPI's projections over the period 2018-2035 of imported volumes and import prices of crude oil under BAU and 2DS scenarios. 
Because of lower global demand for petroleum products under 2DS, the price on the international market for crude oil is projected to be lower than under BAU. Because of the lower price, South African demand for crude oil is higher under 2DS. In the middle of the 2020's, the Coke oven manufacture sector (of which oil refinement is a part) will refine about 5\% more crude oil than under BAU. Despite the fact that it is one of the smallest job creators per million Rand of final demand, under 2DS the larger refined volume would sustain about 8000-9000 more jobs throughout the economy. About 2300 jobs would be situated within the sector itself, another 2300 in the Trade sector and 800 in the Transport sector.



\li{below, to be completed, suggestion to compare the explicit risk distribution over sectors / actors between both studies}


To measure the risk different agents bear, the CPI study uses the difference of the net present values under the 2DS and the BAU scenarios of the agent's future operating and capital flows. The operating cash flow is defined as total revenue excluding net capital income, minus all costs incurred, including the costs of inputs, taxes and interest expenses but excluding depreciation. The capital cash flow includes (is it investment cash flow, or does it include financing cash flow, but the latter is assumed zero I suppose ?) the net income from investments and the expenses minus income from buying/selling capital assets. The total cash flow thus defined, operating plus capital, is equal \lies{Correct or not, what is the difference if not ?} to gross operating surplus.

The biggest share of the external transition risk is due to the difference of coal export revenue under both scenarios. The transition risk due to coal exports amounts to 83.7 billion USD net present value\footnote{How is it calculated, what is the discount rate used?}. Of this total value at risk, according to CPI analysis, 66.7 billion USD of explicit risk would lie with investors and 17 billion USD with the government. The 66.7 billion USD further distributes between international majors, BEEs\footnote{Black Economic Empowerment firms}, state-owned banks, SOE's and other South African companies\lies{In what sectors?}. %(The numbers are given for coal export + domeestic + oil together.)

The IO analysis decomposes future revenue flows for the coal sector into imports of inputs, taxes and value added for the coal sector as well as for its directly and indirectly supplying sectors. The avoided cost of imported inputs, due to the reduced export revenue under 2DS, is an attenuating factor for the total value at risk. For demand addressed to the coal sector, about 13\% of revenue is spent on imported inputs, throughout the production chain. This only accounts for consumables. The share of investment (gross fixed capital formation) that is imported is for the coal sector larger than on average in the economy, due to the machines and equipment used in the coal mines and the mines' increasing mechanization over the years. In 2014, more than half of investments generated by final demand for coal were imported. %Thus, the total amount of revenue spent on imports is higher.

The remaining part of future revenue is distributed to government in the form of taxes, to workers in the form of salaries and to investors in the form of gross operating surplus. In 2035\lies{For comparison of results, use not 2035 shares but NPV with same discount rate?}, of the almost 90,000 million Rand (6.4 billion USD) at risk, about 2\% would be borne by government (taxes), 30\% by workers (salaries) and the remaining 55\% by investors (gross operating surplus). Of the risk borne by investors, ~35-40\% is situated with investors in the mining sector, xx\% in the transport sector, xx\% in the trade sector and xx\% in other sectors.


%The study focuses on the transition risk --- i.e. the risk arising from the adoption of an economic apparatus on a world-wide scale that would make up a low-carbon economy --- due to external sources, outside of the influence of South African decision-makers. This external risk consists in the fact that the demand on the international market for certain goods exported and imported by South Africa, is impacted by a low-carbon transition and unanticipated at the time investments were made in assets to produce/transform exported/imported goods. 
%A low-carbon transition of the South African economy itself is not considered here. In this aspect the present report follows the approach taken by the Climate Policy Initiative (CPI) who conducted an analysis of the external transition risk faced by South Africa taking a micro-economic, sectoral, approach. At times when judged necessary, risk is in CPI's analysis evaluated at the level of individual assets : for example, because for the coal-mining sector the effect of coal-export demand on the profitability of mines depends on the type of contracts with purchasers each mine has already concluded at present, a min-per-mine analysis is made. 
%
%The present report differs from CPI's in that it approaches the issue from a macroeconomic perspective, offering a complementary point of view on the conclusions obtained from the micro-analysis. To allow for maximum comparability between the results obtained from both studies, while the toolboxes used are considerably different, the scenarios analyzed with the toolboxes are as much as possible the same. 
%
%Two scenarios are compared : a business as usual (BAU) scenario where current economic policy choices are assumed to hold for the period covered by the study (2108-2035), and a 2 degrees (2DS) scenario where on the contrary choices are made resulting in the development of a low-carbon socio-technological system that allows to contain global warming with a 50\% chance within the limit of 2°C with respect to pre-industrial times. Each of these scenarios implies distinct prospective export volume and/or price trajectories over the period under study for key sectors/commodities. The hypothesis for trajectories are obtained from CPI's models. Key export commodities are selected based on 1) whether export demand for them is different under BAU and 2DS and 2) whether they are important for the SA economy ; they are thermal coal, platinum and iron ore. A key imported commodity is crude oil, demand for which is satisfied by close to 100\% imports and of which the price on the international market might be lower under a 2DS versus under BAU. 

%This analysis complements the sector-, firm- and asset-level analysis conducted by CPI. The purpose of this complementary analysis is to evaluate whether the conclusions reached via two distinct methods are qualitatively similar. The two methods can in short be juxtaposed as follows : both on the micro and the macro level, the impact of changes to external conditions (export prices, export demand volumes) to which the SA economy is impacted is simulated/evaluated. For the goods exported by SA and identified as undergoing a significant change in demand in case of a low carbon trajectory followed by the world economy, CPI used scenarios for volume and price on the global markets under a 2DS and a BAU scenario. In the case of the micro analysis, the impact of these scenarios is evaluated on the economy on a sector, firm, and/or asset based level. The impacts on the individual units are then aggregated to obtain a figure representing the impact on the whole economy. The macro analysis follows the opposite route : the impact on the South African economy of the alternative demand scenarios for the main SA export goods that would be impacted by a world-scale low-carbon transition is evaluated on the level of the economy as a whole, from the macro economic point of view. From this economy-wide point of view, an effort is then made to pinpoint the observed effects on the macro-level to specific economic units (subsectors, firms). The two approaches are in a certain sense situated on the opposite side of the spectrum of how the economy can be described. 

%%The impact on the South African economy is assessed if the world adopts a transformation of the economy compatible with a xx\% chance of keeping the raise of the average global temperature under 2°C. This impact is explored in detail by CPI taking a microeconomic approach. CPI identified the key sectors that would be impacted by a global 2DS scenario and analyzed for each of the main actors/firms making up those sectors the impact an external 2DS would have on them. 

%%The development of a low-carbon economy on a worldwide scale, has implications for the quantities of fossil fuels used and, the demand for fossil fuels, specifically for coal, and for the demand for certain metals, etc. Thus, the nations that are providers, through export, of the primary commodities of which demand would be impacted by a low-carbon world-economy, will see there revenues impacted. 




\section{Method}

%The issues under scrutiny in this study are the ones identified by the CPI study. 
Under a 2DS scenario, South Africa is expected to export significantly less coal than under a BAU scenario. Secondly, the market price for crude oil, imported by South Africa, is expected to be lower under 2DS. %In addition, the CPI analysis showed that lower revenues from exported coal might result in a higher domestic price.

%when the coal sector faces different demand for export purposes under both scenarios, the differential impact on the economy is not limited to the coal sector but extends to the sectors that sell intermediate inputs to the coal sector. The macroeconomic picture provided by the IOT allows to systematically evaluate the impact of the external scenarios throughout the economy, taking into account how each sector depends on others for its revenue.

The differential impact of export and import volumes and prices under a BAU versus under a 2DS is not limited to the industrial sector doing the exporting and importing. The impact extends to the rest of the economy because commodities are made in value chains, linking industries through procurement and supply relationships. The impact on other industries arises from two different mechanisms. The first mechanism consists of how a change in the volume of the output of one sector impacts its supplying sectors. Indeed, if output doubles in volume terms, the demand for inputs addressed to other sectors will react with the same order of magnitude. This effect can be substantial : in terms of employment, the number of jobs created throughout the economy via demand for inputs, can be higher than the number of jobs created in the sector using the inputs. The second mechanism works in the opposite direction : a change in the price of a product impacts the sectors using its product as input. The change in price has repercussions on their profitability and possibly in turn on the price of their products. This is how increases in the price of energy can cause inflation in the economy.
%The first is how the volume of output of a sector  determines the volume of inputs bought from others.

%The macroeconomic picture provided by the IOT allows to systematically evaluate the impact of the external scenarios throughout the economy, taking into account how each sector depends on others for its revenue.

The methodology here used allows to take into account, at the macro-level, how one sector's demand or price influences the rest of the economy. The information needed to perform such an analysis %take into account these dependencies 
is provided by an input output table (IOT). %These tables provide quantitative information on what inputs each sector uses from the other sectors and on how the output from each of the sectors is used.  It is based on the input-output table (IOT) of the South African economy. 
An IOT is a quantitative representation of the economy as a whole showing the inter-dependencies between sectors formed by the procurement relationships. %These inter-dependencies can be described in two ways, both based on the same fact : the goods produced by every industrial sector are used on the one hand for final uses, such as electrical appliances bought by households, and on the other hand used as inputs by other sectors. Examples of the latter are  This fact is described in the IOT firstly 1) by showing 
The IOT shows how the output of each industrial sector is used throughout the economy : final goods for export, investment, for consumption by households, government as inputs for a number of other sectors and how much is imported. Inputs can be primary commodities such as coal or metal ores, processed goods such as steel or parts that can be assembled into machinery, etc. Because the IOT shows how the output of each sector is used, it consequently also shows what inputs every sector uses. These inputs are purchased from all sectors of the economy, including from itself. Alternatively put, the IOT shows how each sector's activity creates demand, and how much of it, to other sectors. For example, making cars out of steel generates economic activity and jobs in the steel sector. 

%The accounting equations represented in 
The IOT, in conjunction with sectoral employment data, can be used to estimate the number of jobs directly and indirectly generated by a given sector's activity. The table can then be used as the basis for creating a straightforward model to make estimates of how future developments in a sector's activity impact both GDP and the jobs sustained throughout the economy by demand addressed to the sector.

%The method here used provides a way to make an estimate, when demand to a sector changes --- for example following the above scenario, due to decreased exports or lower export prices --- of what the impact on the whole economy  will be. Indeed, not only the sector that directly faces the changed conditions is impacted, but also the sectors from which this first sector uses inputs. In their turn, this latter set of sectors use inputs, and thus the changed demand they face impacts the sectors from whom they use inputs, and so forth. The initially changed export demand thus generates a change in the volume of production, and also in the price they face and their profitability, of a whole number of industrial sectors. The impact on employment will thus not be limited to the industrial sector directly concerned, but will spread to a certain extent throughout the economy. For every scenario, both the amount of jobs directly and indirectly generated by an industry's productive activity are estimated.

The study performed by CPI also looks into the impact on the value chain in which the identified key sectors are situated. For example, the demand for coal does not only impact the corresponding part of the mining sector, but also the actor, Transnet, providing the transport by rail of the coal to the exporting ports. The present study evaluates these indirect impacts in a systematic manner.

The methodology is summarized as follows. The same price and volume trajectories for key export goods %impacted by a 2DS, 
evaluated at a micro-level by CPI, are evaluated at the macro level with the IO-model. The IO-model yields the difference in the impact of these export scenarios at the level of the economy as a whole between the BAU and 2DS scenarios. A distinction is made between the impact on the sector to which demand is addressed and the rest of the economy. %, the former due to different final (export) demand and the latter being the impact due to the consequently different intermediate demands. 
These impact will be estimated both in terms of loss or gain in GDP or growth, and in terms of loss of jobs. Indeed, a key aspect of evaluating the impact of export scenarios on the economy is not only to evaluate the lost revenue, but also to evaluate which economic actors bear this cost. CPI makes a detailed analysis of how the costs are born between the owners of capital. %stakeholders (firms, the government), ... from a financial point of view and on jobs via ... 
The input-output analysis here performed complements the analysis of the transition risk by estimating the impact on employment. 


%The impact on employment is assessed of the external threats, as identified by CPI, facing South Africa under a 2DS, from a macroeconomic standpoint. 
%The effect on employment (and global cost to the economy) assuming the baseline scenario, which is IEA's New Policies Scenario (NPS), is compared with CPI's 2DS scenario. The 2DS is an adaptation of the IEA's SDS. 


\subsection{The input-output table}

An input-output table is a quantitative representation of the economy as a whole, showing the inter-dependencies between sectors, formed by their mutual procurement and supply relationships. %These inter-dependencies can be described in two ways, both based on the same fact : the goods produced by every industrial sector are used on the one hand for final uses, such as electrical appliances bought by households, and on the other hand used as inputs by other sectors. Examples of the latter are  This fact is described in the IOT firstly 1) by showing 
The precise structure of an input-output table varies, %along with in line ith 
depending on the level of detail of the data collected. The more detailed the table, the heavier the data requirements. %The IOT is more detailed when the economy is disaggregated into more sectors, 

Figure \ref{IOT_basic} represents the structure of the IOT as supplied by the South African statistical agency (Stat SA). The first $n$ rows of the table list the $n$ industrial sectors (into which the producing economy is disaggregated) in their capacity as producers. The column the most to the right states for each their total production or output $x_i$. 
The columns represent the different agents, or institutional units, of the economy in their capacity as buyers. In each column is specified for what value they procure each of the goods in the $n$ rows. %The first $n$ columns are the $n$ industrial sectors, now in their role as buyers. 
The industries buy goods to use as intermediate inputs\footnote{The intermediate inputs are goods that are entirely consumed in the production process ; they are non-durable goods. The acquisition of durable goods, i.e. investment, is not specified in the IOT.} in their production process. Thus, this first $n \times n$ block in the IOT shows the industrial sectors in their buyer-seller relationships of intermediate inputs. 

The next columns, grouped under "Final demand", show what amount of each type of product\footnote{Product signifies what is produced, it stands for both goods and services.} is bought by the different agents as final consumption. These are the goods and services bought as is, not as an input for the production of another product. The five types of final expenditure are Consumption by households $c_i$ and government $g_i$ (Government expenditure), Exports $e_i$, Gross fixed capital formation (GFCF) $i_i$ and Changes in inventories $inv_i$. These first $n+5$ columns constitute total demand, for intermediate and final goods. The last column is "Imports" and specifies what amount $m_i$ of each type of good is imported. Imports are the second source of goods in addition to domestic production $x_i$. Demand is thus supplied through domestic production and importation. By attaching a minus-sign to imports, the first $n+6$ columns add row-wise up to domestic output. Thus, the rows show how the total output of each sector is used in the economy, or alternatively put, where demand for each good originates.

The first $n$ rows of the table can be written either in \emph{quantities} or in \emph{value}. Indeed, by construction, every sector $i$ in the IOT produces and sells a single type of composite product of type $i$. Thus, the different uses made of product $i$, minus imports, can be row-wise summed --- either in quantities or in value --- to obtain output, measured in respectively quantities or value. When measured in value, the price at which product $i$ is valued, is the price the producer receives for its product, after he has paid taxes and received subsidies. This is the basic price. %It is the price the producer obtained for its products, after he has paid taxes and received subsidies. 
With valuation at basic prices, the total output $x_i$ represents %corresponds to 
the %value 
resources (amount of money) that the producer has at its disposal to pay costs and distribute as profit. %that the produced obtained for its products, after he has paid taxes and received subsidies.


\begin{table}[!t]
	\centering
	\begin{adjustbox}{width=\textwidth}
		\renewcommand*{\arraystretch}{1.15}
		%\footnotesize
		\small
		\begin{tabular}{cr|ccccc|b{30pt}b{30pt}p{30pt}b{30pt}b{30pt}|c|c}
			%\toprule
			%& \STAB{\rotatebox[origin=l]{90}{Sector $1$}} & &\STAB{\rotatebox[origin=r]{90}{Sector $j$}} & & \STAB{\rotatebox[origin=l]{90}{Sector $n$}} & \STAB{\rotatebox[origin=r]{90}{Consumption}} & \STAB{\rotatebox[origin=l]{90}{Government expenditures}} & \STAB{\rotatebox[origin=r]{90}{Exports}} & \STAB{\rotatebox[origin=l]{90}{GFCF}} & \STAB{\rotatebox[origin=r]{90}{Changes in inventories}} &  \STAB{\rotatebox[origin=l]{90}{Imports}} & \STAB{\rotatebox[origin=r]{90}{Output}}  \\ 		
			\multicolumn{2}{c}{\ }		& \multicolumn{5}{c}{Using/ buying sectors} & \multicolumn{5}{c}{Final demand}  &  \multicolumn{2}{c}{\ }	   \\
			%\cmidrule{3-7}
			
			&		& Sector $1$ & \ldots& Sector $j$ &\ldots & Sector $n$ & Cons. & Gov. exp. & Exports & GFCF & Ch. inv. & Imports & Output  \\
			\hline
			\multirow{5}{15pt}{\STAB{\rotatebox[origin=c]{90}{Supplying/}}\STAB{\rotatebox[origin=c]{90}{selling sectors}}}& Sector $1$& $z_{11}$ & \ldots& $z_{1j}$ & \ldots & $z_{1n}$ & $c_1$ & $g_1$ & $e_1$ & $i_i$ & $inv_i$ & $-m_i$ & $x_i$  \\ 
			&\STAB{\rotatebox[origin=l]{90}{\ldots}}\ \ \ \ \ \ &  \STAB{\rotatebox[origin=l]{90}{\ldots}}&  & \STAB{\rotatebox[origin=l]{90}{\ldots}}&  &\STAB{\rotatebox[origin=l]{90}{\ldots}}&  &  & \STAB{\rotatebox[origin=l]{90}{\ldots}} &  &  & \STAB{\rotatebox[origin=l]{90}{\ldots}} &  \STAB{\rotatebox[origin=l]{90}{\ldots}} \\ 
			&Sector $i$& $z_{i1}$ & \ldots & $z_{ij}$ &\ldots & $z_{in}$ & $c_i$ & $g_i$ & $e_i$ & $i_i$ & $inv_i$ & $-m_i$ & $x_i$  \\ 
			&\STAB{\rotatebox[origin=l]{90}{\ldots}} \ \ \ \ \ \ &  \STAB{\rotatebox[origin=l]{90}{\ldots}}&  & \STAB{\rotatebox[origin=l]{90}{\ldots}}&  &\STAB{\rotatebox[origin=l]{90}{\ldots}}&  &  & \STAB{\rotatebox[origin=l]{90}{\ldots}} &  &  & \STAB{\rotatebox[origin=l]{90}{\ldots}} &  \STAB{\rotatebox[origin=l]{90}{\ldots}} \\ 
			&Sector $n$& $z_{n1}$ &\ldots & $z_{nj}$ & \ldots& $z_{nn}$ & $c_n$ & $g_n$ & $e_n$ & $i_n$ & $inv_n$ & $-m_n$ & $x_n$  \\ 
			\hline
			\multicolumn{2}{r|}{Total expenditures (Int./final)} & $z_1$ & & $z_j$ & & $z_n$ & $c$ & $g$ & $e$ & $i$ & $inv$ & $-m$ & $x$  \\  
			\hline
			\multicolumn{2}{r|}{Net taxes on products}& $t_1$ &\ldots & $t_j$ & \ldots& $t_n$ &  \multicolumn{7}{c}{\ }     \\ 
			\cline{1-7} 
			\multirow{3}{15pt}{\STAB{\rotatebox[origin=c]{90}{Value}}\STAB{\rotatebox[origin=c]{90}{added}}}&Wages& $w_1$ &\ldots & $w_j$ &\ldots & $w_n$ &  \multicolumn{7}{c}{\ }  \\ 
			&Other taxes& $ot_1$ &\ldots & $ot_j$ & \ldots& $ot_n$ &   \multicolumn{7}{c}{\ }    \\ 		
			&Gross op. surplus& $gs_1$ & \ldots& $gs_j$ &\ldots & $gs_n$ &  \multicolumn{7}{c}{\ }     \\
			\cline{1-7} 		
			&Output& $x_1$ & & $x_j$ & & $x_n$ &  \multicolumn{7}{c}{\ }  \\ 		
		\end{tabular}
	\end{adjustbox}
	
	\caption{\label{IOT_basic}Structure of a single-country Input-Output table. The IOT supplied by the South African Statistical Agency has this form.	 }
\end{table} 


Column-wise, in the first $n$ elements of each column, we can see what different goods each institutional unit is buying. The purchases by the $n$ industries, when considered in quantities, reveal their production structure\footnote{Why it is quantities that are relevant for the production structure can be understood by considering a toy example : to make one car, one needs $800kg$ of steel, $100kg$ of plastics, $1kg$ of paper, 1/2 a computer, etc.}. 

The columns as a whole can only be considered in value. They show how each sector's income obtained from selling it's output $x_i$ is spent. %: on intermediate inputs $z_i$, taxes minus subsidies (net taxes) on products $t_i$, wages $w_i$, other taxes $ot_i$, and gross operating surplus $gs_i$.
The first $n$ elements of each column sum, in value, for each sector $i$, to the total cost of intermediate inputs at basic prices $z_i$. However, from the purchaser's point of view, this is not the total cost of intermediate inputs. To $z_i$ one has to add the taxes minus subsidies on the products purchased $t_i$. What remains of output $x_i$ after paying the cost of intermediate consumption $z_i + t_i$ is value added $va_i$. %It represents the capacity of sector $i$ to create value by using non-durable goods, at their effective cost for the purchaser. 
Value added is made up of wages $w_i$, other taxes $ot_i$ and gross operating surplus $gs_i$.

A downside of the IOT in figure \ref{IOT_basic} is that it does not provide separate information on how domestic production and imports are used. Both are added to form a pool of total resources per type $x_i + m_i$, of which is shown how much of the resources are allocated to what use (\{$z_{ij}\}_{j=1,\ldots n}$, $c_i$, $g_i$, $e_i$, $i_i$, $inv_i$). However, per type of use is not specified how much stems from domestic production and how much from importation. This is important, because the IOT will serve to create a model that will shed light on the question of how exports %, which is final demand for the country in question, 
generate economic activity throughout the economy, via the procurement of inputs. If these inputs are imported, their production will not generate domestic activity and jobs. It is when the inputs are domestically produced, that demand for them generates domestic value added. 

%For the intended analysis, the distinction between the uses made of imported versus domestically produced goods and services is important. Indeed, consumption of domestic output creates economic activity, and thus jobs and value, within the domestic economy. In contrast, when for example an industry uses imported goods as intermediate inputs for its activity, this does not give rise to domestic but to foreign jobs. Is is only when domestically produced goods are used as intermediate inputs that jobs are generated and value available domestically is created, indirectly, through the interlocking dependencies between industrial sectors.

Figure \ref{IOT_importsUses} represents an IOT that integrates this information. Each of the uses \{$z_{ij}$, $c_i$, $g_i$, $e_i$, $i_i$,  $inv_i$\} is now split up in a domestic and an imported part, indicated with respective superfixes $^d$ and $^m$. For example, the industries' intermediate inputs $z_{ij}$ of table \ref{IOT_basic} ar split in two parts $z_{ij}^d$ and $z_{ij}^m$. The same equivalent relationship holds for all the other uses. We thus get : 
\begin{align}
z_{ij} &= z_{ij}^d + z_{ij}^m. \label{usesDecompMX} \\
c_i &= c_i^d + c_i^m \nonumber \\
&\ldots \nonumber
\end{align}  Note that the relationship is also assumed for exports $e_i$. Indeed, some of imported goods are re-exported. 	

On the basis of an IOT, a model can be built making specific assumptions on the future structure of the IOT based on the year of reference. The first set of assumptions are based on the first $n$ rows of the IOT in quantities and it's interpretation as a %. inspired by interpreting the first $n$ elements in the columns of each industry as a 
production technology or a production function. %Specifically, it is assumed that the amount of inputs per unit of output is constant. 
The second set of assumption are based on the IOT in value. They determine the shares of output spent on taxes, labour costs and gross operating surplus.

\begin{table}[!t]
	\centering
	\begin{adjustbox}{width=\textwidth}
		\renewcommand*{\arraystretch}{1.15}
		%\footnotesize
		\small
		\begin{tabular}{cr|ccccc|b{30pt}b{30pt}p{30pt}b{30pt}b{30pt}|c|c}
			%\toprule
			%& \STAB{\rotatebox[origin=l]{90}{Sector $1$}} & &\STAB{\rotatebox[origin=r]{90}{Sector $j$}} & & \STAB{\rotatebox[origin=l]{90}{Sector $n$}} & \STAB{\rotatebox[origin=r]{90}{Consumption}} & \STAB{\rotatebox[origin=l]{90}{Government expenditures}} & \STAB{\rotatebox[origin=r]{90}{Exports}} & \STAB{\rotatebox[origin=l]{90}{GFCF}} & \STAB{\rotatebox[origin=r]{90}{Changes in inventories}} &  \STAB{\rotatebox[origin=l]{90}{Imports}} & \STAB{\rotatebox[origin=r]{90}{Output}}  \\ 		
			\multicolumn{2}{c}{\ }		& \multicolumn{5}{c}{Using/ buying sectors} & \multicolumn{5}{c}{Final demand}  &  \multicolumn{2}{c}{\ }	   \\
			%\cmidrule{3-7}
			
			&		& Sector $1$ & \ldots& Sector $j$ &\ldots & Sector $n$ & Cons. & Gov. exp. & Exports & GFCF & Ch. inv. & Imports & Output  \\
			\hline
			\multirow{5}{25pt}{\STAB{\rotatebox[origin=c]{90}{Supplying/}}\STAB{\rotatebox[origin=c]{90}{selling sectors}}\STAB{\rotatebox[origin=c]{90}{(Domestic)}}}& Sector $1$& $z^d_{11}$ & \ldots& $z^d_{1j}$ & \ldots & $z^d_{1n}$ & $c^d_1$ & $g^d_1$ & $e^d_1$ & $i^d_i$ & $inv^d_i$ &  & $x_i$  \\ 
						&\STAB{\rotatebox[origin=l]{90}{\ldots}}\ \ \ \ \ \ &  \STAB{\rotatebox[origin=l]{90}{\ldots}}&  & \STAB{\rotatebox[origin=l]{90}{\ldots}}&  &\STAB{\rotatebox[origin=l]{90}{\ldots}}&  &  & \STAB{\rotatebox[origin=l]{90}{\ldots}} &  &  & &  \STAB{\rotatebox[origin=l]{90}{\ldots}} \\ 
			&Sector $i$& $z^d_{i1}$ & \ldots & $z^d_{ij}$ &\ldots & $z^d_{in}$ & $c^d_i$ & $g^d_i$ & $e^d_i$ & $i^d_i$ & $inv^d_i$ &  & $x_i$  \\ 
			&\STAB{\rotatebox[origin=l]{90}{\ldots}} \ \ \ \ \ \ &  \STAB{\rotatebox[origin=l]{90}{\ldots}}&  & \STAB{\rotatebox[origin=l]{90}{\ldots}}&  &\STAB{\rotatebox[origin=l]{90}{\ldots}}&  &  & \STAB{\rotatebox[origin=l]{90}{\ldots}} &  &  &  &  \STAB{\rotatebox[origin=l]{90}{\ldots}} \\ 
			&Sector $n$& $z^d_{n1}$ &\ldots & $z^d_{nj}$ & \ldots& $z^d_{nn}$ & $c^d_n$ & $g^d_n$ & $e^d_n$ & $i^d_n$ & $inv^d_n$ &  & $x_n$  \\ 
			\hline
			\multirow{5}{25pt}{\STAB{\rotatebox[origin=c]{90}{Supplying/}}\STAB{\rotatebox[origin=c]{90}{selling sectors}}\STAB{\rotatebox[origin=c]{90}{(Rest of the world)}}}& Sector $1$& $z^m_{11}$ & \ldots& $z^m_{1j}$ & \ldots & $z^m_{1n}$ & $c^m_1$ & $g^m_1$ & $e^m_1$ & $i^m_i$ & $inv^m_i$ & $-m_i$ &  \\ 
			&\STAB{\rotatebox[origin=l]{90}{\ldots}}\ \ \ \ \ \ &  \STAB{\rotatebox[origin=l]{90}{\ldots}}&  & \STAB{\rotatebox[origin=l]{90}{\ldots}}&  &\STAB{\rotatebox[origin=l]{90}{\ldots}}&  &  & \STAB{\rotatebox[origin=l]{90}{\ldots}} &  &  & \STAB{\rotatebox[origin=l]{90}{\ldots}} &   \\ 
			&Sector $i$& $z^m_{i1}$ & \ldots & $z^m_{ij}$ &\ldots & $z^m_{in}$ & $c^m_i$ & $g^m_i$ & $e^m_i$ & $i^m_i$ & $inv^m_i$ & $-m_i$ &   \\ 
			&\STAB{\rotatebox[origin=l]{90}{\ldots}} \ \ \ \ \ \ &  \STAB{\rotatebox[origin=l]{90}{\ldots}}&  & \STAB{\rotatebox[origin=l]{90}{\ldots}}&  &\STAB{\rotatebox[origin=l]{90}{\ldots}}&  &  & \STAB{\rotatebox[origin=l]{90}{\ldots}} &  &  & \STAB{\rotatebox[origin=l]{90}{\ldots}} &  \\ 
			&Sector $n$& $z^m_{n1}$ &\ldots & $z^m_{nj}$ & \ldots& $z^m_{nn}$ & $c^m_n$ & $g^m_n$ & $e^m_n$ & $i^m_n$ & $inv^m_n$ & $-m_n$ &   \\ 
			\hline
		\multicolumn{2}{r|}{Total expenditures (Int./final)} & $z_1$ & & $z_j$ & & $z_n$ & $c$ & $g$ & $e$ & $i$ & $inv$ & $-m$ & $x$  \\  
		\hline
		\multicolumn{2}{r|}{Net taxes on products}& $t_1$ &\ldots & $t_j$ & \ldots& $t_n$ &  \multicolumn{7}{c}{\ }     \\ 
		\cline{1-7} 
		\multirow{3}{15pt}{\STAB{\rotatebox[origin=c]{90}{Value}}\STAB{\rotatebox[origin=c]{90}{added}}}&Wages& $w_1$ &\ldots & $w_j$ &\ldots & $w_n$ &  \multicolumn{7}{c}{\ }  \\ 
		&Other taxes& $ot_1$ &\ldots & $ot_j$ & \ldots& $ot_n$ &   \multicolumn{7}{c}{\ }    \\ 		
		&Gross op. surplus& $gs_1$ & \ldots& $gs_j$ &\ldots & $gs_n$ &  \multicolumn{7}{c}{\ }     \\
		\cline{1-7} 		
		&Output& $x_1$ & & $x_j$ & & $x_n$ &  \multicolumn{7}{c}{\ }  \\ 		
			
		\end{tabular}
	\end{adjustbox}
	
	\caption{\label{IOT_importsUses}Structure of a single-country Input-Output table where the uses of imports are specified.}
\end{table} 


%\clearpage

\subsection{Input-output demand-pull model}

A much used application of the IOT is the demand-pull model. The demand-pull model allows to calculate, for a given scenario of final demand, how much output each of the sectors needs to provide, and how many goods need to be imported, to satisfy that demand. %The model calculates the total production requirements to satisfy demand. 
The total output needed is made up of the sum of the products used in different stages in the production process. The first component of output is of course final demand itself. To that, one has to add the inputs needed to produce these final goods and services. These inputs are called the direct requirements. In addition, for the production of these inputs, an additional set of inputs are needed, the production of which requires in their turn again more inputs, etc. The sum of all these latter input requirements are called the indirect input requirements. The sum of direct and indirect requirements are the total input requirements. Total requirements and final demand together constitute output. It is through this process that final demand addressed to one sector also influences the level of output of many or all other sectors.

In the context of this paper, the comparison of different export scenarios corresponding to BAU and 2DS assumptions, translates into comparing different sets of output demanded from the $n$ industrial sectors. The different projected revenue streams (from selling output) under both scenarios imply different levels of employment, taxes collected and capital income over all sectors. The latter affects the valuation of individual companies that make up the most impacted industries. 

In addition, because the model constructed here is based on the IOT in figure \ref{IOT_importsUses} that specifies the use of imports, it allows to determine what share of revenue stemming from the sale of final products is spent on imports. 

The demand-pull model is based on the interpretation of the first $n$ rows of the IOT as a production function, which needs an IOT in quantities. The IOT for South Africa is provided in value (millon Rand), as is generally the case. Using its values to construct a demand-pull model implies that all prices are assumed equal for all types of goods and constant over the different uses made by the purchasing agents. For mining and energy sectors however, data are often available in physical units, tons or energy units. Hence for these sectors, actual quantity units can be used in the IOT. In this study, this is the case for the output of the coal mining sector which is specified in million tons in the IOT.

Thus, the demand-pull model is based on the tenet that for each sector $j$, the ratio of each type of input $z_{ij}^k$, $i$ $\in$ \{1,\ldots, n\}, $k$ $\in$ \{d, m\} to the sector's output $x_j$ is constant\footnote{Cf. the car example above.}. These ratios are denoted $a_{ij}^k$. They represent the output needed of a certain type to make one unit of output : 
\begin{equation}
a_{ij}^k = z_{ij}^k/x_j. \label{sharesIntermediate}
\end{equation}
Secondly, we make the hypothesis of constant shares of domestically produced final goods for each type of final demand to the %, consumption, government expenditure, exports, capital formation, of each type of good $i$, to the 
total (imported plus domestically produced) final demand of that type, is constant. For example for consumption, we suppose the constant share $s^d_{ci}$ (eqn. \ref{sharesFinal}). Similar hypothesis are made for government expenditure, exports and capital formation of all the $n$ types of goods.
\begin{align}
s^d_{ci}&=c_i^d/c_i, \ \ \  i=1:n \label{sharesFinal}\\
s^d_{gi}&=g_i^d/g_i, \ \ \  i=1:n \nonumber\\
&\ldots \nonumber
\end{align} 

The combination of the hypothesis embodied in equations \ref{sharesIntermediate} and \ref{sharesFinal} and the accounting equations embodied in the rows of the IOT (figure \ref{IOT_importsUses}) yield the model. The accounting equations hold that (a) for every sector $i$ ($i \in \{1,\ldots,n\}$), its total output $x_i$ is given by the sum of total intermediate inputs plus final demand minus imports of that good, (b) resources stemming from importation are used for final and intermediate consumption and (c) the same holds for domestic production. Using eqns.\ref{usesDecompMX} and \ref{sharesIntermediate}, we get : %: $x_i = \sum_j z_{ij} + c_i + g_i + i_i + inv_i + e_i - m_i$, where $z_{ij}$ stands for the intermediate inputs used by sector $j$ and provided by sector $i$, $c_i$ and $g_i$ stand for consumption by household and government, $i_i$ is investment, $e_i$ exports and $m_i$ is imports. This equation can be rewritten as equation \ref{IOT} to show, on the left-hand side, how the total amount of goods available to be used by the domestic economy is obtained, and on the right-hand side how the goods are used :  
\begin{align}
x_i + m_i &= \sum_j a_{ij} \, x_j + c_i + g_i + i_i + inv_i + e_i \label{useAll} \\
m_i &= \sum_j a_{ij}^m \, x_j+ c_i^m + g_i^m + i_i^m + inv_i^m + e_i^m \label{useImports} \\
x_i &= \sum_j a_{ij}^d \, x_j + c_i^d + g_i^d + i_i^d + inv_i^d + e_i^d \label{useOutput} 
%x_i + m_i &= \sum_j (z_{ij}^d + z_{ij}^m) + c^d_i + g_i^d + i_i^d + inv_i^d + e_i^d +  c_i^m + g_i^m + i_i^m + inv_i^m + e_i^m \label{allUSes}
\end{align}

%The goods obtained through domestic production and imports are consumed domestically, for intermediary ($z_{ij}$) or final  purposes ($c_i$, $g_i$, $i_i$, $inv_i$) and the remaining share is exported and a source of revenue. However, the equation does not provide information on what share of consumption by households of good $i$ stems from domestic production $c_i^d$ versus from imports $c_i^m = c_i - c_i^d$. The same is true for the other uses $u_i$, $u_i \in \{z_{ij}, c_i, g_i, i_i, inv_i, e_i\}$ on the right-hand side of eqn. \ref{useAll}.

%(We suppose that the imported goods are not exported. This assumption is not neutral and thus not necessarily hold for all sectors, given the aggregation level of the SA IO table. For example, for the sector Motor vehicles ($I32$), a substantial share, $32\%$ of the value of the sector's inputs are from within the same sector. Additionally, $25\%$ of total output is exported and $76\%$ of total input is imported (+ imports are larger than total intermediate inputs). This suggests that there is both imports of finished vehicles, since import is larger than intermediate use in this category. Additionally, given the high value of imports, it suggests that there is import of parts which get assembled in SA.)

%To take into account that the demand for imported goods does not directly create value or give rise to jobs domestically (although the imported goods might be necessary for a particular production process), separate IOTs for domestic output and imported goods and services are needed\footnote{The OECD provides such tables at a level of detail of 35 sectors, the most recent one for the year 2011. The South African statistical agency provides a table with a level of detail of 50 sectors, the most recent for the year 2014, but that does not distinguish the use of imports.}. Each of these specifies accounting equations \ref{useOutput} and \ref{useImports} that permit to decompose eqn. \ref{IOT} in eqn. \ref{allUSes}:

We can now use the two sets of equations \ref{useOutput} and \ref{useImports} to calculate, for a given final demand scenario $y_i= c_i + g_i + i_i + inv_i + e_i$, the output required of all sectors and importation of all types of goods, assuming the $a_{ij}^k$, $i$ $\in$ \{1,\ldots, n\}, $k$ $\in$ \{d, m\} are known.
%To study the impact of changes in export or domestic price changes of specific goods and services on the domestic economy, only the set of equations \ref{useOutput} is needed.
The set of equations \ref{useOutput} yield domestic production which in turn allows the calculation of imports with the equations \ref{useImports}. 

For easier notation, we write these two sets of equations in matrix format. Output, imports and the five types of final demand for the $n$ types of goods are each grouped in $n \times 1$ column vectors, which are denoted in bold small letters. Capital letters indicate $n \times n$ matrices. The total, domestic and imports direct requirements matrices, $A$, $A^d$ and $A^m$, have as elements on row $i$ and column $j$ the technical coefficients $a_{ij}$, $a^d_{ij}$ and $a^m_{ij}$. Five matrices are introduced that contain the shares of domestic/imported final demand in total final demand (given by equation \ref{sharesFinal}) on the diagonal and zero's elsewhere  : $S^k_{c}$, $S^k_{g}$, $S^k_{e}$, $S^k_{i}$ and $S^k_{inv}$, $k$ $\in$ \{d, m\}. Of course, $S^d_{c} + S^m_{c} = \mathbb{I}$, where $\mathbb{I}$ is the $n \times n$ unitary matrix.

\begin{align}
\boldsymbol{x} &= A^d\,  \boldsymbol{x} + S^d_{c} \, \boldsymbol{c} + S^d_{g} \, \boldsymbol{g} + S^d_{i} \, \boldsymbol{i} + S^d_{inv} \, \boldsymbol{inv} + S^d_{e}\,  \boldsymbol{e} \label{useOutputsDom} \\
\boldsymbol{m} &= A^m \, \boldsymbol{x} + S^m_{c} \,\boldsymbol{c} + S^m_{g}\, \boldsymbol{g} + S^m_{i}\, \boldsymbol{i} + S^m_{inv}\, \boldsymbol{inv} + S^m_{e}\, \boldsymbol{e} \label{useOutputsImport} 
\end{align}

Rearranging equation \ref{useOutputsDom}, we get :
\begin{align}
\boldsymbol{y}^d &= S^d_{c} \, \boldsymbol{c} + S^d_{g} \, \boldsymbol{g} + S^d_{i} \, \boldsymbol{i} + S^d_{inv} \, \boldsymbol{inv} + S^d_{e} \, \boldsymbol{e} \nonumber \\
\boldsymbol{y}^m &= S^m_{c} \, \boldsymbol{c} + S^m_{g} \, \boldsymbol{g} + S^m_{i} \, \boldsymbol{i} + S^m_{inv} \, \boldsymbol{inv} + S^m_{e} \, \boldsymbol{e} \nonumber  \\
\boldsymbol{x} &= \big(\mathbb{I} - A^d \big)^{-1} \, \boldsymbol{y}^d \label{outputVector} \\
\boldsymbol{m} &= A^m \, \boldsymbol{x} + \boldsymbol{y}^m \label{importVector} 
%\boldsymbol{m} &= A^m  \boldsymbol{x} + \boldsymbol{c}^m + \boldsymbol{g}^m + \boldsymbol{i}^m + \boldsymbol{inv}^m + \boldsymbol{e}^m \label{useOutputsImport} 
\end{align}
Matrix $\big(\mathbb{I} - A^d \big)^{-1}$ is the domestic Leontief inverse which we note $B^d$. A basic hypothesis made here is that the inputs needed to produce one unit of output are the same at all stages of the production process. With "stages" is meant the production of the final products, of the direct and of the indirect requirements. The matrix $B^d$ is also called the domestic total requirements matrix.
The elements $x_i$ of $\boldsymbol{x}$ are equal to a sum of $n$ terms : $x_i = \sum_j b_{ij}y_j$. Each of these terms are interpreted as follows : $b_{ij}y_j$ represents the output of sector $i$ stemming from the final demand addressed to sector $j$. The total $x_i$ is the total output needed from sector $i$ to satisfy the demand scenario $\boldsymbol{y}$. Because we are interested in each of these terms, the diagonal matrices $C$, $G$, $E$, $I$, $INV$ and $Y$ are introduced with the elements of the respective vectors (denoted by the same letter in lower case and bold) on the diagonal and zeros off-diagonal. Then, eqn. \ref{outputVector} becomes $X = \big(\mathbb{I} - A^d \big)^{-1} \, Y^d$, where $X$ has elements $x_{ij} = b_{ij}y_{jj}$. The interpretation of these elements is the same as for $b_{ij}y_j$ given above, since $y_j = y_{jj}$.

Similarly, we are interested in the terms that make up the vector $\boldsymbol{m}$ as given by eqn. \ref{importVector} or of the matrix $M$ given by $M = A^m \, X + Y^m$. Element $m_{ik}$ of $M$ is given by $m_{ik} = \sum_j a_{ij}x_{jk}$ which represents the the import of good $i$ by all sectors in response to demand of sector $k$. Element $m_i$ of $\boldsymbol{m}$ is then the import of good $i$ in response to demand from all sectors. However, for our analysis, we would like to know the import of all $n$ types of goods, done by each sector, in response to demand from sector $k$. Thus, we would like to know the sum $\sum_i a_{ij}x_{jk}$, which is the import expense of sector $j$ in response to demand from sector $k$. 

To this end we introduce the matrices $X^k$, which are diagonal matrices with the $k$th column of $X$ on the diagonal and zero elsewhere. The demand-pull IO-model, which allows to calculate the output and import by all the sectors needed to satisfy given final demand $\boldsymbol{y}$ (and if relevant its components, eg. consumption,\ldots), then becomes :
\begin{align}
Y^d &= S^d_{c} \, C + S^d_{g} \, G + S^d_{i} \, I + S^d_{inv} \, INV + S^d_{e} \, E \label{IOmodel_Yd} \\
Y^m &= S^m_{c} \,C + S^m_{g}\, G + S^m_{i}\, I + S^m_{inv}\, INV + S^m_{e}\, E 		\label{IOmodel_Ym} \\
X &= \big(\mathbb{I} - A^d \big)^{-1} \, Y^d  										\label{IOmodel_X} \\
%M^k &= A^m \, X^k + Y^m, \ \ \  k \in \{1,\ldots,n\} \label{Mk}
M^k &= A^m \, X^k, \ \ \  k \in \{1,\ldots,n\} 							\label{IOmodel_Mk}
%\boldsymbol{m} &= A^m  \boldsymbol{x} + \boldsymbol{c}^m + \boldsymbol{g}^m + \boldsymbol{i}^m + \boldsymbol{inv}^m + \boldsymbol{e}^m \label{useOutputsImport} 
\end{align}
In summary, the above model allows to calculate the output of all sectors and import by all sectors needed to satisfy a given final demand scenario. The matrices with parameters of the model are the direct requirement matrices $A^d$ and $A^m$, and the share matrices $S^d_{c}$, $S^d_{g}$, $S^d_{e}$, $S^d_{i}$ and $S^d_{inv}$. Final demand for domestic production is given by $Y^d$ which is zero off-diagonal ; its $i$th diagonal element is final demand for domestically produced goods addressed to sector $i$. Final demand for imported goods is given by $Y^m$, of which the $i$th diagonal element represents the final demand for imported goods of sector $i$ ; again, the off-diagonal elements are zero. The results of the model are to be found in the output matrix $X$ and, the $n$ import matrices $M^k$ and the matrix with imports of final goods $Y^m$. The elements $x_{ij}$ of $X$ are the output sector $j$ needs from sector $i$ to satisfy the final demand scenario. The elements $m^k_{ij}$ of $M^k$ represent the imports of good $i$ done by sector $j$ to satisfy demand (in domestically produced goods) from sector $k$. Thus, the columns sums of $M^k$ are the imported goods, intermediate and final, of all types, sector $j$ needs to satisfy demand from sector $k$.


%In the absence of seperate IOTs for domestic production and imports, the fast solution is to suppose that for every type of use $u_i$, the share $s_{i}^d$ of domestic output within the total use is equal over all the types of uses made :
%\begin{align}
%s_i^d &= \frac{y_i}{y_i + m_i} \\
%u_i^d &= s_i^d \, u_i \label{shareM} \\
%u_i^m &= (1-s_i^d) \, u_i
%\end{align}
%where $u_i^d$ and $u_i^m$ stand for all the possible uses made of respectively domestic output and imported products : $u_i^m \in \{z_{ij}^m, c_i^m, g_i^m, i_i^m, inv_i^m, x_i^m\}$ and $u_i^d \in \{z_{ij}^d, c_i^d, g_i^d, i_i^d, inv_i^d, x_i^d\}$.

%Going to back to matrix notation and denoting $S^d$ the diagonal matrix of the shares $s_i^d$ and 0's off-diagonal, then :
%\begin{align}
%A^d &= S^d \, A \\
%C^d &= S^d \, C \\
%G^d &= S^d \, G \\
%I^d &= S^d \, I \\
%INV^d &= S^d \, INV \\
%E^d &= S^d \, E 
%\end{align}

%Combining the above with eqn. \ref{useOutputsDom} and denoting $\mathbb{I}$ the identity matrix, we obtain the demand-driven input-output model that allows to determine for a given scenario of final demand $D = (C, G, I, INV, X )$, the output per sector needed :
%\begin{equation}
%%Y  = S^d \, A \, Y + S^d \, (C + G + I + INV + X)  \label{IO1} \\
%X = (\mathbb{I} - S^d \, A)^{-1} \,  S^d \, (C + G + I + INV + E)  \label{IOdemand-driven}
%\end{equation}

\subsection{Decomposition of final demand-revenue} \label{Section_decomposition}

A share of the revenue obtained from selling final goods is spent on inputs. The remainder is value added and pays workers, goes to the state via taxes and remunerates capital. Capital remuneration includes payments on debt and to shareholders. In addition, a share of inputs and of final demand is imported. This is represented in the column-wise accounting equations that result from the IOT in value. Moreover, the share of revenue that is domestically spent on inputs is a cost to the sector but not a loss for the domestic economy : %the cost to sector $i$ 
it is revenue for supplying sectors, who after paying in turn their suppliers, use the remainder, value added, to pay wages, taxes and remunerate capital. This suggests that the export revenue can be traced back into a part leaving the country, and a part distributed as value added --- wages, taxes and capital remuneration --- throughout the economy. 

%The revenue coming into the country from selling goods to other countries is distributed to different agents. Part of the export revenue pays workers, part goes to the state via taxes, and the remaining part represents the remuneration of capital, which is in practice the money used to pay back debt, pay shareholders, etc. However, not all the revenue coming into the country via exports of say sector $i$ stays within the country. A share of revenue is used to import intermediate goods used in the production of the exported goods and thus leaves the country. The part of the export revenue that is used by sector $i$ to buy intermediate goods for its production (of exported goods) is a cost to the sector but not a loss for the domestic economy : the cost to sector $i$ is revenue for supplying sectors, who after paying in turn their supplying sector, use the remainder, value added, to pay wages, taxes and remunerate capital. 

%The cost of imported inputs represent export revenue for the rest of the world (RoW) and thus represent foreign value added, which is embedded in gross exports, part of final demand y. 

First we show that the revenue coming in via final demand, in our case we are most interested in exports, can be decomposed entirely in value added created in the different sector of the economy.

In the previous section, the quantities of output from all sectors were determined to satisfy demand for a quantity of final goods. From the output matrix in quantities $X$, with elements $x_{ij}$ representing the output from sector $i$ resulting from final demand addressed to sector $j$, the output matrix in value $X^v$ can be obtained. Denoting $P$ the diagonal matrix with elements $p_{ii}$ equal to the price of product $i$ and the off-diagonal elements equal to zero, then : $$X^v = P \, X$$  %detailing the output needed from all sectors in response to final demand addressed to all sectors, 
Final demand revenue can then be decomposed into imports, value added and net taxes on products. To show this, every element $x^v_{ij}$ is first decomposed into domestically produced $z^{tot,d}_{ij}$ and imported intermediate inputs $z^{tot,m}_{ij}$, taxes on products $t_{ij}$ and the constituent parts of value added, wages $w_{ij}$, other taxes $ot_{ij}$ and gross operating surplus $gs_{ij}$ :   
\begin{equation}
x^v_{ij} = \underbrace{z^{tot,d}_{ij} + z^{tot,m}_{ij} + t_{ij}}_{cost\, of\, inputs} + \underbrace{w_{ij} + ot_{ij} + \mli{gs}_{ij}}_{value\, added} \label{decompositionOutput1} \\
\end{equation}
Each of the elements on the right hand side is determined based on the assumption 
%The assumption that underlies this decomposition is 
that for each type of product $i$, its production technology (represented by the $i$th column of $A$)%, denoted as $\boldsymbol{a}_j$)
, the shares per output of wages, taxes and gross operating surplus, are only dependent on the type $i$ and, importantly, independent of what use is made of the output of sector $i$ (as inputs by any of the sectors $j$ or as final demand). For example, for all sectors $i$, the share of wages per output of good $i$ is independent of the origin of demand for the good $i$ : $w_{ij}/x^v_{ij} = w_{ik}/x^v_{ik}$, $\forall j,k \in \{i, \dots n\}$. The analogous assumption is made for all the other variables on the right-hand side of equation \ref{decompositionOutput1}, %Otherwise put, the share of taxes, wages, each of the intermediate input costs (imported and domestically produced), etc. is, for every $i$ the same for all $x_{ij}$, $i$ $\in$ $\{1,\ldots, n\}$. Thus, the elements of $X$ can be decomposed as :
which allows to determine them as : 
\begin{align}
%x_{ij} &= z^{tot,d}_{ij} + z^{tot,m}_{ij} + t_{ij} + w_{ij} + ot_{ij} + gs_{ij} \label{decompositionOutput1} \\
z^{tot,d}_{ij} &= x^v_{ij} \, z^d_j/x_j \\
z^{tot,m}_{ij} &= x^v_{ij} \, z^m_j/x_j \label{totImports} \\
t_{ij} &= x^v_{ij} \, \, t_j/x_j \\
w_{ij} &= x^v_{ij} \, w_j/x_j \\
ot_{ij} &= x^v_{ij} \, ot_j/x_j \\
gs_{ij} &= x^v_{ij} \, gs_j/x_j
\end{align}
In the above equations, for example $z^{tot,d}_{ij}$ represents the total cost of domestically produced intermediate inputs that went into the production of $x_{ij}$. %The interpretation of the other variables on the right-hand side of equation \ref{decompositionOutput1} is similar. 

Summing equation \ref{decompositionOutput1} over $j$, we get :
\begin{equation}
x^v_i - \sum_j z^{tot,d}_{ij}  = \sum_j z^{tot,m}_{ij} + \sum_j \big(t_{ij} + w_{ij} + ot_{ij} + gs_{ij}\big) \label{decompositionOutput2}
\end{equation}
The left-hand side of the above equation is equal to the value of final demand for good $i$, domestically produced $y^{v,d}_i$. The first term on the right-hand side is the importation of product $i$ to be used as intermediate inputs. %$m^{inp}_i$. 
Summing equation \ref{decompositionOutput2} over $i$ and reversing the order of the sums over $i$ and $j$ results in :
\begin{equation}
\sum_i y^{v,d}_i  = \sum_{i}\sum_{j} z^{tot,m}_{ij} + \sum_j \big( \sum_i t_{ij} + \sum_i (w_{ij} + ot_{ij} + gs_{ij})\big) \label{decompositionOutput3}
\end{equation}
The two terms within the second sum over $j$ on the right-hand side of the above equation, $\sum_i t_{ij}$ and $\sum_i (w_{ij} + ot_{ij} + gs_{ij})$ represent respectively net taxes on products of sector $j$ and value added created by the sector $j$, decomposed into wages, other taxes and gross operating surplus. Thus, final domestic demand decomposes into imports, net taxes on products and value added. Adding final demand satisfied through imports $y^m_i$ for products $i$ we can write :
\begin{equation}
\sum_i y^v_i =  \sum_i (\sum_{j} z^{tot,m}_{ij} + y^{v,m}_i) + \sum_j \big( \sum_i t_{ij} + \sum_i (w_{ij} + ot_{ij} + gs_{ij})\big) \label{decompositionOutput4}
\end{equation}
The above equation shows how the revenue gained from final demand decomposes into importation costs and into value added created throughout the economy. Finally, we note the link between the imports stemming from final demand to sector $k$ as decomposed in matrix $M^k$ (equation \ref{IOmodel_Mk}) and as given by the above decomposition of output (equation \ref{totImports}). The total cost of intermediate inputs imported by sector $j$ resulting from demand addressed to sector $k$ can be calculated in two equivalent ways : $z^{tot,m}_{jk} = \sum_i m_{ij}^k$. %how intermediate and final demand for domestically produced goods, and not for imported goods, create value added throughout the economy. Imports represent an outgoing flux for the domestic economy. (This comment is meant as an illustration of the functioning of the economy, not as an argument against importing goods.) 


\subsection{Employment content}\label{ec}

% First explain the ECT matrix. Say that every element is a mix of 2 pieces of information, 1 sector specific row-wise, the second sector-sepcific column-wise. Say that this information will be split, because of effect of averaging. Say that employment content is columns sum, the general form of the decomposition is ... thus summing terms will average out...

Given a final demand scenario $Y$ (or its constituent components \{$C$, $G$, $I$, $\mli{INV}$, $E$\}), equations \ref{IOmodel_Yd} - \ref{IOmodel_Mk} allow to calculate the necessary output and imports by each of the sectors to satisfy demand. In addition, the IO-model can also be used to estimate employment created throughout the economy by a given demand scenario. To capture both the jobs created by final demand $y_j$ %for good $i$ 
within the sector $j$ as well as the jobs created in other sectors, the concept of the employment content (ec) is used. The employment content $ec_j$ of sector $j$ is defined as the number of direct and indirect jobs generated by one unit of final demand addressed to %the production of one unit of output of 
sector $j$. To calculate the employment content of all sectors, define the $n \times n$ matrix $EC$ where $ec_{ij}$  is the number of jobs in sector $i$ created by demand addressed to sector $j$. Define $E$ the $n\times n$ matrix with zeros off-diagonal and with on the diagonal the number of people employed in sector $i$ per unit of the sector's output in physical units, $e_{ii}=ne_i/x^{ph}_i$, where $ne_i$ is the number of employees, in the year of the IOT, in sector $i$. The hypothesis here is that the amount of workers employed in for example the mining sector is determined by the tons of coal mined, and not by the price obtained. Given the final demand matrices $Y^d$ and $Y^m$ as defined in equations \ref{IOmodel_Yd} and \ref{IOmodel_Ym}, the employment content matrix $EC$ is calculated as :
\begin{align}
EC	 &= E\,  \, B^d \, S^d \label{EC} \\ 
B^d &=  (\mathbb{I}-A^d)^{-1} \\
S^d	 &=  Y^d \, (Y^d+Y^m)^{-1}  
\end{align}
%Direct employment in sector $j$ is then defined as $ec_{jj}$ and indirect employment as $\sum_{i, i\neq j} ec_{ij}$. 
The total employment content $ec_j$ of sector $j$ is given by $\sum_i ec_{ij}$. The matrix $B^d$ is the inverse Leontief matrix with elements $b^d_{ij}$. The diagonal matrix $S^d$ with elements $s^d_{ij}$ (with zeros off-diagonal) captures that the more final goods are imported, the less domestic economic activity is created by demand for those goods.
%The employment content is then defined as the number of jobs created through a single unit of final demand, thus by setting $C + G + I + INV + X=\mathbb{I}$, where $\mathbb{I}$ is the identity matrix. 
%\begin{equation}
%	EC	= E\, LI^d \, S^d \label{EC} \\ 
%\end{equation}
%where $S^d = \mathbb{I} - S^m$ contains the shares of domestic output in total supply of goods, i.e. supply through imports and through domestic production, available for domestic use and for exports. In the above equation, $E$ and $S^d$ are diagonal matrices with 0's off-diagonal.

The elements $ec_{ij}$ of EC are composed of three factors, 
\begin{equation}ec_{ij} = e_{ii} \, b^d_{ij} \, s^d_{jj}, \label{EC_decomp}\end{equation}
that capture different aspects of the terms $ec_{ij}$ that make up the employment content $ec_j$. The number of jobs created in sector $i$ due to a unit of final demand addressed to sector $j$ can be high because a) sector $i$ employs many people per unit of output ($e_{ii}$ is high), b) because production of final product $j$ demands many domestically produced direct and indirect inputs from sector $i$ ($b^d_{ij}$ is high), c) final good of type $j$ is primarily sourced through domestic production ($s^d_{jj}$ is high). To further understand the reasons behind the differences in the employment content between sectors, we apply the decomposition of the ec as proposed in \citep{perrier2017transition}. The approach further decomposes firstly the factor $e_{ii}$ to understand some reasons why employment per unit of output is different between sectors, and secondly $b^d_{ij}$ to separate out the effects of the input-intensity of final product $j$ and whether inputs are sourced domestically or imported.

The decomposition of $e_{ii}$ is based on the accounting equations that are present in the columns of the IOT (see figure \ref{IOT_importsUses}) :
\begin{equation}
x_i = z^d_i + z^m_i + t_i + ot_i + w_i + gs_i
\end{equation}
Since total employment $e_i$ is equal to compensation of employees $w_i$ divided by the average wage $\bar{w}_i$, we get :
\begin{equation}
 e_i = (x_i - gs_i - z^d_i - z^m_i - t_i - ot_i )/\bar{w}_i \label{empl_accounting}
\end{equation}
From the above equation can be deduced that employment is high when 1) wages are low, 2) taxes are low, 3) compensation of employees (equal to the nominator) is high, 4) imports of inputs are low and 5) inputs in general are low\footnote{However, high inputs means low employment in the sector in case, but high demand, and thus possibly large employment in sectors supplying to sector $i$.}. To capture this observation, employment per unit of output $e_{ii}$ is decomposed as :
\begin{align}
%E &= \frac{\mli{VA^d}}{Y} \, \frac{\mli{VA}^{HT}}{\mli{VA}^d} \, \frac{\mli{COMP}}{\mli{VA^{HT}}} \, \frac{\mli{NPE}}{\mli{COMP}}   \label{decompE} \\ 
e_{ii} = \frac{ne_i}{x_i}&= \frac{va_i}{x_i} \ \frac{va^{HT}_i}{va_i}  \ \frac{w_i}{va^{HT}_i}  \ \frac{ne_i}{w_i} \nonumber \\
&= \frac{va_i}{x_i} \ t_i \, ls_i \, n_i \label{ECdecomp_e} \\
t_i &= va^{HT}_i/va_i \nonumber  \\
ls_i &= w_i\, /va^{HT}_i \nonumber \\
n_i &= \mli{ne}_i/w_i \  (\, =1/\bar{w}_i)	  \nonumber \\
e_{ij, i\neq j} &= 0		  \nonumber 
\end{align}
%where $\mli{VA^d}$ is, $\mli{VA}^{HT}$, $\mli{COMP}$,$T$, $L$, $N$ and are all diagonal matrices with 0's off-diagonal.
In the above, $va_i$ and $va_i^{HT}$ denote respectively value added including all taxes, $va_i= w_i + gs_i + t_i + ot_i$, and excluding all taxes $va^{HT}_i= w_i + gs_i$.

The factor $b^d_{ij}$ is decomposed by defining the ratios $mi_{ij}$ of the total requirements coefficients, domestically produced, to total requirements, imported and domestic :
\begin{align}
mi_{ij} &= b^d_{ij} / b_{ij} \label{ECdecomp_bd} 
\end{align}
 
Combining equations \ref{EC_decomp}, \ref{ECdecomp_e} and \ref{ECdecomp_bd}, the desired factorial decomposition is, almost, obtained :
\begin{equation}
ec_{ij} = \frac{va_i}{x_i} \ t_i \ ls_i \ n_i \ mi_{ij} \ b_{ij} \ s_{jj}
\end{equation}
The above equation contains two distinct parts. The last three factors % %The factors $va_i\ b_{ij}/x_i$ and $s_{jj}$ 
have an influence on employment through a scale effect : the more output is produced by sector $i$, the more people it will employ.%\footnote{Given that we assume that employment per unit of output is constant per sector.}. 
The three factors $mi_{ij} \ b_{ij} \ s_{jj}$ are characteristics of sector $j$ to which demand is addressed: they capture whether final goods of type $j$ are imported or not, what types of inputs the sector uses --- i.e., inputs of what type and whether they are imported or not. Indeed, inputs might be required that are not available on the domestic market. The first four factors have an influence on employment because of characteristics specific to the producing sector $i$. To finalize the decomposition, the factors representing value added created in sector $i$ because of demand addressed to sector $j$, if all intermediate goods were produced domestically, $\frac{va_i}{x_i} b_{ij}$, is used as a ventilation factor :
\begin{align}
ec_{ij} &= (v_{ij}^{1/4} \ t_i) \ (v_{ij}^{1/4}ls_i) \ (v_{ij}^{1/4}n_i) \ (v_{ij}^{1/4}mi_{ij})  \ s_{jj}  \\
ec_{ij} &=  \ t^v_{ij} \ ls^v_{ij} \ n^v_{ij} \ mi^v_{ij}  \ s_{jj} \label{ECdecomp_ventil} \\
v_{ij} &=  \frac{va_i}{x_i} \, b_{ij}
\end{align}
Next, to compare the employment content of each sector and their decompositions to the average ec and its decomposition, the average $\bar{ec}$ is defined, not by averaging over the $ec_j$'s, but by defining means of the decomposition factors on the right-hand side of 
\ref{ECdecomp_ventil}. The definitions of the means are given in appendix \ref{ECmean}. The matrix of which the columns sums are the mean employment contents of the economy (all column sums are thus equal) $\bar{EC}$ has elements $\bar{ec}_{ij}$ defined as : 
\begin{align}
%ec_{ij} &= (v_{ij}^{1/4} \ t_i) \ (v_{ij}^{1/4}ls_i) \ (v_{ij}^{1/4}n_i) \ (v_{ij}^{1/4}mi_{ij})  \ s_{jj} \label{ECdecomp_ventil} \\
\bar{ec}_{ij} &=  \ \bar{t}^v_{ij} \ \bar{ls}^v_{ij} \ \bar{n}^v_{ij} \ \bar{mi}^v_{ij}  \ \bar{s}_{jj} \label{ECdecomp_ventil_mean} \\
%v_{ij} &= b_{ij} \frac{va_i}{x_i} 
\end{align}
The diagonal and the non-diagonal elements of $\bar{EC}$ are different, all the diagonal elements are equal to one another as are all the non-diagonal elements. The diagonal elements represent the average number of jobs created by demand addressed to sector $j$ within the same sector. The non-diagonal elements represent the the average number of jobs in a sector by demand coming from a different sector. 

To transform the factorial decomposition in an additive decomposition which allows for easy visual apprehension of the data, the logarithmic mean divisia index is used as detailed in \citep{ang2005lmdi}. Taking the logarithm of the ratio of \ref{ECdecomp_ventil} and \ref{ECdecomp_ventil_mean} results in a sum of terms where each term is, for small deviations of the mean, the percent change of $ec_{ij}$ from the mean $\bar{ec}_{ij}$. The same interpretation holds for each of the decomposition factors :
\begin{align}
%ec_{ij} &= (v_{ij}^{1/4} \ t_i) \ (v_{ij}^{1/4}ls_i) \ (v_{ij}^{1/4}n_i) \ (v_{ij}^{1/4}mi_{ij})  \ s_{jj} \label{ECdecomp_ventil} \\
log\bigg(\frac{ec_{ij}}{\bar{ec}_{ij}}\bigg) &=  \ log\bigg(\frac{t^v_{ij}}{\bar{t}^v_{ij}}\bigg) +\ log\bigg(\frac{ls^v_{ij}}{\bar{ls}^v_{ij}}\bigg) +\ log\bigg(\frac{n^v_{ij}}{\bar{n}^v_{ij}} \bigg) +\ log\bigg(\frac{mi^v_{ij}}{\bar{mi}^v_{ij}}\bigg)  +\ log\bigg(\frac{s_{jj}}{\bar{s}_{jj}}\bigg)  %\label{ECdecomp_ventil_final}
%v_{ij} &= b_{ij} \frac{va_i}{x_i} 
\end{align}
Scaling the above additive composition leads to the final decomposition equation that sheds a light on why all $ec_{ij}$'s are different, by comparing them to the means $\bar{ec}_{ij, i\neq j}$ and $\bar{ec}_{jj}$. The decomposition of the deviation of the total employment content $ec_j$ to the economy's mean $\bar{ec}$ is then given by summing over $i$. 
\begin{align}
%ec_{ij} &= (v_{ij}^{1/4} \ t_i) \ (v_{ij}^{1/4}ls_i) \ (v_{ij}^{1/4}n_i) \ (v_{ij}^{1/4}mi_{ij})  \ s_{jj} \label{ECdecomp_ventil} \\
ec_j - \bar{ec} &= \sum_i \big( ec_{ij} - \bar{ec}_{ij} \big) \nonumber \\
&= \sum_i \frac{ec_{ij} - \bar{ec}_{ij}}{log(\frac{ec_{ij}}{\bar{ec}_{ij}})}\ \Bigg( log\bigg(\frac{t^v_{ij}}{\bar{t}^v_{ij}}\bigg) +\ log\bigg(\frac{ls^v_{ij}}{\bar{ls}^v_{ij}}\bigg) +\ log\bigg(\frac{n^v_{ij}}{\bar{n}^v_{ij}} \bigg) +\ log\bigg(\frac{mi^v_{ij}}{\bar{mi}^v_{ij}}\bigg)  +\ log\bigg(\frac{s_{jj}}{\bar{s}_{jj}}\bigg) \Bigg)  \label{ECdecomp_ventil_final}
%v_{ij} &= b_{ij} \frac{va_i}{x_i} 
\end{align}
To obtain detailed information on the type of employment created by final demand addressed to a specific sector $j$, one can both look at the decomposition of the difference between the total employment $ec_j$ and the mean $\bar{ec}$ as well as to the decomposition of the individual terms of equation \ref{ECdecomp_ventil_final}, i.e. before summing over $i$. There are several reasons for this. A first reason is that the total requirements coefficients on the diagonal $b_{jj}$ are typically much larger than off-diagonal $b_{ij, i\neq j}$. Thus, in the sum over $i$, the diagonal elements will tend to dominate the off-diagonal elements. In addition, the sum over $i$ off the off-diagonal elements will do exactly what an average does, it is average out. For example when, on average, the labour share of the jobs created throughout the economy by final demand for product $j$ is close to the mean, this can be due to the fact that this is true for all the sectors in which jobs are sustained, or it can be due to a situation where part of the jobs are in sectors with an above average labour share and part with a below average labour share. This effect will further contribute to the dominance of the diagonal over the off-diagonal elements. %Hence, summing provides a correction to the masks the type of jobs created by demand for product $j$ throughout the economy by the jobs sustained in sector $j$ itself.  
To characterize what types of jobs are created throughout the economy, when applying the above decomposition to the South African economy, we will provide separately the decomposition of firstly the jobs created within the sector to which final demand is addressed, $ec_{jj} - \bar{ec}_{jj}$ and secondly and thirdly of all the jobs in sectors with above and below average $ec_{ij}$, i.e. for $i\neq j$, $\sum_i (ec_{ij} - \bar{ec}_{ij})$, with $ec_{ij} - \bar{ec}_{ij} >0$ and with $ec_{ij} - \bar{ec}_{ij} <0$.

%To show these effects, the method proposes a factorial decomposition of the employment content of each sector. The employment content of sector $i$ is the sum over all sectors of the employment (number of jobs) that is created in each one by a unit of final demand addressed to sector $i$. However, the reasons for high/low employment as evoked by equation \ref{empl_accounting} are sector-specific. Thus, when we sum over the whole economy, in some sectors wages will be above the mean and in others below ; thus, summing will tend to average out deviations from the mean. Thus, when the wages component is close to average of the employment content, this may be because is some sectors wages are above and in others below average. Or, because indeed, demand from sector $i$ generated activity in sectors with mostly average wages.
%
%For this reason, we will rather decompose employment in each sector, as inspired by \label{empl_accounting}, and 
%
%
%decomposes the difference between each sectors total employment and the economy's mean EC content into a sum of five terms. Each of the terms sheds a light on whether the employment content is above or below average because 1) wages are low/high, the labour share is high/low, taxes are 
%The approach we take here is slightly different from 
%
%Employment per unit of output is decomposed as :
%\begin{align}
%	E &= \frac{\mli{VA^d}}{Y} \, \frac{\mli{VA}^{HT}}{\mli{VA}^d} \, \frac{\mli{COMP}}{\mli{VA^{HT}}} \, \frac{\mli{NPE}}{\mli{COMP}}   \label{decompE} \\ 
%	&= \frac{\mli{VA^d}}{Y} \, T \, L \, N \\
%	T &= \\
%	L &= \\
%	N &= 
%\end{align}
%where $\mli{VA^d}$ is, $\mli{VA}^{HT}$, $\mli{COMP}$,$T$, $L$, $N$ and are all diagonal matrices with 0's off-diagonal.
%
%The domestic leontief inverse is decomposed into the total leontief inverse and factors that reflect the import rates of intermediate inputs : 
%\begin{equation}
%	LI^d = \mli{LI} \odot \mli{MI} \label{decompLId}
%\end{equation}
%where $\odot$ designates the element wise Hadamard multiplication of matrices.
%
%Combining \ref{EC}, \ref{decompE} and \ref{decompLId}, brings us close to the final decomposition that will be used :
%
%\begin{equation}
%EC	= \frac{\mli{VA}^d}{Y} \, T \, L \, N \, \mli{LI} \odot \mli{MI} \, S^d,   \label{decompEC1} \\ 
%\end{equation}
%Again, note that the matrices with possibly non-zero elements off-diagonal are $EC$, $LI$ and $MI$. Next QQ sets :
%
%$$ V = \frac{\mli{VA}^d}{Y} \mli{LI}$$ 
%
%such that :
%\begin{equation}
%EC	= T \, L \, N \, V \odot \mli{MI} \, S^d,   \label{decompEC1} \\ 
%\end{equation}
%which can equivalently be written as :
%
%\begin{equation}
%EC	= \bar{T} \odot \bar{L} \odot \bar{N} \odot V \odot \mli{MI} \odot \bar{\bar{S^d}},   \label{decompEC2} \\ 
%\end{equation}
%
%where the operation denoted by the single bar $\bar{\ }$ is right multiplication with $\mathbb{J}$ the $n\times n$ matrix with all ones, $\bar{T} = T \, \mathbb{J}$, and the operation denoted by the double bar $\bar{\bar{\ }}$ is left multiplication with $\mathbb{J}$, $\bar{\bar{S^d}} = \mathbb{J} \, S^d$. 

\subsection{Cost-push model}

\li{Check notation}The input-output table can also be used to determine, when the price of an input $i$ changes with respect to a base scenario, what the impact would be on the prices of all the other sectors, assuming that the increased cost is fully transferred to buyers. This implies that the profitability of each sector is not impacted and that they still use the same quantities of all inputs. 

Given a reference IOT measured in value, then for the base scenario, prices $p_i$ for all sectors $i$ can be fixed at 1, such that that the IOT in quantities has the same numerical values as the one in value. For the base scenario, the vector of all prices is written as $\boldsymbol{p}$ (and is thus a vector of all 1's).

If $Z$ is the inter-industry matrix (in quantity) of size $n\times n$ with all sectors, then we denote $Z^{-i}$ the inter-industry matrix of size $n-1 \times n-1$ where the row of inputs from sector $i$ to other sectors and the columns of uses of $i$ are omitted. Denoting the $1 \times n-1$ row of inputs of sector $i$ to other sectors except for to itself as $(inp^{-i}_i)^T$, the accounting equations that follow from the first $n$ columns of the IOT then read :

\begin{equation}
(p^{-i})^T \, Y^{-i} = (p^{-i})^T Z^{-i} + (inp^{-i}_i)^T + (va^{-i})^T + (t^{-i})^T \label{cost-push1}
%(y^{-i})^T =  Z^{-i} + (inp^{-i}_i)^T + (gva^{-i})^T + (t^{-i})^T
\end{equation}

where $Y^{-1}$ is the diagonal $n-1 \times n-1$ matrix of output, measured in quantities, and $(p^{-1})^T$, $(va^{-1})^T$  and $(t^{-1})^T$ are the $1 \times n-1$ row-vectors of respectively prices, gross value added a net taxes on products. Right-multiplying equation \ref{cost-push1} with the diagonal matrix with the inverse of total output measured in quantities on the diagonal and 0's elsewhere, $\underline{Y}^{-i}$,we get :
\begin{equation}
(p^{-i})^T  = (p^{-i})^T A^{-i} + \Big((inp^{-i}_i)^T + (va^{-i})^T + (t^{-i})^T \Big)\, \underline{Y}^{-i} \label{cost-push2}
\end{equation}
and equivalently :
\begin{equation}
(p^{-i})^T  =  \Big((inp^{-i}_i)^T + (va^{-i})^T + (t^{-i})^T \Big)\, \underline{Y}^{-i} \, (I - A^{-i})^{-1} \label{cost-push3}
\end{equation}
where the vector $p^{-i}$ contains all 1's.

Consider an alternative scenario $A$ where the price of input $i$ changes by $\lambda\%$ for all sectors with respect to the base scenario, then the value of inputs $inp^{A^{-i}}_{i} = (1+\lambda/100)\, inp^{-i}_i$. The resulting prices in all the other sectors are given by :
\begin{equation}
(p^{A^{-i}})^T  =  \Big((inp^{A^{-i}}_i)^T + (gva^{-i})^T + (t^{-i})^T \Big)\, \underline{Y}^{-i} \, (I - A^{-i})^{-1} \label{cost-push-scen}
\end{equation}

This price increase can be interpreted as a measure for how the total cost of the price change is distributed over all sectors. In addition, the cost to final consumers can be determined. 

%\subsection{Demand scenarios}

%The actual demand scenarios used to estimate the effect of different export values in 2035 are as follows. For the BAU demand scenario, demand as in the South-african economy in the year 2014, as provided by the IOT, is used. As demand scenario for 2DS, the same as BAU is used with the sole difference that the volume of coal exports is determined such that the ratio of the BAU versus the 2DS volume is equal to the ratio of the volumes specified by CPI's scenarios.


\section{Datasources}

The data used for the present study are obtained from Statistics South Africa (Stat SA), the South African national statistical agency, and from the OECD. The latter are in itself based on the data collected by Stat SA. The latter offer additional information because of the treatments data underwent (aggregation, seasonal adjustment). %and because the data for South Africa have been put in relation to data  are however useful in some instances because they offer access to different types of pre-treatments . 

\subsection{IOT}	

To take into account that the demand for imported goods does not create value or give rise to jobs domestically through their production, separate IOTs for domestic output and imported goods are needed. The most recent input output table available from Stat SA is for the year 2014. The level of detail is at 50 sectors. However, no table is available that specifies uses of imports versus uses of domestic production. Such a table is provided by the OECD, at a detail of 35 sectors ; the most recent one is for 2011 at the time of writing.

To construct a separate IOT for domestic production and for imports, the 2011 OECD tables were used to estimate such tables at the 50-sector level for the year 2014. This was done using an optimization procedure, detailed in appendix \ref{IOTopti}.

%\footnote{The OECD provides such tables at a level of detail of 35 sectors, the most recent one for the year 2011. The South African statistical agency provides a table with a level of detail of 50 sectors, the most recent for the year 2014, but that does not distinguish the use of imports.}. 

\subsection{Employment data}
Employment data of the same sectoral aggregation as the 50-sector IOT were constructed on the basis of two sources of employment data provided by the South African statistics agency. In addition more detailed data for certain sectors, all collected by and available from Stat SA, were used.
 
Data on employment are gathered by Stat SA through two channels. The Quarterly Employment Survey (QES) is conducted with enterprises \citep{QES2018}. The Quarterly Labour Force Survey (QLFS) is a survey taken from households from individuals aged 15 years and over. Contrary to the former, the data thus obtained cover also the agricultural sector, self-employed workers whose businesses are unincorporated, unpaid family workers, and private household workers among the employed \citep{QLFS2018}.

The QES offers employment numbers for the whole economy, excluding agriculture and private households, at a level of detail of 91 sectors. The QLFS offers data at a disaggregation of the economy of 10 sectors. Thus, while the QLFS covers more types of workers, the QES is more detailed.

The IOT, although on average less detailed than the employment data sourced from the QES, offers for a few sectors a different or more detailed disaggregation than the 91 sectors used by the former. This is the case for "Agriculture" (SIC 1), the "Mining and quarrying" sector (SIC 2), and for "Coke oven products; petroleum refineries; processing of nuclear fuel" (SIC 33). To obtain estimates compatible with the IOT, assumptions and additional data are used. Also, to obtain final numbers, the choice is made to include all types of workers, self-employed workers whose businesses are unincorporated, unpaid family workers, and private household workers. The last assumption implies that the final data set results in the same numbers when aggregated to the 10 sectors of QLFS. 

To obtain data for agriculture, the QLFS estimate is used, disaggregated to Agriculture (SIC 11), Forestry (SIC 12), Fishing (SIC 13) according to their shares in output for the year 2014. 

The employment data for the mining sector are disaggregated into gold and non-gold in the QES. To get numbers for the same disaggregation as the IOT table (SIC 21 coal, 23-24 gold and metals, 25 other), data are used from the 2015 report from Stat SA on the Mining industry \citep{mining2015}. In the report, both direct employees of the mining industry are included, as employees through labour brokers, of subcontractors and capital employees\lies{what are those?}.%, for xx subsectors of the mining industry. %The total thus obtained in the report is 490146, versus 459271 in the quarterly employment series 2009-2018. Also, the proportion working for the gold mines SIC 23 is different. In the quarterly series, 21\% of people in the mining sector work for gold, versus 25\% according to the survey.
To obtain an estimate for the number of people working in SIC 21, 23+24 and 25, the non-gold share of the quarterly series is disaggregated according to the shares of the mining report. 

To obtain the disaggregation of sector 33 as in the IOT, it is assumed that shares of workers employed in each subsector are equal to the shares of output.

Finally, after the QES data are aggregated into 50 sectors, volume adjustments are made based on the QLFS numbers.

%The conversion between SIC and ISIC rev.3 https://www.statssa.gov.za/additional_services/sic/descrip6.htm
%https://www.statssa.gov.za/additional_services/sic/preface.htm

Employment data are also available from the OECD at a level of 7 sectors. These allow to make a final check of the obtained data (figure \ref{ComparisonOECD-STATSA-Empldata_2014}). The main difference found is between sectors ISIC rev.3 G-I and L-P. G-I covers Wholesale and Retail trade and L-P service activities among which Private Households.\li{Question sent, no answer yet.}

\begin{figure}[!h]
	\centering
	\includegraphics[width=0.8\textwidth]{../graphs/ComparisonOECD-STATSA-Empldata_2014.pdf}
	\caption{\label{ComparisonOECD-STATSA-Empldata_2014}Comparison of data sources on the number of persons employed per sector in South Africa, 2014.}
\end{figure}


%\subsubsection{Scenarios}

\

%\section{Results}
%\section{Descriptive statistics of the data used / macro description of the South African economy}
\clearpage
\section{Sectoral employment contents and their decomposition}\label{ec_results}

%Figures \ref{Decomposition_direct_indirect_avg} and \ref{Decomposition_direct_indirect_absolute} shows the total employment content --- the number of persons employed to generate 1 million rand of final output --- for each of the 50 sectors of the IOT, decomposed into jobs that are directly generated within the concerned sector ("Direct" in red), and jobs that are situated in other sectors, due to demand stemming from the concerned sector for intermediate products ("Indirect" in blue).

Figure \ref{Decomposition_PQ} shows the result of the decomposition of the employment content (ec) of each sector. %as specified in section \ref{ec}. 
On average in the economy, 2.24 jobs are sustained by 1 million Rand of final demand. Of these, 1.30 are on average situated within the sector to which demand is addressed and 0.94 in the other sectors (figure \ref{Decomposition_direct_indirect_absolute}). Overall, it can be seen that low/high wages and a high/low labour share are the main drivers for an above/below average employment content. Examples of sectors where employment is higher than average because of these drivers are Agriculture, Forestry and fishing, Construction, Trade, Hotels and restaurants, Computer activities, Education, Health and social work and Other community activities. 

In some sectors, the employment content is lower than average because of a low labour share and high wages. This suggests that the higher capital intensity implies that skilled workers are needed who are paid above average wages. Examples are the mining sectors Coal and lignite, Metal ores and Other mining, Beverages and tobacco, Coke oven manufacture, which includes the distillation of crude oil and the coal-to-liquids operation of Sasol, the utilities sectors Electricity, gas and water and the Distribution of water, Transport, Telecommunications, Financial intermediation, Insurance and pensions, Research and Other services nec. The manufacturing sectors as a group tend to have a below average employment content (figure \ref{Decomposition_direct_indirect_absolute}). In manufacturing, the drivers of a low ec are above average wages and the higher than average imported share of both final goods and inputs. 

%Figure \ref{Decomposition_direct_indirect_absolute} shows that about as many jobs are created within the coal sector as in the rest of the economy by a unit of final demand for coal. This means that for every job created or destroyed in the mining sector because of changes in final demand, another job is created or destroyed throughout the economy, in other sectors. This contrasts with the findings of \cite{ChamberofMines2018Strategy}, who find that in 2015, 173.093 jobs were created in the rest of the economy, in addition to 77.747 directly in the coal sector (source : \cite{MineralsCouncil2018Facts}). The coal sector thus emerges as a sector where wages are relatively high, but as a below average job creator, measured in number of jobs.

%These figures are specific for the year 2014, with 2014 export and domestic coal prices. Employment in the coal sector can be thought of as more determined by the tons of coal mines than by the price obtained. Thus, the employment content of the coal sector might fluctuate from year to year along with the average price obtained. This does not however alter the qualitatively overall trend that the sector's ec is low, mainly because of the sector's capital intensity.

%In figures \ref{Decomposition_PQ}, \ref{Decomposition_PQ_within} and \ref{Decomposition_PQ_other}, the difference between the employment content (EC) for each sector and the average employment content over the economy is visualized. In figure \ref{Decomposition_direct_indirect_avg},  the EC is decomposed into its direct and indirect components, and how those relate to the average number of jobs directly and indirectly generated. In figure \ref{Decomposition_PQ}, total employment content is decomposed according to eqn. \ref{}, specifying whether high/low employment in a particular sector is mainly due to low/high wages, high/low domestic demand, etc.

%The mining industry generates less jobs per unit of final output than the economy's average. Both the mining of coal and lignite and of metal ores generate as many jobs through directly as indirectly per unit of final demand. The number of indirect jobs created is close to the average. It is the amount of directly generated jobs that is well below the average. Figure \ref{ECdecomposition-avg-PQ_2014.pdf} shows the reasons behind the lower-than average employment content. A low labour share and high wages are the main reason ; there is a positive contribution from	the factor designated by "local final demand" which contributes positively to the labour share when imports are small or when exports are large. Both are the case for the two mining sectors. In Other mining (25), which includes the import of crude oil, there is a similar contribution to a low employment content from a low labour share and higher-than-average wages ; additionally, due to the large share of imports in this sector, the employment content sinks even further below the economy's average. 


\begin{figure}[!ht]
	\centering
	\includegraphics[width=0.95\textwidth]{../graphs/ECdecomposition-avg-PQ_2014.pdf}
	\caption{\label{Decomposition_PQ} \footnotesize The horizontal bars show for each sector a decomposition of the total employment content into five factors, compared to the average composition of the employment content over the whole economy, excluding SIC 01/ISIC P Private households with employed persons. The factors give an indication of the sources of the differences in employment content. The average employment content is 2.24 jobs/million Rand. The components of a horizontal bar situated to the left/right of the 2.24 mark are smaller/bigger than the average. The interpretation of each of the five components is as follows : 1) when "1/Wages" is bigger than the average, this means that wages are smaller in this sector than the average wage throughout the economy, 2) when "Labour share" is bigger than the average, the sector's labour share is bigger than the average, 3) when "1/Taxes" is bigger than the economy's average, the tax burden on the sector is smaller than the average, 4) when "Local demand inputs" is situated on the left, then the importation of intermediate inputs is higher than the economy's average, 5) when "Local final demand is smaller than the average, then local demand is relatively smaller and imports are relatively bigger.}
\end{figure}	


\begin{figure}[!ht]
	\centering
	\thispagestyle{empty}
	\includegraphics[width=\textwidth]{../graphs/ECdecomposition-direct-indirect_2014.pdf}
	\caption{\label{Decomposition_direct_indirect_absolute} The length of the horizontal bars shows the employment content in absolute terms for the South African economy decomposed into 50 sectors. %The employment content is measured as the number of jobs created, throughout the economy, per million rand of final demand. 
		Total employment content is decomposed into jobs, created within each sector to which final demand is addressed, and jobs sustained in all the other sectors due to demand for inputs. The dotted gray lines indicate the average number of jobs created within the sector to which 1 million Rand of final demand is addressed (1.30) and the total employment content (2.24).}
\end{figure}	


\clearpage
\section{Key sectors}
%The difference in direct and indirect effects generated by the 2DS versus the BAU scenarios will be estimated with the the aid of the most recent input-output table (2014). Throughout the analysis, it will be assumed that the inter-dependencies between the different sectors in the economy --- the value of input bought by sector $i$ from $j$ to produce 1 mR of output by $i$ --- are assumed to be as in the year 2014. %In a few (1 ?) cases,  
%
%In reality, this assumption does not hold. The 2014 IOT used is measured in value and its numbers might change both because of changes in the relative price of commodities and because over time sector's systems of production change, as reflected in and defined by the amount of inputs used per unit of output. For example, the cost of coal use for 1 $kWh$ of electricity can change on the one hand because the domestic price of coal changes, without changing the $kg$s used, or because solar panels replace coal.
%
%The scenario comparisons in section \ref{scenarioAnalysis} cover the period 2018-2035, over which both relative prices and systems of production will have changed, both under BAU as under 2DS assumptions. This study will not attempt to estimate how the IOT might change over the modeling period ; the assumption is that the 2014 IOT is a good approximation. In order to get picture of the information the 2014 IOT provides, the following subsections describe key sectors for transition scenarios, their place within the economy and how they are linked to other sectors. This will give a first idea of the dominant risks, via indirect effects, of the external transition scenarios. Information on the two types of transmission channels is set forth : how changes in one sector impact others because they use the former's commodity as input and secondly because it uses inputs from others. 
%

\subsection{Coal}

With 9,9 million tons of proven coal reserves at the end of 2017, South Africa has 1.0\% of the world's proven reserves and ranks as such 12th, after the United States who possess 24.2\%, the Russian federation 15.5\%, Australia 14.0\%, China 13.4\%, India 9.4\%, Germany, Ukraine, Poland, Kazakhstan, Indonesia and Turkey \citepalias{british2018bpcoal}. %\footnote{In the 2018 BP Statistical Review of World Energy, South Africa's proven coal reserves were revised downwards to  million tons (for hitherto unknown reasons). Hence, at the end of 2017, SA ranks 12th, .}. 
This is down from 2006 when it ranked 6th in the list with 48,8 million ton, reflecting mainly initial overestimation and little investment. Extraction has increased from about 220 million tons in 2001 to 250 million tons in 2007. Since, total production has remained stable between 245 and 260 million tons per annum.  % a note on exploration from ... department of energy coal report % note on where the coal is situated and how much

Over the period 2001-2017, $28\%$ $\pm$ $2\%$ of coal sales were exported and the remainder was consumed domestically \citepalias{MineralsCouncil2018Facts, MineralsCouncil2010Facts}. Domestic use is concentrated in electricity production and in the production of liquid fuel by Sasol (cf. table \ref{CoalUse2014}, Sasol is situated in the Coke oven manufacture sector). South Africa has no own petroleum resources and produces about 28\% of its liquid fuel demand from coal \citep{coal2013DoE}, importing the remainder. 

The share of exports in total coal production is much higher in value terms than in volume. Again over the period 2001-2017, the export coal accounts for $53\%$ $\pm$ $6\%$ of the revenue obtained from coal mining. The export price is thus much higher per ton than the domestic cost of coal.

In 2017, export and domestic demand for coal contributed about 2.4\% to GDP, counting both the value created within the coal industry as well as in the chain of directly and indirectly supplying sectors\footnote{This figure is calculated as the total sales of coal, from which an estimation of imports is deducted compared to GDP.}. In 2014, of total coal sales, 13\% was used for imports of intermediate products, 60\% was distributed as value added in the coal industry and 25\% in the rest of the economy.

On average in the economy (in 2014), per Rand of final demand, % 33\% is spent on imports\footnote{This figure on imports seemilingly exaggerates the import intensity of the SOuth African  } of final and intermediate goods, 
37\% is distributed as value added within the sector to which demand is addressed and 29\% to other sectors. The coal sector thus emerges as a sector that generates less activity throughout the economy than on average. This is because the sector is highly capital intensive. Indeed, on average in the economy, 21\% of a sector's output is distributed as gross operating profit to investors, whereas in the coal sector this figure is 41\% (Source : 2014 IOT).

%Domestically, coal is used mainly for electricity generation (~ x\% ) as well as for producing liquid fuel (~ 20\%) some coking coal for the production of iron (~x\%) \citep{}. In 85-90\% of electricity is produced with coal. 

\begin{table}[!t]
	\centering
	\renewcommand*{\arraystretch}{1.15}
	\begin{tabular}{lp{5pt}lccc}
		\toprule
		\multicolumn{3}{l}{\multirow{2}{120pt}{Coal and lignite use}} &\multicolumn{1}{c}{\makecell{million tons \\2014}} \\ 
		\midrule
		\multicolumn{3}{l}{Export}  		 & \ 75.8  \\ 	
%		\multicolumn{3}{l}{Household use}  &\ \ 3.5  \\ 
		\multicolumn{3}{l}{Household use}  &\ \ 2.2  \\ 
%		\multicolumn{3}{l}{Inventories}  & \ 10.0  \\ 
		\multicolumn{3}{l}{Inventories}  & \ \ 6.2  \\ 
		\multicolumn{3}{l}{Industry}  & 176.5  \\ 
%		& 						 & Electricity, gas and water &\ 97.6 \\ 
		& 						 & Electricity, gas and water &\ 110 \\ 
%		& 						 & Coke oven manufacture & \ 31.1 \\ 
		& 						 & Coke oven manufacture & \ 40.0 \\ 
%		& 						 & Metal ores &\ 10.6\\
		& 						 & Metal ores &\ 6.6\\
%		& 						 & Paper &\ \ 7.6 \\  
		& 						 & Paper &\ \ 4.7 \\  
%		& 						 & Basic iron and steel &\ \ 7.4 \\  
		& 						 & Basic iron and steel &\ \ 4.6 \\  
		& 						 & Remaining sectors &\  10.6 \\  
		\bottomrule
	\end{tabular}
	\caption{\label{CoalUse2014}The use of \emph{Coal and lignite} (SIC 21) in the South African economy. Data for Electricity and in million tons, obtained from the 2014 IOT, assuming a single domestic price per ton and a single export and import price. Sources : \citetalias{IOT2014} and \citetalias{MineralsCouncil2018Facts}.}
\end{table}


\subsubsection{Interaction between the domestic and export markets}\label{DomesticExport}

\begin{figure}[!t]
	%\centering
	\hspace{-20pt}\includegraphics[width=1.1\textwidth]{../graphs/ExportDestinations_Coal_Value_tons.pdf}
	\caption{\label{ExportDestinations_Coal}\small Export destinations of South African coal in US\$ for the years 1974 - 2016 in SITC classification (coloured bars and left y-axis) and export volume in million tons for the years 2007-2017 (black lines, right y-axis). Sources : Value in US\$ of South African coal export by trade partner : The Center for International Data from Robert Feenstra \citep{feenstra2005world} (1974-2000) and \href{https://comtrade.un.org/}{UN Comtrade} (2001-2016), both obtained from the \href{https://atlas.media.mit.edu/en/visualize/line/sitc/export/zaf/show/3222/1962.2016/}{Observatory of Economic Complexity} ; export of South African coal in tons, \citetalias{MineralsCouncil2018Facts}.}
\end{figure}

As a coal exporter, South Africa is with 3.7 billion US\$ in 2016 one of the majors together with Australia (27 billion US\$ in 2016), Indonesia (12 billion US\$), Russia (7.7 billion US\$), USA (4.4 billion US\$) and Columbia (4.4 billion US\$) (UN comtrade). Coal is an important source of foreign currency for South Africa. %The course of the Rand against the dollar is an important factor in the value of foreign currency obtained through exports. 
As can be seen in figure \ref{ExportDestinations_Coal}, whereas the volume of coal exported is stable since 2011 fluctuating above 70 million tons, the value obtained in US\$ has fallen along with the devaluation of the Rand against the US dollar from 7.3 R/US\$ in 2011 to 14.7 R/US\$ in 2016 (World Bank\footnote{Official exchange rate, LCU per US\$, period average, International Monetary Fund.}). 

Historically, South Africa exported most of its coal to Europe, with additional export destinations in Asia and the Middle East. Export to Europe reached a peak (in value terms) in 2008 and has since collapsed. Around the same time, export to India has increased to pick up the lost demand from Europe. The total volume of coal exported is constrained by the capacity of the railway, Richard Bay's Coal Line, that connects exporting mines to the Richard Bay's Coal Terminal. The designed annual capacity of the terminal is 91 million tons per year.

The displacement of exports from Europe to Asia has had an impact on the domestic market. European demand was for coal of high calorific value (CV) and obtained a higher price \citep{burton2018coal}, whereas domestically lower CV coal is used for electricity generation. Both markets were thus not in competition. Asian demand is however for low CV coal and is in competition with the local market. This has contributed to a higher ratio of the domestic price versus the export price which has risen, with significant annual fluctuations, from 26\% in 2001 to 44\% in 2017, hitting 50\% in 2015. Importantly, it is one of the factors in the increase of the electricity price in South Africa. Because of historic access to cheap coal, 85 - 90 \% of electricity is generated in coal plants, by state-owned Eskom.

The competition between the domestic and the export market plays in practice out mainly through two channels. In multi-product mines tied to the electricity producer Eskom, the revenue from exports would sustain low-cost coal for plants and cheap electricity (cost-plus contracts).
%A first channel passes by multi-product mines that source both coal of both lower and higher calorific value. %and specifically for the cost of electricity production  through the mines that have both access to the export market, via th Richards Bay railway, and the domestic market. A system was in place where mines would export high CV coal and sell low CV coal to power plants run by state-owned Eskom. The exported higher-grade coal would thus sustain a low domestic coal and electricity price. The switch of exports from high CV to low CV coal causes coal power plants to be in competition with the export market. The mines that have access to the main exporting port, Richard Bay terminal, prefer to export their low CV coal. This is one reason why the average domestic coal price has soared from 107 R/ton to 381 R/ton \citepalias{MineralsCouncil2018Facts}.
A second channel is constituted of mines that have access to the export market, via the railway to the RBCT and also supply the domestic market, and where Eskom does not have contractual priority. These mines will supply the export market first, causing Eskom to pay a higher price.

Other factors of the increase in the electricity price are under-investment in mines tied to power plants and procurement through short and medium term contracts from other smaller mines. As a result transport occurs increasingly by truck, which significantly increases transport costs to the plant. A major pressure on the cost of electricity production unrelated to procurement of coal, are investment needs, both to comply with environmental regulations as to maintain old plants, such that they can reach their programmed end-of-life. Only two existing plants are less than 30 years old \citep{burton2018coal}.

Sasol, the coal-to-liquid operator, has been shielded from the price increases since it operates its own mines. %; it invested in own mines ... etc. 

\subsubsection{Employment in the coal industry}
%\citep{holz2018peakcoal}

The number of persons employed in coal mining was 82,248 in 2017.  Over the past ten years, employment in the sector has risen from 60,439 in 2007 to more than 88,000 in 2013 \citepalias{MineralsCouncil2018Facts}. In addition, demand for coal also creates jobs in other sectors of the economy via procurement relationships. In 2014, for every person employed in the coal sector, roughly speaking, another person was employed in another sector in the economy because of demand for coal. A third of the jobs thus created via the procurement relationships are situated in the transport sector (trucking and rail). However, compared to the economy's averages, less jobs are created through coal demand within the sector as well as in other sectors per million Rands of final demand (cf. section \ref{ec_results} and figures \ref{Decomposition_PQ} and \ref{Decomposition_direct_indirect_absolute}). %\lies{add the number of jobs created throughout the economy by coal demand in 2014 of which mainly in the transport sector}
From the macroeconomic perspective, the coal sector is hence, compared to the South African economy in general, not a big generator of jobs\footnote{The jobs sustained by the coal sector are however regionally concentrated.}. This is in part because the sector is capital intensive. Gross operating surplus represented in 2014 $69\%$ of the sector's value added, compared to the economy's weighted average of 46\%. The labour share in the coal mining sector is thus relatively low.
%The numbers given above established that the sector is capital intensive, consequently, the labour share is relatively low, explaining the much lower contribution of coal export to employment than to GDP. 

%However, why is the amount of jobs created in other parts of the economy also below average? 
Figure \ref{ECdecomposition-coal_2014} sheds a light on why the amount of jobs created in other parts of the economy is also below average. %on explains in more detail why the number of jobs sustained by coal export is lower than the average numbers of jobs sustained by final demand for other types of goods. 
The figure shows the total employment content (which is the number of jobs created per unit of final demand addressed to a sector) of the coal sector and its decomposition, compared to the economy's average. %The detailed methodology is given in section \ref{ec}. %We here recapitulate how to read the graph.
%The reason the risk posed by a reduction of coal export is spread differently throughout the economy in terms of value added versus in terms of employment, can be deduced from figure \ref{ECdecomposition-coal_2014}. 
The decomposition of the employment content is performed in two ways. The total employment content of the coal sector is decomposed firstly according to in which sectors, listed on the graph's vertical axis, the jobs are created. The sectors are divided into three groups : (1) the coal sector itself and (2) \& (3) the directly and indirectly supplying sectors. This latter group is subdivided in two, according to whether the number of jobs created in the supplying sector via demand for inputs to coal mining is higher/lower than the average number of jobs created by demand for inputs (generated by one unit of final demand). For each of these three groups, the net length of the corresponding horizontal bars says how many more or less jobs are created in the group of sectors (compared to the average number of jobs created in a group of sectors of equal size). 

Then, the second way in which the employment content is decomposed, is within each of the three groups. The horizontal decomposition  links final demand to employment, passing by imports of final goods and inputs, taxes, the labour share and wages. Indeed, for a given level of final demand, employment in all participating sectors will be higher, the more demand is supplied through domestic production, the more inputs are supplied domestically, the lower taxes are, the higher the labour share and the lower wages. Note that how much final demand is domestically produced vs. imported is a characteristic of the sector where demand originates and of which the employment content is determined ; all the other factors (taxes, \ldots) are characteristics of the supplying sectors, where the jobs are located. Employment in a supplying sector is also positively correlated with demand for inputs (direct and indirect) to the sector, and additionally, when value added is high. These two drivers are distributed as a scaling factor over wages, taxes, labour share and local demand for inputs. 
%Thus, when the net length of the horizontal bars is positive, the number of jobs created by  The net length of the horizontal bars of the graph compare the number of jobs created by a unit of final demand in the coal sector, to the average of the number of jobs created in the economy by a unit of final demand per sector. The net length of the bars is decomposed into different terms that shed light on the reasons the employment content is higher or lower than the economy's average.

\begin{figure}[!t]
	\centering
	\hspace{-16pt}\includegraphics[width=0.9\textwidth]{../graphs/ECdecomposition-coal_2014.pdf}
	\caption{\label{ECdecomposition-coal_2014}\small Comparison of the number of jobs created throughout the economy by 1 million rand of final demand for coal, to the average number of jobs created by 1 million rand of final demand. When the net length of the horizontal bars is positive/negative, job creation in the related industries is above/below the economy's average. The decomposition indicates why. For example, for the top group of sectors (Transport, Trade, \ldots), local demand for intermediate and final products is above average, whereas for the second group (Manufacturing, \ldots), local demand for inputs is below the economy's average. "1/Wages" has to be read inversely : for "Coal and lignite", wages are higher than the average in the economy which contributes to a lower than average number of jobs created by one unit of final demand. }
\end{figure}

%On average in the South African economy (in 2014), 2.24 jobs are created throughout the economy for one million rand of final demand, of which 1.30 are situated within the sector to which final demand is addressed and 0.94 in the rest of the economy. 
The employment content of the coal sector is obtained by summing the net lengths of each of the three horizontal bars of graph \ref{ECdecomposition-coal_2014} and equals 1.52, of which 47\% is within the coal sector --- 0.72 --- with 0.80 jobs spread throughout the economy. %Whereas the ec of the coal sector is below the economy's average (2.24), 
The latter group of jobs are created in sectors with different characteristics. 42\% of the jobs are situated in sectors where the number of jobs per unit of output is higher than the economy's average. This is the case for the jobs in the Transport, Trade, Computer activities, General Machinery, Structural metal and Finance sectors. Employment in these sectors is relatively high because inputs are sourced more often locally than on average, taxes are relatively low, the labour share is higher than the economy's average and wages are lower. These characteristics contrast with the 11\% jobs created in the Manufacturing, Electricity and services sectors where more inputs are imported than on average in the economy, taxes are higher, the labour share lower and wages higher. The coal mining sector itself employs less people per unit of output mainly because of a low wage share and secondly because of high wages, which reflects that coal worker's are more skilled than the economy's average, and more skilled than in the past \citep{burton2018coal}. These effects dominate the fact that taxes are lower and that coal is sourced locally and very little imported. In all three groups, the factor local demand for final products is higher than the economy's average. It's contribution is per definition the same for all three groups because it is a characteristic of the coal sector, where demand originates. %These figures  are the figures for 2014, and are assumed to hold throughout the period. 

In summary, final demand for coal sustains about as many jobs within the coal sector as in the rest of the economy for 1 million Rand of final demand. This means that for every job created or destroyed in the mining sector because of changes in export demand, another job is created or destroyed throughout the economy, in other sectors. This contrasts with the findings of \citepalias{ChamberofMines2018Strategy}, who find that in 2015, 173.093 jobs were created in the rest of the economy, in addition to 77.747 directly in the coal sector. Of the total number of jobs created by coal-demand, 58\% are jobs with higher than average wages and in sectors with a low labour share (coal, manufacturing, elec, services). The remaining 42\% of created jobs are characterized by lower than average wages (trade, transport, computer activities, general machinery, structural metal and finance). %The coal sector thus emerges as a sector where wages are relatively high, but as a below average job creator, measured in number of jobs. 

%In summary, coal demand generates mostly jobs in sectors where employment per unit of output is lower than the economy's average, because of a low labour share (coal sector), high wages (coal, manufacturing, elec, services), high taxes and many imported inputs (manufacturing, elec, services). The jobs created in the trade, transport, computer activities, general machinery, structural metal and finance are as a group jobs with wages below the average.


These figures are specific for the year 2014, with 2014 export and domestic coal prices. Employment in the coal sector can be thought of as more determined by the tons of coal mines than by the price obtained. Thus, the employment content of the coal sector might fluctuate from year to year along with the average price obtained. This does not however alter the qualitatively overall trend that the sector's ec is low, mainly because of the sector's capital intensity.




%\begin{table}[!h]
%	\centering
%	\begin{tabular}{rc|cl}
%		\hline
%		& Uses & Resources  \\ 
%		\hline
%		Exports & 55.5\% & &\\ 
%		Intermediary inputs &  41.3\%  & 99.6\% &  Domestic production\\ 
%		Inventories &  2.4\% & 0.5\% & Imports    \\ 
%		Consumption & 0.8\% &  &\\ 
%		\hline
%	\end{tabular}
%	\caption{\label{CoalUse2014}The use of \emph{Coal and lignite} (SIC 21) in the South African economy. Data are in million tons, obtained from the 2014 IOT, assuming a single domestic price per ton and a single export and import price. Source, \cite{IOT2014}.}
%\end{table} 


%Coal is mainly consumed close to where it is sourced. 15\% of coal consumption is exchanged o


%\ref{ChamberofMines2018Strategy}
%\ref{MineralsCouncil2018Facts}
%\ref{burton2018coal}
%
%At the beginning of the period 2010-2017, the mining sector (SIC 2) represents about 8\% of GDP, contributing about half a percentage point less to GDP towards the end of the period \citep{P0441StatSA2018Q2} (measured in constant 2010 prices). 
%
%Within the mining sector, coal mining (SIC 21, Coal and lignite) contributed 24.4\% in 2012 and 28.1\% in 2015 to the sector's total income \citep{mining2015}. The primary source of income of the coal mining industry is exports, which represent 55.5\% of production. Most of the remaining share is sold to industries who use coal as inputs (41.3\%), notably for electricity production. Products of the coal mining industry are almost basically obtained through domestic production, only 0.4\% of the coal used in the economy is imported (table \ref{Mining2104_UR} and figure \ref{Uses_Resources_percent_2014}).
%
%\begin{table}[!h]
%	\centering
%	\begin{tabular}{rc|cl}
%		\hline
%		& Uses & Resources  \\ 
%		\hline
%		Exports & 55.5\% & &\\ 
%		Intermediary inputs &  41.3\%  & 99.6\% &  Domestic production\\ 
%		Inventories &  2.4\% & 0.4\% & Imports    \\ 
%		Consumption & 0.8\% &  &\\ 
%		\hline
%	\end{tabular}
%	\caption{\label{Mining2104_UR}Sector \emph{Coal and lignite}, SIC 21. The right hand column, Resources, shows how the total amount of the sector's commodity available to the South-African economy is obtained. The left hand column, Uses, shows how the sector's commodity is used domestically and how much is exported. The data are for the year 2014. Source, \cite{IOT2014}.}
%\end{table} 
%
%Hence, the export of coal directly contributed in 2014 about 1.2\% to GDP, with similar shares for more recent years (which are ?). This implies that movements in coal exports impact GDP significantly. 
%
%Figure \ref{ElecPiechart} shows that all sectors, including households, spend rather similar shares of expenses on electricity.% in a rather evenly distributed manner. 
%Electricity costs on average xx \% $\pm$ xx\% of 

%A possible increase of the price of electricity will thus impact the entire industry. Households spend about 2.5\% of total expenditures on electricity\footnote{All mentioned figures are for the year 2014, obtained from the 2014 input-output table \citep{IOT2014}}. However, % depending on how the increase of the cost of electricity propagates throughout the economy --- either the profitability of industries is impacted and/or the cost increase is transferred in the price of commodities --- 
%final consumers and notably households might bear more of the increased cost of coal if the cost is transferred to final consumption goods then what is suggested by their direct electricity consumption.
%
%% latex table generated in R 3.4.4 by xtable 1.8-3 package
% Thu Sep 27 16:41:51 2018
\begin{table}[ht]
\centering
\begin{tabular}{rrr}
  \hline
 & \% intermediate inputs & \%, accumulated \\ 
  \hline
Electricity, gas and water & 57.2 & 57.2 \\ 
  Coke oven manufacture & 18.0 & 75.2 \\ 
  Metal ores & 6.2 & 81.4 \\ 
  Paper & 4.4 & 85.8 \\ 
  Basic iron and steel & 4.3 & 90.1 \\ 
  Distribution of water & 2.5 & 92.6 \\ 
  Non-metallic minerals & 2.0 & 94.6 \\ 
  Other mining & 1.1 & 95.7 \\ 
  Other community activities & 1.0 & 96.7 \\ 
  Food & 0.9 & 97.6 \\ 
   \hline
\end{tabular}
\end{table}

%
%% latex table generated in R 3.4.4 by xtable 1.8-3 package
% Thu Sep 27 16:41:51 2018
\begin{table}[ht]
\centering
\begin{tabular}{rrr}
  \hline
 & \% intermediate inputs & \%, accumulated \\ 
  \hline
Metal ores & 19.8 & 19.8 \\ 
  Nuclear fuel & 8.4 & 28.3 \\ 
  Electricity, gas and water & 8.1 & 36.4 \\ 
  Real estate activities & 7.7 & 44.0 \\ 
  Basic iron and steel & 6.7 & 50.7 \\ 
  Computer activities & 5.2 & 55.9 \\ 
  Trade & 4.3 & 60.2 \\ 
  Other chemicals & 4.2 & 64.4 \\ 
  Agriculture & 3.6 & 68.0 \\ 
  Transport & 3.2 & 71.2 \\ 
   \hline
\end{tabular}
\end{table}

%
%The above describes how different coal export demand scenarios can have an effect on the economy as a whole because coal is used as an input, directly by some industries and indirectly by many through electricity use. The second channel through which different export scenarios impact the economy is due to the demand addressed to other sectors by the coal sector. Table \ref{} evaluates this effect in terms of the sector's employment content, defined as the number of jobs created --- throughout the economy --- by 1 million Rand of final demand addressed to the coal industry. For the coal sector, final demand is for 94.5\% made up of exports\footnote{Figure obtained from table \ref{Mining2104_UR}.}. 
%
%%\input{../tables/Ind4_E_VA_2014.tex}
%% latex table generated in R 3.4.4 by xtable 1.8-3 package
% Tue Sep 25 16:58:17 2018
\begin{table}[ht]
\centering
\begin{tabular}{rllrrrr}
  \hline
 & Industry & Code & Ec & E, 2014 & VAc & VA, 2014 \\ 
  \hline
1 & Coal and lignite & 21 & 0.72 & 25581.44 & 0.60 & 21293.75 \\ 
  2 & Transport & 71-74 & 0.28 & 9847.72 & 0.09 & 3360.13 \\ 
  3 & Trade & 61-63 & 0.20 & 6963.92 & 0.03 & 933.20 \\ 
  4 & Computer activities & 86\_88 & 0.11 & 3801.96 &  &  \\ 
  5 & Financial intermediation & 81 & 0.03 & 931.21 & 0.01 & 505.48 \\ 
  6 & General machinery & 356-9 & 0.02 & 877.14 &  &  \\ 
  7 & Structural metal & 354-5 & 0.02 & 843.29 &  &  \\ 
  8 & Other services nec & 95\_96\_99 &  &  & 0.02 & 748.47 \\ 
  9 & Auxiliary financial & 83 &  &  & 0.02 & 630.31 \\ 
  10 & Electricity, gas and water & 41 &  &  & 0.01 & 496.87 \\ 
   \hline
\end{tabular}
\end{table}

%
%The total employment content of the mining sector is lower than the economy's average, as can be seen in figure \ref{Decomposition_direct_indirect_avg}. This means that for 1 Rand of final demand addressed to the mining industry, less jobs are generated throughout the economy than when the same value of demand is distributed evenly over all sectors. On average in the economy, 2.48 jobs are created for 1 mR of final demand (local demand + exports) whereas in coal mining, 1.58 jobs are sustained throughout the economy. %by 1 mR of final demand addressed to the sector, by the sector's activity, of which. 
%The direct employment content, which is the jobs generated within the coal sector, equals 0.72. The remaining share is the indirect employment content and equals 0.86. This is the number of jobs generated in other sectors by the export of coal. Table \ref{} shows that 68\% of these indirectly created jobs are situated in three sectors : Transport (33\%), Trade (23\%) and Computer activities (13\%). 
%
%The Transport sector is thus, after coal mining itself, the first sector impacted by different coal export scenarios. 
%
%% Look at what percentage of jobs comes from coal.
%
%%The economic activity generated by the demand for final goods addressed to the coal mining sector, which is primarily export goods, generated by the sector's use of intermediate inputs, is situated in descending order in terms of the number of jobs created, in the Transport sector (SIC 74-71) with 0.27 jobs (per 1 mR final demand addressed to the mining sector), 0,19 to the Trade sector, 0,11 to Computer activities; 0,026 of Financial intermediation and 0,05 jobs divided equally over sectors 354-359, Structural metal and General Machinery.
%
%\begin{table}[ht]
%	\centering
%	\renewcommand*{\arraystretch}{1.15}
%	\begin{tabular}{lp{20pt}lcc | c}
%		\toprule
%		%		\multicolumn{3}{l}{Inputs} &\multicolumn{3}{c}{} \\ 
%		%\multicolumn{3}{c}{Origin of jobs} &\multicolumn{2}{c}{Export scenarios} &  \\ 
%		\multicolumn{3}{c}{Inputs} & 2014 & 2017 & 2014, direct \& indirect\\ 
%		\midrule
%		\multicolumn{3}{l}{Wages (\% of output)}  		 & 17.8 & 17.2 &  \\ 	
%		\multicolumn{3}{l}{Intermediate inputs (\% of output)}  & 39.4  & 46.8  & 44.0 \\ 
%		%& 						 & Local & 31.2 &  37.0 &  35.4\\ 
%		%& 						 & Imports & 8.1 &  9.8  & 8.6 \\
%		& 						 & Local (\% of total input costs) & 79.4 &  79.0 &  80.5\\ 
%		& 						 & Imports (\% of total input costs) & 20.6 &  21.0  & 19.5 \\
%		\bottomrule
%	\end{tabular}
%	\caption{\label{costShares_2014_2017}\small Input and wage costs as a share of output in the Coal sector in 2014 and in 2017. The first to columns on the left, "2014" and "2017" provide direct input costs. The column on the right "2014, direct \& indirect" takes into account the total output, direct and indirect, required of all sectors to satisfy coal export demand. The input costs are the input costs summed over all sectors. Source : \cite{IOT2014} and \cite{MineralsCouncil2018Facts}.}
%\end{table}


\subsection{Electricity}

%\citep{von2016energy}

\begin{figure}[!h]
	\centering
	\includegraphics[width=0.5\textwidth]{../../biblio/Eskom/EskomPE.png}
	\caption{\label{Eskom2017PE}Breakdown of primary energy cost of electricity generation, 2017. Between parentheses, the share of generated volume per technology ; the numbers are total cost per technology in million Rand. Source : Eskom Integrated report 2017 \citep{Eskom2017AR}.}
\end{figure}

Electricity is in South Africa provided by state-owned Eskom. In the first eight months of 2018, 91 \% of electricity was generated by Eskom \citep{P4141_201808}. The remaining share was generated by independent power producers (IPPs), which are mostly renewables. IPPs sell generated power to Eskom who is the sole transmitter and provides 60\% of distribution ; 40\% is distributed by municipalities.

In 2017, 85 \% of electricity was generated through coal-plants \citep{Eskom2017AR}. The remaining 15\% stems from the country's sole nuclear plant, gas-turbines, 4\% from renewables produced by IPPs and a small share is imported (figure \ref{Eskom2017PE}). Coal is historically the primary energy source of choice in South Africa, because easily mined domestically and thus cheap. Even though coal-prices have been rising, in 2018, the 85\% of coal-generated electricity represented 63 \% of the total primary energy cost for Eskom. Notably, the primary energy cost per $MWh$ of coal is 6-7 times lower than the price Eskom pays for electricity generated by IPPs \citep{Eskom2017AR}.  

\begin{figure}[!t]
	\centering
	\includegraphics[width=0.7\textwidth]{../../biblio/electricity/Eskom-electricity-prices.jpg}
	\caption{\label{Eskom-electricity-prices} The evolution of the real and nominal price of electricity in South Africa from 1974-2019. Source : \href{https://mybroadband.co.za/news/business/68724-how-lecturers-tore-eskoms-tariff-case-apart-in-25-minutes.html}{Eberhard, A. and Logan, C.}.}
\end{figure}


\begin{figure}[!t]
	\centering
	\includegraphics[width=0.9\textwidth]{../../biblio/StatSAreports/Electricity/ElecSA2010-2018_P4141_201808.png}
	\caption{\label{Elecvolume_2010_2018}The volume of electricity generated from 2010-2018. Source : \citep{P4141_201808}.}
\end{figure}

Since 2008, real electricity prices have risen drastically in South Africa (figure \ref{Eskom-electricity-prices}). Part of the price increases are due to the rising cost of coal and part to the necessity to raise cash for investment in new capacity. The steady rise of the price is seen as a factor in the decline of electricity demand over the period 2011-2015 (figure \ref{Elecvolume_2010_2018}). Several studies have investigated the effect of the price hikes on demand, on an aggregate and on a sectoral level, for industry and for consumers, see \cite{goliger2018electricity} for an overview.

%It is against this backdrop that the additional source of increase in the generation cost of electricity under a 2DS --- the loss of cross-subsidies for the domestic coal-price because of reduced export --- has to be evaluated. 

It is against this backdrop that the possible higher domestic coal-price under 2DS has to evaluated. Indeed, the loss of cross-subsidies for coal consumed domestically due to lower export-volumes under 2DS constitute an additional source of increase in the generation cost of electricity under a 2DS versus a BAU scenario.

%\citep{inglesi2010forecasting}


%\subsubsection{Platinum metal group}
%
%The mining of metal ores generates 1.89 jobs throughout the economy per mR of final demand. 
%
%\subsubsection{Oil}
%
%The importation of crude oil falls under mining sector SIC 22, which is classified in the IOT under Other mining. The import figure of this sector corresponds to the import of crude oil. 
%
%Refinement of crude oil falls under SIC 332, which is in the IOT grouped with 331 (coke oven products) 

%\subsubsection{Global value chains}
%
%Under a 2DS scenario, supply chains are necessarily "greener". 
%
%How is South Africa's economy possibly exposed to the greening of global supply chains ? Is it risk for the economy ?
%- the most obvious factor is the risk posed to coal exports
%- what share of the SA economy is dependent on its inscription in global supply chains ? Are the economic activity and the jobs hence created under risk ? 
%- the importance generally of global supply chains might be affected if transportation cost increases because of the necessity to reduce emissions
%- companies might start to take into account the carbon content of supplies/intermediate inputs

\clearpage
\section{Prospective analysis}

\subsection{Export and domestic demand for coal}

%\subsubsection{Coal export}

%using the macroeconomic picture of the economy provided by the IO-model, of the effect on employment of the difference in coal export demand in 2035 under 2DS versus BAU, a comparison is made between the South African economy in the year 2014 and the same economy with the sole difference that coal exports are diminished by 90\%, following CPI's projections. 
%Both the direct and the indirect impacts in terms of revenue and jobs will be discussed. In a second instance the impact is traced to specific subsectors to evaluate what share of the activity of the concerned economic units is impacted.

\begin{figure}[!t]
	\centering
	\includegraphics[width=\textwidth]{../graphs/Coal-Export_CPI_VolPrice_BAU_2Deg.pdf}
	\caption{\label{Coal-Export_CPI_VolPrice_BAU_2DS}\small Projections for the period 2018-2035 of the volume of coal exported by South Africa in million tons (mt), the export price in USD/mt and the total value of exports in USD under CPI's business as usual (BAU) scenario and 2 degrees (2DS) scenario. Source : CPI model, based on data from Wood MacKenzie.}
\end{figure}

\begin{figure}[!t]
	\centering
	\includegraphics[width=\textwidth]{../graphs/Coal-Domestic_CPI_VolPrice_BAU_2Deg.pdf}
	\caption{\label{Coal-Domestic_CPI_VolPrice_BAU_2DS}\small Projections for the period 2018-2035 of the volume of coal domestically consumed in South Africa in million tons (mt), the domestic average price in m Rand/mt and the total value of exports in m Rand under CPI's business as usual (BAU) scenario and 2 degrees (2DS) scenario. Source : CPI.}
\end{figure}

In a world economy developing in a way that is compatible with average global warming limited to 2°C degrees, global demand for coal will be drastically lower. Over the period 2018-2035, South Africa is estimated by CPI's models to export about half the volume of coal under the 2DS versus a under the BAU scenario (left graph of figure \ref{Coal-Export_CPI_VolPrice_BAU_2DS}). %CPI's models estimate that the volume of coal exported by South Africa over the period 2018-2035 under a 2DS is about half the volume under BAU, with considerably lower export demand 
Towards 2035, the volume of exported coal drops below 10 million tons in the 2DS scenario which is about 85\% lower than under BAU. Moreover, the price on the global market is expected to be lower under 2DS, resulting in the lower profitability of the sector. %on average and in pressure %  first instance in a lower profitability of the sector in general and in particular in the %on the financial sustainability of mines with access to the export market. % and their possible closing \citep{CPI2019SA}. 
The projected export trajectories under both scenarios are presented in figure \ref{Coal-Export_CPI_VolPrice_BAU_2DS}. The area between the two lines of the graph on the right represents the revenue from coal exports at risk under the 2DS scenario. %The low price fetched for coal on the export market exacerbates the effect of the revenue loss from low demand under 2DS. 

%The risk assessed in the CPI and the present study is the external transition risk, due to the difference in demand for steam coal on the international market under a 2DS versus a BAU scenario.
The domestic coal demand and price scenarios are represented in figure \ref{Coal-Domestic_CPI_VolPrice_BAU_2DS}. The lower export-revenues under 2DS do not lead to a higher domestic price for coal under 2DS versus BAU in CPI's domestic 2018-2035 scenarios, contrary to historical dynamics (cf. section \ref{DomesticExport}). %The dynamics on the international coal market do not influence the volume of coal consumed domestically 
The reason for this is that the assumptions on the formation of the domestic coal-price lead to a higher price under 2DS for the coal mined in Limpopo but a lower price for the coal mined in Mpumalanga\footnote{Source :  communication with CPI.}. Upwards pressure on the domestic price is due to the loss of export revenues to pay fixed costs, which are now only paid by the domestic market. This is the reason for the higher price under 2DS in Limpopo. In Mpumalanga, this effect is dominated by increased competition  on the domestic spot market, which lowers the price. Indeed, it is assumed that the coal which is not exported, enters the domestic spot market. Part of the supply to power plants is secured through contracts and is not affected by export dynamics. The remaining part (34\% in 2018 under BAU/2DS) is acquired on the domestic spot market, where competition increases under 2DS. In Mpumalanga, the increased competition under 2DS (50\% of demanded volume is acquired on the sport-market by 2030) on the domestic spot market lowers the price. 

The volume of domestic coal consumption is determined by the installed capacity and lifetime of coal power plants, as well as by demand from industry and from CTL in particular.  the price the electricity producer pays for coal does depend on whether there is competition between the export market and the domestic market.  
%the profits the mines that produce both for the domestic and the international market make through exports. The domestic price is determined by contracts between mines and Eskom. Long-term contracts guarantee a low and stable price for the electricity producer. The remaining share is bought on the spot-market, where prices are higher on average and more volatile. Under 2DS, the loss of revenue from exports, market on which the price is much higher than on the domestic market, translates into higher domestic prices to sustain the profitability of the active mines. (comment on that they impose their price?) 
The projected volume and price trajectories by CPI are given in figure \ref{Coal-Domestic_CPI_VolPrice_BAU_2DS}. Both under BAU and 2DS scenarios, the domestic price increases and becomes more volatile. This is because long term contracts between Eskom and mines come to an end and more coal will be exchanged on the spot market. The projected domestic coal price trajectories are not significantly different between both scenarios ; the average annual increase under BAU is 5.6\% and under 2DS 5.7\%. The total volume consumed peaks under both scenarios in the early 2020's and declines thereafter. This is due to coal plants reaching end-of-life and going of the grid.

The difference in export volume and revenue between the BAU and the 2DS scenarios impacts firstly the coal sector itself and secondly, to varying degrees, the rest of the economy. As set out above, this is because part of the revenue obtained from coal exports is spent on domestically produced inputs, necessary for mining coal, and becomes revenue for the suppliers. The domestic price and volume trajectories are broadly the same under both scenarios. %coal price trajectories are not significantly different between both scenarios ; the average annual increase under BAU is 5.6\% and under 2DS 5.7\%. 
The projected increase in the cost of coal for power producers is an additional factor in the rising cost of power production that has been going on for years and will put further pressure on the electricity price as briefly discussed in section \ref{elec} \citep{}. We here address the effect on GDP and on employment of possible loss of coal exports.

%Thus, the risk posed under a 2DS by the lower revenue obtained from coal exports is not only a risk for the sector itself, but also for other parts of the economy. Furthermore, because the coal sector's suppliers themselves spent part of their sales revenue (obtained from the coal sector who obtained its revenue from export) on inputs, the coal export revenue spreads throughout the economy as value added of (almost) all sectors. Value added is here defined in a broad sense, i.e. as output, which is taken to be the same as total sales revenue from domestic final goods, minus the cost of inputs. It thus includes, referring to the IOT in table \ref{IOT_importsUses}, net taxes on products, other taxes, wages and gross operating surplus. In addition, part of the revenue is spent on imports. The value added of each sector thus flows to government via taxes, to workers in the form of wages and a part of gross operating surplus remunerates investors in the form of debt payments or dividends. 
%the value created throughout the economy by the export of coal benefits different actors in the economy. For each sector, what is left over from sales revenue after paying suppliers splits into three parts : payment to workers, payment of taxes to different layers of government and a remaining share labeled gross operating surplus. This last part is used for investments and exceptional maintenance costs of the production infrastructure and is distributed to its investors, who are the de factor owners of the production infrastructure, in the form of debt payments, payments of dividends to shareholders. %This is the part that is mostly distributed to the owners of the fixed capital of the sector, i.e. to the economic actors that invested in the sector's production infrastructure and expect a return on their investment. 

%\begin{figure}[!b]
%	%	\centering
%	\hspace{-10pt}\includegraphics[width=1.1\textwidth]{../graphs/ExportRevenue_Decomposition_BAU_2Deg.pdf}
%	\caption{\label{ExportRevenue_Decomposition_Diff}\small The graph shows the revenue from the export of coal in constant Rand (2014), over the period 2018-2035, under the business as usual scenario (BAU) on the left and under the 2°degrees (2DS) scenario on the right. The export revenue is decomposed into importation costs of intermediate goods (red), taxes, wages and gross operating surplus, for the coal sector (yellow) and for the rest of the economy (blue). Is is assumed that the price changes of the exported coal impact tax revenue and gross operating surplus. All other domestic prices and wages stay constant over the whole period.  }
%\end{figure}

As laid out in section \ref{Section_decomposition}, the input-output methodology allows the decomposition of export revenue in imports and in value added over all sectors of the economy. Figure \ref{ExportRevenue_Decomposition_Diff} shows the decomposition of the export revenue trajectories over the period 2018-2035, for both scenarios, in imports (in red), taxes and value added (in yellow for the coal sector and blue for the other sectors). The estimates are based on the simplifying assumption that the structure of the economy remains very close throughout the modeling period as it is in 2014. It is assumed that for every sector, the amount of inputs per unit of output remains constant, as do the shares of taxes, value added and its components. All prices are by hypothesis constant throughout the modeling period, including the unitary wage, except for the domestic and export price of coal. The coal export price affects only taxes and gross operating surplus of the coal mining sector.%, such that the sum of its value added and input costs equals its sales revenue. This assumption is maintained both under BAU with sustained prices as in the 2DS scenario with low prices. 

In 2018, 13.2\% of export revenue is spent on imported inputs, 28.9 \% goes to the payment of workers (17.9\% in the coal sector and 10.9\% in other sectors of the economy), 55.2\% becomes gross operating surplus (41.5\% in the coal sector and 13.7\% in other sectors) and 2.6\% taxes (1.5\% paid by the coal sector and 1.1\% by the rest of the economy). %The share of revenue accruing to jobs throughout the economy is only slightly more than half the capital share and is within the coal sector less than half the capital share. This is the main reason why less jobs are created per unit of final demand addressed to the coal sector than the economy's average per unit of final demand, as illustrated in figure \ref{Decomposition_PQ_within} and discussed in section \ref{ec_results}.

Towards 2035, these shares change because of the effect of the coal export price on taxes and gross operating surplus. Under BAU, %the exported volume is approximately equal to the volume at the beginning of the period in 2018. However, 
the price on the international market is projected to be by up to 30\% higher in the 2030's, making the sector much more profitable than today. Assuming real input costs stay constant, the share of the coal sector's value added thus rises from 60.6\% of the total export revenue in the beginning of the period to 72.3\% towards 2035. In this scenario, the higher profitability might have a positive effect on wages. Under 2DS, the opposite effect takes place. %The projected exported volume is in 2035 a tenth of today's. This implies that few mines will be sustained by export demand. Additionally, 
Low prices mean that exporting mines will generate less profit \textit{via} exports : value added in the mining sector captures only 54.8\% of export revenue in 2035 under 2DS because gross operating surplus is squeezed.


\begin{figure}[!t]
	\centering
%	\hspace{-10pt}\includegraphics[width=\textwidth]{../graphs/ValueAdded_Employment_Sector_Coal-Export_BAU_2Deg.pdf}
	\hspace{-10pt}\includegraphics[width=0.7\textwidth]{../graphs/ValueAdded_Employment_Sector_Coal-Domestic-Export_nominal_atRisk2035.pdf}
	\caption{\label{ValueAdded_Employment_Sector_Coal-Export_BAU_2DS}\small The graph shows the revenue from the export of coal in constant Rand (2018), over the period 2018-2035, under the business as usual scenario (BAU) on the left and under the 2°degrees (2DS) scenario on the right. The export revenue is decomposed into importation costs of intermediate goods (red), taxes, wages and gross operating surplus, for the coal sector (yellow) and for the rest of the economy (blue). Is is assumed that the prices changes of the exported coal impact tax revenue and gross operating surplus. All domestic prices and wages stay constant over the whole period, except }
\end{figure}

The difference in the areas between the BAU and the 2DS panels of figure \ref{ExportRevenue_Decomposition_Diff} represents the possible loss in revenue. As a measure for this possible loss, we take the difference between the BAU and 2DS value added figures at the end of the modeling period in 2035, and compare this difference with South Africa's 2017 GDP. % for each of the components of value added. This loss in revenue poses a risk when the economic agent that would receive the revenue is highly dependent on the revenue source in question.
The right panel of figure \ref{ValueAdded_Employment_Sector_Coal-Export_BAU_2DS} shows value added stemming from coal export in 2018 and in 2035 under BAU and under 2DS in million Rand. %To get an idea of the impact on GDP, these numbers are compared to South Africa's 2014 GDP. 
The projected value added generated by coal export in 2018 would have been 1.52\% of 2017 GDP, of which the majority of the contribution would stem from coal with 1.06\% and the remaining 0.47\% from the rest of the economy. In 2035, under BAU, coal export would attain 2.11\% of 2014 GDP and under 2DS only 0.18\% of GDP. Thus, between both scenarios, towards the end of the period, the difference in value added to the economy between both scenarios would be close to 2\% of 2017 GDP. Over two thirds of this loss would be situated in the coal sector itself ; of the remaining lost value, about a third is situated in the transport sector. This is because a big cost to the coal sector is the transport of mined coal destined for export by truck and rail. 

%Figure \ref{ValueAdded_Employment_Sector_Coal-Export_BAU_2DS} compares both value added and employment throughout all sectors in 2035 under BAU and 2DS with the (current) situation in 2018. Whereas in terms of value added, two thirds is concentrated in the coal sector itself and 10\% in the Transport sector. In terms of employment, about half the jobs sustained by coal export are situated within the coal sector and the other half in the other sectors of the economy. Notably, the Transport sector counts for 17.8\% of the total number of jobs sustained by coal export, Trade for 12.7\% and Computer activities for 5.8\%. The number of workers per ton of coal mined and wages are by hypothesis constant throughout the period. Table \ref{NE_Sector_Coal-Export_BAU_2DS_table} lists how many jobs are sustained, in each of the different sectors, by the export of coal and the number of jobs at risk under a 2DS scenario.

\begin{table}[!t]
	\centering
	\renewcommand*{\arraystretch}{1.15}
	\begin{tabular}{lp{20pt}lccc}
		\toprule
		\multicolumn{3}{c}{\multirow{2}{80pt}{Origin of jobs}} &\multicolumn{2}{c}{\makecell{Number of persons \\employed, 2035}} & \multirow{2}{70pt}{Jobs at risk}\\ 
		%\multicolumn{3}{c}{Origin of jobs} &\multicolumn{2}{c}{Export scenarios} &  \\ 
		\multicolumn{3}{c}{} & BAU & 2DS & \\ 
		\midrule
		\multicolumn{3}{l}{Jobs sustained by coal exports, total}  		 & 46179 & \ 16370 & 29809 \\ 	
		\multicolumn{3}{l}{Jobs sustained by coal exports, coal sector}  & 21979 & \ \ \,7810 & 14209 \\ 
		\multicolumn{3}{l}{Jobs sustained by coal exports, rest of economy}  & 24200  & \ \ \,8600 & 15600\\ 
		& 						 & Transport & \, 8186 &  \ \ \,2909 & \ \,5277\\ 
		& 						 & Trade & \, 5834 &  \ \ \,2073  &\ \,3761\\ 
		& 						 & Computer activities & \, 2688 & \ \ \ \, 955  & \ \,1733\\
		& 						 & General machinery & \ \,\ 864 & \ \ \ \, 307 & \ \, 557\\  
		& 						 & Remaining sectors & \ 6629 & \ \ \, 2355 & \   4273\\  
		\bottomrule
	\end{tabular}
	\caption{\label{coalExport_BAUvs2DS_2035_table_employment}\small Comparison of the impact of export demand on the number of jobs sustained by exports of coal, under the BAU and 2DS, in 2035. %The percent difference in coal exports between the two scenarios is equal to that between BAU and 2DS in 2035 as projected by CPI. The BAU scenario is approximated by the South African economy as in the year 2014.
	}
\end{table}
\begin{table}[!t]
	\centering
	\small
	\begin{tabular}{lp{20pt}lcccc}
		\toprule
%		\multicolumn{3}{c}{\multirow{2}{100pt}{Origin of jobs}} &\multicolumn{3}{l}{\makecell{Number of persons employed}} & \multirow{2}{70pt}{Jobs at risk}  \\ 
		\multicolumn{3}{c}{} &\multicolumn{3}{l}{\ Number of persons employed\ } &   \\ 
		\multicolumn{3}{c}{Origin of jobs} & 2018 & \multicolumn{2}{c}{2035} & Jobs at risk  \\ 
		\multicolumn{3}{c}{} & &  BAU & 2DS & all/transition\\ 
		\midrule
		\multicolumn{3}{l}{Jobs in the coal industry, total} & 80,481 &68,999   & 54,830 & 25,651/\,14,169\\ 
		\midrule
		\ \ \ \ & \multicolumn{2}{l}{due to export}  	   & 23,335 &\ 21,979  & \ \,7,810 & \,15,525/\,14,169\\ 
		& \multicolumn{2}{l}{due to domestic use}  & 57,146 & 47,020  & 47,020 & \,10,126/\,\ \ \ \ \ \ \ 0\\ 
		& 	 & Electricity, gas and water 		   & 41,461 & 31,335  & 31,335 & \,10,126/\,\ \ \ \ \ \ \ 0\\ 
		&    & Coal to liquid & \ \,9,923  &\ \,9,923 &\ \,9,923 & \,\ \ \ \ \ \ \ 0/\,\ \ \ \ \ \ \ 0\\ 
		&    & Other industry &\ 5,762 &\ 5,762 &\ 5,762    & \,\ \ \ \ \ \ \ 0/\,\ \ \ \ \ \ \ 0 \\ 
		\midrule
		\multicolumn{3}{l}{Jobs sustained by coal industry, rest of economy}  & 88,615 & 75,973  & 60,362 & 28,253/\,15,611\\ 
		\midrule
		& 						 & Transport & 29,975 & 25,699  &  20,422  & \ \,9,553/ \ 5,277\\ 
		& 						 & Trade & 21,362& 18,315 &  14,554  &\ \,6,808/ \ 3,761 \\ 
		& 						 & Computer activities & \ \,9,844& \, 8,440 & \ \,6,707   &\ \,3,137/ \ 1,733\\
		& 						 & General machinery &\ \,3,162 & \ \,2,711  &  \ \,2,154 &\ \,1,008/ \ \ \ \,557 \\  
		& 						 & Remaining sectors & 24,272 & 20,808&  \ 16,535  &\ \,7,737/ \ 4,273 \\  													
		\bottomrule
	\end{tabular}
	\caption{\label{coalExport_BAUvs2DS_2035}Number of jobs sustained throughout the economy by the projected domestic and export demand for coal, under 2DS and BAU, in 2018 and 2035.}
\end{table}

To compare the effect on employment of the BAU and 2DS export scenarios, we compare the number of jobs sustained by coal export demand throughout the economy under each. The difference between them is the number of jobs at risk.
%The projected number of jobs persons employed because of coal export, throughout the economy, in 2018 is 52.827. This represents about 0.4\% of the total number of jobs in South Africa in 2014. Coal export thus contributes less to total employment than to value added. 
In 2035, about 39,674 jobs less would be sustained by coal export under a 2DS scenario versus under BAU, over the whole economy. Table \ref{coalExport_BAUvs2DS_2035_table_employment} and the left panel of figure \ref{ValueAdded_Employment_Sector_Coal-Export_BAU_2DS} show how the difference in the number of jobs decomposes over the economy. More than half the jobs hypothetically lost are in the directly and indirectly supplying sectors of coal mining. These employment estimates are based on the assumption that the number of employees per unit of output is constant throughout the period and equal to figures for 2014. For the coal mining sector, the assumption is that the number of workers per ton of coal produced is constant and as in 2014. 
% what kind of changes could be expected in reality ?
%In terms of value added, most of the risk is concentrated in the coal sector. 

The total number of jobs possibly lost represents about 0.3\% of the number of persons employed in 2014. However, these possible projected job losses in the mining sector are regionally concentrated in Mpumalanga. Likely, many of the jobs at risk in the rest of the economy are located in the same region.

\begin{figure}[!t]
	%	\centering
	\hspace{-10pt}\includegraphics[width=1.1\textwidth]{../graphs/ExportDomesticRevenue_Decomposition_nominal_BAU_2Deg.pdf}
	\caption{\label{ExportDomesticRevenue_Decomposition_BAU_2DS}\small The graph shows the revenue from both domestic sales (for electricity only) and the export of coal in constant Rand (2014), over the period 2018-2035, under the business as usual scenario (BAU) on the left and under the 2°degrees (2DS) scenario on the right. The export revenue is decomposed into importation costs of intermediate goods (red), taxes, wages and gross operating surplus, for the coal sector (yellow) and for the rest of the economy (blue). Is is assumed that prices changes impact tax revenue and gross operating surplus. All domestic prices and wages stay constant over the whole period, except for the domestic and export coal price. Source : CPI and own calculations. }
\end{figure}

In addition, the lost jobs in 2035 under 2DS versus BAU represent lost wages and thus lost consumption, which further impacts GDP. This effect would lower 2017 GDP with another 0.25\%, in addition to the 1.93\% directly lost due to coal exports. This effect is to be taken as a general indication. Under 2DS, the 39,674 jobs would disappear over a period of 18 years. Aided by accompanying policies, new economic activity and jobs can be created throughout this period. Consumption would likely be less depressed than the figure here estimated, which is a worst case scenario. Reduced export demand under BAU also leads to lower demand for investment goods. Assuming that the share of investment in output is constant and the same as in 2014, lower investment demand under 2DS would reduce 2017 with 0.03\%. This figure is relatively low because more than half of investment demand generated by demand for coal is imported and does not contribute to GDP. Overall, the value lost to reduced coal exports under 2DS and the indirect effects of diminished consumption and investment represent 2.2\% of 2017 GDP.

Table \ref{coalExport_BAUvs2DS_2035} shows the total number of jobs in the coal industry, sustained by domestic demand and exports. Assuming productivity per worker is the same as in 2014, under the BAU scenario the sector employs about 79761 workers and under 2DS 60920. The difference between both (18841) is the number of jobs at risk in the coal sector due to low export demand, as mentioned above (table \ref{coalExport_BAUvs2DS_2035_table_employment}). Under both scenarios, the sector still employs a workforce comparable to current numbers. Demand from industry other than the power sector contributes about 26000 jobs, assuming their coal demand is identical to 2014's. This number could be lower because firms might choose to invest in own power generation other than coal, or could be higher if the sector grows. The electricity sector contributes 31779 jobs assuming CPI's demand scenario.



%Previously, we assessed the impact on GDP and on jobs of the potential, previously unanticipated, reduction of the revenue flow from coal exports. Here, combining projected domestic and export demand trajectories, we will assess how many jobs are sustained by total demand for coal. Additionally, differentiating the price fetched for coal on the seaborne and the domestic market, we will assess to what extent the rising prices on the domestic market ameliorate the profitability of the sector in the 2DS scenario. 


%Unless because of the high price people move away from coal. Unless labour productivity changes.


%\begin{table}[!t]
%	\centering
%	\renewcommand*{\arraystretch}{1.15}
%	\begin{tabular}{lp{20pt}lccc}
%		\toprule
%		\multicolumn{3}{c}{\multirow{2}{120pt}{Origin of jobs}} &\multicolumn{2}{c}{\makecell{Number of persons \\employed, 2035}} & \multirow{2}{70pt}{Jobs at risk}\\ 
%		%\multicolumn{3}{c}{Origin of jobs} &\multicolumn{2}{c}{Export scenarios} &  \\ 
%		\multicolumn{3}{c}{} & BAU & 2DS & \\ 
%		\midrule
%		\multicolumn{3}{l}{Jobs sustained by coal exports, total}  		 & 94966 & 13139 & 81827 \\ 	
%		\multicolumn{3}{l}{Jobs sustained by coal exports, coal sector}  & 45098 & \ 6240 & 38858 \\ 
%		\multicolumn{3}{l}{Jobs sustained by coal exports, rest of economy}  & 49868  & \ 6899 & 42969\\ 
%		& 						 & Transport & 16868 &  \  2334 & 14534\\ 
%		& 						 & Trade & 12022 &  \ 1663  & 10359\\ 
%		& 						 & Computer activities & \, 5540 & \, \, 766  & \ \,4774\\
%		& 						 & General machinery & \ \,1780 &  \, \, 246 & \ \,1534 \\  
%		& 						 & Remaining sectors & 13658 &  \, 1890 & \ 11768 \\  
%		\bottomrule
%	\end{tabular}
%	\caption{\label{coalExport_BAUvs2DS_2035_table_employment}\small Comparison of the impact of export demand on the number of jobs related to the activity in the coal mining sector (SIC 21). The percent difference in coal exports between the two scenarios is equal to that between BAU and 2DS in 2035 as projected by CPI. The BAU scenario is approximated by the South African economy as in the year 2014.}
%\end{table}



%A reduction of coal export thus poses a risk to these other sectors if the loss of revenue stemming from a reduction of demand for inputs by the coal export is concentrated in specific actors of the South-African economy, such as in companies and/or geographical areas.


%The net length of the horizontal bar corresponding to the Coal and lignite sector represents the difference between the average number of jobs created in the economy within the sector to which 1 million Rand of final demand is addressed, and the number within the coal sector sustained by a unit of final demand within the coal sector : $-0.58$.

%The employment content is a sum of the number of jobs over all sectors of the economy. Since we are looking at the deviation of the employment content of a sector and the economy's average, this is equivalent to summing, over all sectors, of deviations between employment created within each sector (by a unit of final demand of a particular sector) and the average employment created within a sector by a unit of final demand. For some sectors this deviation will be positive, for other negative. This means that demand from the coal sector created jobs in the economy in some sectors where employment per unit of output is higher than on average in the economy, and in some lower. 

%total value added created in the domestic economy according to industries.%al sectors where the value is created. In 2018, of the value added created in the domestic economy, 60.6\% is created within the coal sector, 9.2\% by the Transport sector, due to the transport of coal by rail and truck, 2.6\% in the Trade sector. %and x\% in the rest of the economy. 

%Thus, in terms of value added, most of the risk falls on the coal sector itself, and a part of the transport sector that is very directly linked to the coal sector. 

%To make an estimate from a macroeconomic point of view of the effect on the economy of the difference in coal export revenue in 2035 under 2DS versus BAU, the demand-pull IO-model is used. the prospective price and volume scenarios of coal export under a 2DS provided by CPI, and used here as input, already include the impact on the profitability of the sector of the lower coal-price on the international market. A number of unprofitable mines are assumed closed in the 2DS scenario. This means that for the 2DS trajectories of price and volume provided, the corresponding cost structure of the coal sector (column of the IOT) is assumed different than the one in the BAU scenario. We will here not attempt to make an estimate of the more efficient technology of the coal-mining sector on the aggregate level that results from the closing of unprofitable coal-mines under 2DS. The matrix of technical coefficients is assumed the same under both scenarios. This means that the obtained measures of profitability under 2DS will be a low estimate, since too high costs are assumed.
%Denote $p_{E,coal}^{k,2035}$ and $q_{E,coal}^{k,2035}$, $k \in \{\mli{BAU}, \mli{2DS}\}$ the export prices and volumes of coal in the year 2035 for both scenarios. CPI projects export values of coal under respectively BAU and 2DS of $p_{E,coal}^{BAU,2035}\, q_{E,coal}^{BAU,2035}$ and $p_{E,coal}^{2DS,2035}\, q_{E,coal}^{2DS,2035}$. The values fed into the IO-model to determine the difference in output and employment induced per sector are however determined by the BAU price : $p_{E,coal}^{BAU,2035}\, q_{E,coal}^{BAU,2035}$ and $p_{E,coal}^{BAU,2035}\, q_{E,coal}^{2DS,2035}$. The remaining parts of the demand scenarios are as in 2014, as given by the 2014 IOT. 

%Stat SA documents on the transport sector : \citep{LandTransp2014}, \citep{P7000_2013}.

%\subsubsection{The domestic coal-market}

%The impact on employment between the two scenarios can be found in table \ref{coalExport_BAUvs2DS_2035}. The top part of the table shows that the number of jobs in the coal sector under 2DS is about half that under BAU. The IO-model projects about 39 600 jobs in the coal sector under 2DS, which is in the same ballpark as CPI's estimate. Of this total number of jobs in coal mining, 15\% is generated by export and some consumption by households. The biggest share, 85\%, is generated by demand from other sectors. 


%The input-output methodology does not provide information on who are the holders of the fixed capital of each sector. This aspect of the external transition risk for South Africa is covered by the finance-based, sector and asset-level study performed by CPI \citep{CPI2019SA}. 


%\begin{table}[ht]
%	\centering
%	\begin{tabular}{lp{20pt}lcc}
%		%\toprule
%		\multicolumn{3}{c}{Origin of jobs} &\multicolumn{2}{c}{Demand scenarios}   \\ 
%		\multicolumn{3}{c}{} & 2014 & 2014 with 90\% less export \\ 
%		\midrule
%		Jobs in the coal industry,	& \multicolumn{2}{l}{total}  & 82146 & 39601 \\ 
%		& \multicolumn{2}{l}{due to final demand}  & 48490 & \, 5945 \\ 
%		
%		& \multicolumn{2}{l}{due to other sectors}  & 33656 & 33656 \\ 
%		& 	 & Electricity, gas and water & \,\ 7580  & \,\ 7580 \\ 
%		&    & Metal ores& \,\ 4540  & \,\ 4540 \\ 
%		&    & Construction& \,\ 2456  & \,\ 2456  \\ 														
%		&    & Coke oven manufacture& \,\ 2360  & \,\ 2360   \\ 
%		%				&    & Other community activities& \,\ 1560  & \,\ 1560   \\ 
%		%				&    & Basic iron and steel & \,\ 1530  &  \,\ \,\ 769  \\
%		%				&    & Trade & \,\ 1530  & \,\ 1094  \\ 
%		%				&    & Food & \,\ 1276  & \,\ 1229 \\  
%		%				&    & Real estate activities & \,\ 1227  & \,\ 1219 \\
%		%				&    & Transport & \,\ 1165  & \,\ \, 930 \\  
%		\midrule
%		\multicolumn{3}{l}{Jobs in other sectors, due to the coal sector}  & 57542 &  13000 \\ 
%		& 						 & Transport & 18667 &  \ \,2289 \\ 
%		& 						 & Trade & 13200 &  \  1618 \\ 
%		\bottomrule
%	\end{tabular}
%	\caption{\label{coalExport_BAUvs2DS_2035}Comparison of the impact of export demand on the number of jobs related to the activity in the coal mining sector (SIC 21). The percent difference in coal exports between the two scenarios is equal to that between BAU and 2DS in 2035 as projected by CPI. The BAU scenario is approximated by the South African economy as in the year 2014.}
%\end{table}



\subsection{Electricity}\label{elec}

\begin{figure}[!t]
	%	\centering
	\hspace{-10pt}\includegraphics[width=1.1\textwidth]{../graphs/costPush_I4_10.pdf}
	\caption{\label{costPush_I4_10} The above graph shows the increase in the price of all sectors after a 10\% increase in the price of coal, assuming that the cost increase is entirely transferred, without affecting profitability of any of the sectors.}
\end{figure}

CPI projects that under both scenarios, the price electricity producers would pay for coal would increase with about 5,6\% per year on average. If this price increase is fully transferred to the electricity price, it would increase by about 1\%. Figure \ref{costPush_I4_10} shows the increase in the price of all sectors after a 10\% increase in the price of coal. %The resulting increase of the price of electricity is about 1,8\%.
The price of electricity has been rising for years in South Africa, and is expected to continue on this trajectory. For example, \citep{goliger2018electricity} projects tariff scenarios up to 2041 with yearly 5.8\% price hikes on the long term.

A set of two studies investigates whether the continued rise of electricity tariffs might make industrial demand \citep{goliger2018electricity} and household electricity demand \citep{goliger2018household} more elastic, by inducing tipping points. Historically, electricity demand has been rather inelastic to the price increases (references...). However, more recently, industrial demand might become more elastic because of the viability for firms to invest in own generation capacity. \citep{goliger2018electricity} finds that under the low, moderate and high price trajectories of electricity sold by Eskom up to 2041, out of the 21 large companies included in the study, even under the low price assumption, for more than half of the firms investment in own generation is profitable, within the modeling assumptions.



%\subsubsection{Reduced demand due to increase in electricity-price}

%An increase in the price of electricity can imply reduced demand. Motivated by the last (two) decades or price evolution, several studies estimate aggregate and sectoral price-elasticities (\cite{inglesi2010forecasting} and \cite{goliger2018electricity}, references therein). 


%\subsubsection{Change in the electricity generation-mix}
%
%To evaluate the effect of a lower share of coal in the generation mix in 2DS versus BAU, a corresponding Leontief matrix needs to be constructed. 
%
%To determine how the technical coefficients of sector SIC 41 evolve with different shares of coal, keeping the total volume of electricity generated constant, the coefficients are divided in three groups : the ones that are necessary for coal production, the ones that are necessary for one of the other technologies and the ones that are associated with both of the former groups. 
%Table \label{GenerationMixcoefficients} specifies the dominant inputs of the electricity sector and to which of the three groups they are associated. The technical coefficients of the third group are divided in two shares, such that, calibrating all prices in the 2014 IOT to 1, the cost share of coal in te IOT is equal to the cost-share of coal as stated in the 2014 Integrated report from Eskom of the breakdown of the primary energy cost according to generation technology (68\%). Knowing that coal provided in 2014 90\% of $GWh$, then when the share of coal diminishes by a factor $1 - \lambda$, all the coefficients (and shares) associated with coal diminish with ($1-\lambda$) and the others with $1 + 90/10 \, \lambda$.
%
%Because in 2014 coal is cheaper than the average of the other technologies, lower share of coal result in a higher generation cost.
%
%\li{Construct elec technology with GTAP data}
%
%\begin{table}[ht]
%	\centering
%	\begin{tabular}{lclc}
%		\toprule
%		%\midrule
%		\multicolumn{2}{l}{Input, SIC code}	& Technology used for  &  \% cost (2014) \\ 
%		\midrule
%		Coal and lignite & 21 &  Coal &  42.2    \\
%	 	Electricity, gas and water & 41 &  & 13.2     \\
%		Gas &412 & Other &     \\
%		Transport &71 \& 74 &  Coal  &   6.1   \\
%		Coke oven manufacture & 331 \& 332 & Coal &  4.0 \\
%		Nuclear fuel & 333 \& 334 & Other & 0.3   \\
%		\bottomrule
%	\end{tabular}
%	\caption{\label{GenerationMixcoefficients}The technical coefficients of the Leontief matrix that capture the generation mix. From left to right, the columns specify from what sectors inputs are used by sector 41 (Description and SIC code), whether they are used for generating electricity by coal plants are any other generation technology, and what share of total output the input represents. As a reference, intermediate inputs represent 39\% and employee compensation 17\% of output. Source : \cite{IOT2014}.}
%\end{table}
%
%
%\begin{table}[ht]
%	\centering
%	\begin{tabular}{rlrrr}
%		\toprule
%		%\midrule
%		SIC code	& Description  & Output$^*$ & Revenue$^{**}$ & \%   \\ 
%		\midrule
%		4 & Electricity, gas and water supply & 221 633 & & 100 \\ 
%		41 & Electricity, gas, steam and hot water supply &163 081& & 73.6 \\ 
%		411 & Production, collection and distribution of electricity &  &  &  \\ 
%		 & Eskom &  & 147 691 & 66.6 \\ 
%		 41111 & Generation 	  	  	&& & \\
% 		41112 & Distribution of purchased electric energy only 	&&  	&  	  	\\
%		41113 & Generation and/or distribution for own use & 8 454& & 3.8\\
%		412 & Manufacture of gas; distribution of gaseous fuels & & & \\ 
%		413 & Steam and hot water supply &  & &\\ 
%		42 & Collection, purification and distribution of water & 58 552 & &26.4 \\ 
%		\bottomrule
%	\end{tabular}
%	\caption{\label{SIC41composition}The composition of sector SIC 41, Electricity, gas, steam and hot water supply. Output and revenue are in million Rand. The estimates are based on 2014 data. The estimate for the value for subsector 41113 is based on how much sector 41 uses from itself, as stated in the IOT. Sources : \href{http://www.statssa.gov.za/additional_services/sic/mdvdvmg4.htm}{Stat SA website, SIC 41 description}, (*) 2014 IOT in current prices \citep{IOT2014}, (**) \href{http://www.eskom.co.za/IR2015/Documents/EskomIR2015single.pdf}{Integrated report}, \citep{Eskom2014AR}.}
%\end{table}
%
%
%\begin{table}[ht]
%	\centering
%	\begin{tabular}{lrrlr}
%		\toprule
%		%\midrule
%		Primary energy source	& Cost, mR  & Cost, \% & IOT  & IOT\\ 
%		\midrule
%		Coal and gas & 51 554   &  61.8   & Input I4 to I33 &\\
%		Nuclear  fuel &  530  &   0.6  & Input I17 to I33& \\
%		OGCT fuel & 9 546   &   11.4   & Input I33 to I33  & \\
%		Electricity  imports & 3 679   &  4.4  & I33 Import & \\
%		IPPs &  9 453  &  11.3  & Input I33 to I33 & \\
%		Environmental levy & 8353 &   10.0   & I33 Taxes & \\
%		Other &  310  &  0.4  & Input I4 to I33(?) 	& \\
%		\bottomrule
%	\end{tabular}
%	\caption{\label{EskomPrimaryEnergy}Eskom's Primary energy cost breakdown 2014, million Rand. Source : \href{http://www.eskom.co.za/IR2015/Documents/EskomIR2015single.pdf}{Integrated report}, \citep{Eskom2014AR}.}
%\end{table}
%
%
%
%
%\subsection{Oil}
%
%Global demand for crude oil in 2035 would be $24\%$ lower than in BAU and the price $20\%$ lower. 

%\subsection{Platinum group metals}

\subsection{Oil imports}

\begin{figure}[!t]
	\centering
	\includegraphics[width=\textwidth]{../graphs/Oil-Import_CPI_VolPrice_BAU_2DS.pdf}
	\caption{\label{Oil-Import_CPI_VolPrice_BAU_2DS}\small Projections for the period 2018-2035 of the volume of crude oil imported by South Africa in million barrels oil equivalent (mboe), the import price in USD/barrel and the total cost of imports in USD under CPI's business as usual (BAU) scenario and 2 degrees (2DS) scenario. Source : CPI.}
\end{figure}

South Africa has no own petroleum resources and imports all of domestic demand. Figure \ref{Coal-Export_CPI_VolPrice_BAU_2DS} presents CPI's projections over the period 2018-2035 of imported volumes and import prices of crude oil under BAU and 2DS scenarios. Because of lower global demand for petroleum products under 2DS, the price on the international market for crude oil is projected to be lower than under BAU. Because of the lower price, South African demand for crude oil is higher under 2DS. 

The refining sector (which is part of the Coke oven manufacture sector in the IOT) is with 0.91 jobs created per million Rand of final demand one of the smallest job creators \footnote{The sectors where the least jobs are created per Rand of final demand are Electronic valves and Medical appliances, because these goods are mainly imported.} (see figure \ref{Decomposition_direct_indirect_absolute}). Both the number of jobs created within the sector (0.24) and throughout the economy (0.67) per million Rand are below the economy's averages of 1.3 and 0.94. This is due to a low labour share, slightly higher wages than the economy's average and the above average import of inputs. 

Under 2DS, the demand for petroleum products is higher and thus a larger volume of crude oil is refined. In the middle of the 2020's, under 2DS the larger refined volume would sustain about 8000-9000 more jobs throughout the economy, of which about 2300 within the sector itself, another 2300 in the Trade sector and 800 in the Transport sector.



\section{Conclusion / summary}

\begin{table}[!t]
	\centering
	\begin{tabular}{lp{180pt}p{180pt}}
		\toprule
	 &\multicolumn{2}{c}{Net present value} \\	
	 \midrule													
	& CPI & IO \\	
 %\midrule													
& NPV capital and operating revenue 8\% discount rate& NPV export revenue - imports (13\%), 8\% discount rate \\														

	 \midrule													
Whole economy	& 29.4 billion USD & 29.1 billion USD \\	
		\bottomrule
	\end{tabular}
	\caption{\label{External risk}Risk allocation CPI / IO.}
\end{table}



The present study is devised as a complement to the analysis conducted by CPI of the low-carbon transition risk faced by South Africa over the period 2018-2035. The low-carbon transition risk is the risk that would result from the world following a development trajectory compatible with global warming limited to 2°C (a 2DS senario), versus development according to a business as usual (BAU) scenario. The CPI study conducts a sectoral and firm-level study. The present analysis adopts a macroeconomic perspective with a simple input-output model.

The interest in %conducting th
conducting a complementary analysis stems from the difference in the methodologies adopted. Overall, both methodologies allow to make assessments of very different aspects of the transition risk. However, in some cases, the different methodologies assess the same aspects. %That the types of assessments obtained are largely different, make
%The type of results obtained from the sectoral- and firm-based analysis by CPI and from the present macroeconomic IOT analysis of the external transition risk overlap in part.
Where there is overlap, the assessments can be compared. Differences can be traced to the assumptions behind both methodologies and the type of information present in the data used. This is the case for the risk %--- arising from global oil and coal export/import trajectories under BAU and 2DS scenarios --- 
to employment and to the total value of risk. The explicit risk allocation over economic actors can also be compared to some extent between both studies. %for estimates, under BAU and 2DS scenarios, of employment in the coal sector --- both due to foreign and domestic demand --- and for the revenue different sectors and agents obtain stemming from coal exports. The difference between the estimates under both scenarios is a measure, in each methodology, for the number of jobs and the revenue at risk.
Before discussing the comparable results from both studies, a recap is given of the scope and methodology of the CPI study. This makes clear how the input-output study is related to CPI's. 

The CPI analysis covers both external risk, due to factors outside of the influence of South-Africa, as well as risk within the control of the South-African government, stemming from energy and industrial policies. External risk arises from lower volume and export prices for coal, lower oil import prices, changes in metals and minerals demand and lower adaptation costs under a 2DS.
% domestic sectors that would come under pressure in the case of a global 2DS scenario. sectors that are most exposed to transition risk. These are identified as the coal and oil sectors, 
The risks within the influence of the South African government arise from 
%Under the risks arising from domestic policy falls risk stemming from 
choices for the development of the domestic power sector, which majorly impacts the coal mining sector, and also for the coal-to-liquids and oil refinery industries. Some of these risks were assessed by CPI with a detailed quantitative analysis, and some were qualitatively analyzed. Both the total value of the risk was estimated, as the allocation of the risk over the economy's actors, i.e. investors, government, workers. To assess how risk is ultimately distributed, CPI proceeds in three stages. First the explicit risk is calculated, as determined by demand addressed to each sector, regulation, policy and contracts and assuming companies are allowed to fail. Secondly, CPI models how this risk could be transferred within the economy by companies and government trying to mitigate their risk. Finally, when the residual risk cannot be borne by a company (leading for example to default), the risk will be passed on to investors or the government. 

CPI estimates the total value at risk at 123 billion USD, of which the explicit risk allocation puts close to 80\% on the shoulders of investors. However, after taking into account implicit risk transfers and transfers arising through contingent liabilities, more than half the risk would fall on the South African government. Finally, the CPI study makes recommendations and suggests policies to mitigate the low-carbon transition risk.

The present IO study is more restricted in scope. It estimates the total value of external risk arising from the difference in international demand for coal and oil under BAU and 2DS scenarios. In terms of risk allocation, it allows to asses the explicit distribution over industrial sectors. The distribution of the risk over actors is mostly limited to a distribution over workers on the one hand and investors, as a homogeneous group, on the other. Distribution of risk to government can be assessed to some extent via estimations of tax receipts. To further distribute risk over actors, information would need to be incorporated on who are the invested actors in the firms that make up the affected sectors. 

The coal and oil sectors are selected because for these, CPI conducts a detailed quantitative analysis. %, calculating the value at risk for key economic actors impacted : South African firms (SOE's, international majors, state-owned banks, BEE's and other SA firms), local and national government and workers. 
The starting points for the analyses are export/import volume and price trajectories for coal and oil. In the CPI analysis these are used as input to the sectoral and firm-level models to yield income projections for ... For the coal sector, total demand (export and domestic) is allocated over different mines according to contracts, location, type of coal mined etc.\lies{à compléter}. In the present study, the same volume and price trajectories are used in conjunction with a demand-driven IO-model to assess how the difference in demand between BAU and 2DS scenarios impacts not only the primarily affected importing and exporting sectors, but the whole economy. 





%In addition,  just as the CPI analysis yields many results 
The IOT analysis %is not equipped for, the latter 
allows to make a systematic assessment of how the export/import trajectories under both scenarios spread to possibly all sectors through the procurement relationships linking the economy.	% throughout the South-African economy. 
The CPI analysis makes detailed assessments of how the transition risk is spread over individual companies, mines and assets and on how this risk might be ultimately transferred to other agents, in particular the State. However, whereas it does provide an assessment of the risk to some infrastructure that supports export/import activity --- such as the Richard's Bay Coal line, asset owned by Transnet --- the methodology used is not equipped to make an estimate of how %systematically assess how 
different export-demand trajectories impact the economy as a whole in a systematic manner. In addition, the simple macroeconomic IO model allows to make projections of how risk to final demand might impact jobs, household income, consumption, investment and GDP.

The IO analysis projects that under a BAU scenario, the coal industry would sustain close to 80.000 jobs in 2035 and close to 61.000 under a 2DS scenario (table \ref{coalExport_BAUvs2DS_2035}). The difference stems from the loss of the volume of exported coal, which would put about 19.000 jobs at risk. The essential assumptions underlying these estimates are 1) that the number of workers is dictated by the number of tons mined, and not by the type of coal extracted (thermal coal or higher CV) or the price obtained per ton, and 2) that the productivity of all mines, whether they are multi-product mines, old or new, is the same. CPI \lies{Why the very similar projections under BAU and 2DS ?} projects 25,000-35,000 jobs lost under both scenarios. Under 2DS, there would be an earlier reduction of about 7,000 jobs (coal-related?) with respect to BAU. Towards the end of the period, the sector would sustain about 50,000 jobs. 

In addition, the IOT analysis projects close to 21,000 additional jobs at risk in other sectors of the economy, induced by lower export demand under 2DS versus BAU. A third of these is situated in the Transport sector (table \ref{coalExport_BAUvs2DS_2035_table_employment}).

The revenue at risk from coal exports represents about 1.93\% of 2017 GDP. In addition, the lost jobs represent lost income for households and thus diminished consumption, which further impacts GDP. This effect would lower 2014 GDP with another 0.25\%. %in addition to the 2.36\% directly lost due to coal exports. This effect is to be taken as a general indication. Under 2DS, the 39674 jobs would disappear over a period of 18 years. Aided by accompanying policies, new economic activity and jobs can be created throughout this period.
Less demand throughout the economy also lowers investment needs, by about 0.03\% of 2017 GDP. Overall, the value lost to reduced coal exports under 2DS and the indirect effects of diminished consumption and investment represent 2.2\% of 2017 GDP.



A low-carbon transition also implies lower global demand for petroleum products. South Africa imports all of its domestic demand in crude oil and refines domestically. %Figure \ref{Coal-Export_CPI_VolPrice_BAU_2DS} presents CPI's projections over the period 2018-2035 of imported volumes and import prices of crude oil under BAU and 2DS scenarios. 
Because of lower global demand for petroleum products under 2DS, the price on the international market for crude oil is projected to be lower than under BAU. Because of the lower price, South African demand for crude oil is higher under 2DS. In the middle of the 2020's, the Coke oven manufacture sector (of which oil refinement is a part) will refine about 5\% more crude oil than under BAU. Despite the fact that it is one of the smallest job creators per million Rand of final demand, under 2DS the larger refined volume would sustain about 8000-9000 more jobs throughout the economy. About 2300 jobs would be situated within the sector itself, another 2300 in the Trade sector and 800 in the Transport sector.



\li{below, to be completed, suggestion to compare the explicit risk distribution over sectors / actors between both studies}


To measure the risk different agents bear, the CPI study uses the difference of the net present values under the 2DS and the BAU scenarios of the agent's future operating and capital flows. The operating cash flow is defined as total revenue excluding net capital income, minus all costs incurred, including the costs of inputs, taxes and interest expenses but excluding depreciation. The capital cash flow includes (is it investment cash flow, or does it include financing cash flow, but the latter is assumed zero I suppose ?) the net income from investments and the expenses minus income from buying/selling capital assets. The total cash flow thus defined, operating plus capital, is equal \lies{Correct or not, what is the difference if not ?} to gross operating surplus.

The biggest share of the external transition risk is due to the difference of coal export revenue under both scenarios. The transition risk due to coal exports amounts to 83.7 billion USD net present value\footnote{How is it calculated, what is the discount rate used?}. Of this total value at risk, according to CPI analysis, 66.7 billion USD of explicit risk would lie with investors and 17 billion USD with the government. The 66.7 billion USD further distributes between international majors, BEEs\footnote{Black Economic Empowerment firms}, state-owned banks, SOE's and other South African companies\lies{In what sectors?}. %(The numbers are given for coal export + domeestic + oil together.)

The IO analysis decomposes future revenue flows for the coal sector into imports of inputs, taxes and value added for the coal sector as well as for its directly and indirectly supplying sectors. The avoided cost of imported inputs, due to the reduced export revenue under 2DS, is an attenuating factor for the total value at risk. For demand addressed to the coal sector, about 13\% of revenue is spent on imported inputs, throughout the production chain. This only accounts for consumables. The share of investment (gross fixed capital formation) that is imported is for the coal sector larger than on average in the economy, due to the machines and equipment used in the coal mines and the mines' increasing mechanization over the years. In 2014, more than half of investments generated by final demand for coal were imported. %Thus, the total amount of revenue spent on imports is higher.

The remaining part of future revenue is distributed to government in the form of taxes, to workers in the form of salaries and to investors in the form of gross operating surplus. In 2035\lies{For comparison of results, use not 2035 shares but NPV with same discount rate?}, of the almost 90,000 million Rand (6.4 billion USD) at risk, about 2\% would be borne by government (taxes), 30\% by workers (salaries) and the remaining 55\% by investors (gross operating surplus). Of the risk borne by investors, ~35-40\% is situated with investors in the mining sector, xx\% in the transport sector, xx\% in the trade sector and xx\% in other sectors.




\clearpage

% SA.bib
\newpage
\bibliography{SA}
\bibliographystyle{apalike}
\clearpage

\appendix

\section{Mean employment content} \label{ECmean}

%\section{What is in the data ?}

\section{Correspondence tables}

\clearpage
\begin{table}[ht]
	\centering
	%\small
	\vspace{-36pt}\hspace{-10pt}\caption{\label{correspondance_OECD_ZA}Correspondence between the 50-sector-classification of the South African IOTs in national SIC classification and the 34-sector-classification in OECD IOTs following ISIC rev.3.1. Sources : \href{http://www.oecd.org/sti/ind/IOT_Industries_Items.pdf}{OECD code to ISIC rev. 3}, \href{https://www.statssa.gov.za/additional_services/sic/descrip6.htm}{SIC to ISIC}, \cite{IOT2014} and \citetalias{united2002isic}.}
	\vspace{8pt}%\hspace{-10pt}
	\begin{adjustbox}{width=\textwidth}
	%\scriptsize
	\normalsize
		\begin{tabular}{lp{500pt}lll}
		\toprule
		Code & Industry Description & SIC & ISIC rev.3.1 & OECD code \\ \arrayrulecolor{black!30}\midrule
		I1 & Agriculture, hunting and related services & 11 & 01 & C01T05 \\ \midrule
		I2 & Forestry, logging and related services & 12 & 02 & C01T05 \\ \midrule
		I3 & Fishing, operation of fish hatcheries and fish farms & 13 & 05 & C01T05 \\ \midrule
		I4 & Mining of coal and lignite & 21 & 10 & C10T14 \\ \midrule
		I5 & Mining of gold, uranium and metal ores  & 23-24 & 12-13 & C10T14 \\ \midrule
		I6 & Other mining and quarrying & 25 & 14 & C10T14 \\ \midrule
		I7 & Manufacture of food products & 301-4 & 151-154 & C15T16 \\ \midrule
		I8 & Manufacture of beverage and tobacco products & 305-6 & 155\_16 & C15T16 \\ \midrule
		I9 & Spinning, weaving, finishing of textiles and manufacture of other textiles & 311-2 & 17 & C17T19 \\ \midrule
		I10 & Manufacture of knitted, crocheted fabrics, wearing apparel, fur articles and dying of fur & 313-5 & 18 & C17T19 \\ \midrule
		I11 & Tanning and dressing of leather, manufacture of luggage, handbags, saddler and harness & 316 & 191 & C17T19 \\ \midrule
		I12 & Manufacture of footwear & 317 & 192 & C17T19 \\ \midrule
		I13 & Manufacture of products of wood, cork, straw and plaiting materials%; and sawmilling and planning of wood 
		& 321-2 & 20 & C20 \\ \midrule
		I14 & Manufacture of paper and paper products & 323 & 21 & C21T22 \\ \midrule
		I15 & Publishing, printing and service activities related to printing; reproduction of recorded media & 324-6 & 22 & C21T22 \\ \midrule
		I16 & Manufacture of coke oven products; and petroleum refineries/ synthesisers & 331-2 & 231-232 & C23 \\ \midrule
		I17 & Processing of nuclear fuel; and basic chemicals & 333-4 & 233 & C23 \\ \midrule
		I18 & Manufacture of other chemical products, and man-made fibres & 335-6 & 24 & C24 \\ \midrule
		I19 & Manufacture of rubber products & 337 & 251 & C25 \\ \midrule
		I20 & Manufacture of plastic products & 338 & 252 & C25 \\ \midrule
		I21 & Manufacture of glass and glass products & 341 & 261 & C26 \\ \midrule
		I22 & Manufacture of non-metallic mineral products (nec) & 342 & 269 & C26 \\ \midrule
		I23 & Manufacture of furniture & 391 & 36 & C36T37 \\ \midrule
		I24 & Manufacture nec and recycling nec & 392\_395 & 37 & C36T37 \\ \midrule
		I25 & Manufacture of basic iron and steel and casting of metals & 351\_353 & 271\_273 & C27 \\ \midrule
		I26 & Manufacture of precious metals and non-ferrous metals & 352 & 272 & C27 \\ \midrule
		I27 & Manufacture of structural metal products, tanks, %reservoirs 
		and steam generators, and other %fabricated metal products, metalwork service activities 
		& 354-5 & 28 & C28 \\ \midrule
		I28 & Manufacture of general/special purpose machinery, %special purpose machinery, 
		household appliances and office %, accounting and computing 
		machinery & 356-9 & 29-30 & C29, C30T33X \\ \midrule
		I29 & Manufacture of electrical machinery and apparatus nec & 36 & 31 & C31 \\ \midrule
		I30 & Manufacture of %electronic valves, tubes, other 
		electric components; television, %and 
		radio, %transmitters, apparatus for line 
		telephony, %and line telegraphy; and television and radio receivers, sound or 
		video %recording or reproducing 
		apparatus %and associated goods 
		& 371-3 & 32 & C30T33X \\ \midrule
		I31 & Manufacture of medical and measuring appliances, %and instruments and appliances for measuring, checking, testing, navigating and other purposes; manufacture 
		of optical instruments %and photographic equipment
		; %and watches and 
		clocks & 374-6 & 33 & C30T33X \\ \midrule
		I32 & Manufacture of motor vehicles and parts, %bodies (coachwork) for motor vehicles, 
		locomotives, aircraft, spacecraft, trailers and semi-trailers; %motor vehicle and engine parts and accessories 
		; building and repairing ships and boats%, and manufacture of transport equipment not elsewhere classified. 
		& 381-387 & 34-35 & C34, C35 \\ \midrule
		I33 & Electricity, gas steam and hot water supply & 41 & 40 & C40T41 \\ \midrule
		I34 & Collection, purification and distribution of water & 42 & 41 & C40T41 \\ \midrule
		I35 & Construction & 5 & 45 & C45 \\ \midrule
		I36 & Wholesale and commission trade, retail trade; and sale, maintenance and repair of motor vehicles and motor cycles, retail trade in automotive fuel & 61-63 & 51\_52\_50 & C50T52 \\ \midrule
		I37 & Hotels and restaurants & 64 & 55 & C55 \\ \midrule
		I38 & Transport (land, water, air) ; %and supporting and auxiliary transport activities, 
		activities of travel agencies & 71-74 & 60-63 & C60T63 \\ \midrule
		I39 & Post and telecommunications & 75 & 64 & C64 \\ \midrule
		I40 & Financial intermediation, except insurance and pension funding & 81 & 65 & C65T67 \\ \midrule
		I41 & Insurance and pension funding, except compulsory social security  & 82 & 66 & C65T67 \\ \midrule
		I42 & Activities auxiliary to financial intermediation & 83 & 67 & C65T67 \\ \midrule
		I43 & Real estate activities & 84 & 70 & C70 \\ \midrule
		I44 & Renting of machinery and equipment, without operator, and of personal and household goods & 85 & 71 & C71 \\ \midrule
		I45 & Research and development  & 87 & 73 & C73T74 \\ \midrule
		I46 & Computer and related activities; and other business activities & 86\_88 & 72\_74 & C72, C73T74 \\ \midrule
		I47 & Other community, social and personal service activities (including government) & 91\_94 & 75\_90 & C75,  \\ \midrule
		I48 & Education & 92 & 80 & C80 \\ \midrule
		I49 & Health and social work & 93 & 85 & C85 \\ \midrule
		I50 & Other service activities nec & 95\_96\_99 & 91\_92\_93 & C90T93 \\
		\arrayrulecolor{black}\bottomrule	
	\end{tabular}
	\end{adjustbox}
	\thispagestyle{empty}
\end{table}

\clearpage
\section{Estimation of separate domestic and import IOTs}\label{IOTopti}


%\section{Decomposition}
%
%Equation \label{allUSes} can be rewritten as :
%\begin{equation}
%%y_i + m_i - d_i^m= \sum_j \Bigg(\frac{z_{ij}^d}{y_i + m_i-d_i} + \frac{z_{ij}^m}{y_i + m_i-d_i} \Bigg) (y_i + m_i) + d_i^d
%y_i + m_i - d_i^m= \sum_j \big(z_{ij}^d + z_{ij}^m \big) + d_i^d, 
%\end{equation}
%where $d^r = c^r_i + g_i^r + i_i^r + inv_i^r + x_i^r$ , $r\in\{d,m\}$ denote final demand for domestically produced goods and services and for imported goods. The aggregates $m_i - d_i^m$ are the imports used as intermediate inputs, or differently put, the importation needed for domestic production.
%
%We are interested in, given a final demand vector for domestically produced goods $\boldsymbol{d}^d$, not only how much domestic production this entails, but also how much importation this generates because of the use of imported goods as intermediate inputs. To this end, we define modified matrices of technical coefficients $\tilde{A}^r$ as the matrices with elements $\tilde{a}^r_{ij}$ : 
%\begin{align}
%\tilde{a}^r_{ij} = \frac{z_{ij}^r}{y_i + m_i - d_i^m} \label{defAtilde}
%\end{align}
%These matrices tell us how much intermediary inputs, respectively of domestic and of imported origin, are needed per unit of domestic output and imported intermediate goods. Additionally, we define the technical coefficients matrix $\tilde{A}= \tilde{A}^d + \tilde{A}^m$ that states the total intermediary inputs, of any origin, per unit of imports + domestic output.
%
%%We then define the matrices $Y$ and $M_{int}$ such that :
%
%Given $D^d$ the matrix with demand $d^d$ on the diagonal and 0's elsewher, we then define the matrix $\mli{YM}$ as :
%\begin{equation}
%\mli{YM} = (\mathbb{I} - \tilde{A})^{-1} \, D^d \label{YM}
%\end{equation}
%The element $ym_{ij}$ of $\mli{YM}$ is the sum of domestic production and imported intermediate goods of sector $i$ needed to satisfy final demand by sector j $d_j^d$.
%
%\li{The column sums of $M_{int}$ are the same as the column sums of $Z^m$, but are the two matrices the same ?}
%
%We now want to split matrix $\mli{YM}$ up into domestic output and intermediate inputs that were imported.
%
%For this we will decompose the factor that is analogous to the Leontief inverse :
%
%\begin{align}
%(\mathbb{I} - \tilde{A})^{-1} &= (\mathbb{I} - \tilde{A}^d - \tilde{A}^m)^{-1} \nonumber \\
%&= (\mathbb{I} - \tilde{A}^d)^{-1} + (\mathbb{I} - \tilde{A}^m)^{-1} - \mathbb{I} + rest \label{decomp1}
%\end{align}
%
%The first term on the right hand side measures domestic output necessary to satisfy demand $D^d$. We take a closer look at the second and third terms at the right-hand side of  \ref{decomp1} by developing the power series : 
%\begin{equation}
%(\mathbb{I} - \tilde{A}^m)^{-1} - \mathbb{I} = \sum_{k=1}^{+\infty}  (\tilde{A}^m)^k  \label{importedInputsneeded}
%\end{equation}
%Whereas the significance of $\tilde{A}^m D^d$ is quite clear --- it determines the imported intermediate inputs needed to satisfy domestically produced final demand, the meaning of the following terms $(\tilde{A}^m)^k D^d$ with $k>1$ seems nebulous when interpreted along the same line. This line of interpretation looks at the matrix $\tilde{A}^m D^d$ row-wise : we seek an answer to the question how much output from each sector is needed to satisfy the demanded production from other sectors. In this case, the "output from each sector needed" are imported goods. The term $\tilde{A}^m \tilde{A}^m D^d$ does not make sense in this interpretation : what are the import needed to generate the imports ? Rather, one can think of $\tilde{A}^m D^d$ column-wise and interpret each element $a^m_{ij}\,d^d_j$as a contribution to the cost of producing a good $j$, or a contribution to the value of the good $j$ by using the intermediate input $i$, obtained through importation. Then, $\tilde{A}^m \tilde{A}^m D^d$ also takes on a clear significance : it represents the value generated indirectly by using imported intermediate inputs to %generate value by 
%produce output that was produced with imported inputs. 
%%Instead of thinking in terms of the output of sector $i$ needed to satisfy demand from sector $j$, think of it vice-versa : it is the value of output of sector $j$ originating from using input $i$. Of course, a single input cannot produce a good. This is why the former sentence contains "value of". Using input $i$ contributes to the cost of production or to the value of good $j$. 
%
%Why would one determine how much imported goods are needed to import goods ? 
%
%To able to create the value generated by using imported inputs, a certain amount of imported inputs are needed, etc.
%Beware, is is using the imported inputs that creates value domestically ; producing obviously doesn't.
%
%It is now clear that the terms \ref{importedInputsneeded} contribute to the amount of imported goods needed to satisfy domestically produced final demand. The terms in the $rest$ term contain both $\tilde{A}^d$ and $\tilde{A}^m$ factors. For example $\tilde{A}^m \tilde{A}^d D^d$ represents the value generated by using imported goods to produce the intermediate inputs necessary for the production of the intermediate inputs $\tilde{A}^d D^d$, where this last term can be worded as the value created by using locally produced inputs to produce demand $D^d$.
%
%The rest term is known, given by $$ rest =(\mathbb{I} - \tilde{A})^{-1} - (\mathbb{I} - \tilde{A}^d)^{-1} - (\mathbb{I} - \tilde{A}^m)^{-1} + \mathbb{I}.$$
%It contains both value created by using imported and domestically produced inputs and will her be roughly split up according to the last step in the production process :
%$$
%rest = \tilde{A}^d (\tilde{A}^m + T) + \tilde{A}^m(\tilde{A}^d + T)
%$$
%The first term on the right-hand side is counted as domestic production and the second term as imports. (The second term are all the imports needed/used to produce the intermediary inputs --- direct inputs and the chain of indirect inputs --- that have been produced using both imported and domestic inputs.)
%
%The unknown tail $T$ is easily determined : $$ T = ( \tilde{A}^d +  \tilde{A}^m)^{-1}\, ( \tilde{A}^m  \tilde{A}^d +  \tilde{A}^d \tilde{A}^m)$$
%Matrix $\mli{LM}$ as given in eqn. \ref{YM} is then split up as :
%\begin{align}
%\tilde{Y} &= (\mathbb{I} - \tilde{A}^d)^{-1} \, D^d + \tilde{A}^d \big(\tilde{A}^m +  ( \tilde{A}^d +  \tilde{A}^m)^{-1}\, ( \tilde{A}^m  \tilde{A}^d +  \tilde{A}^d \tilde{A}^m)\big) \, D^d \\
%\tilde{M} &= (\mathbb{I} - \tilde{A}^m)^{-1} \, D^d - \mathbb{I} + \tilde{A}^m \big(\tilde{A}^d +  ( \tilde{A}^d +  \tilde{A}^m)^{-1}\, ( \tilde{A}^m  \tilde{A}^d +  \tilde{A}^d \tilde{A}^m)\big) \, D^d
%\end{align}
%
%
%%The value created by producing a product is the value of the product. The value of using the product as intermediate input is 
%
%%To resolve this issue, we determine the relationship between $\tilde{A}^m$ and $A^m$, where the latter is the technical coefficients matrix for imported goods, defined as usual. Starting from \ref{defAtilde} in matrix notation :
%
%%\begin{align}
%%\tilde{A}^m &= Z^m \, (diag(\boldsymbol{y + m - d^m}) )^{-1} \nonumber\\
%%&=  Z^m \, (diag(\boldsymbol{y}))^{-1} \, \big( \mathbb{I} +  (diag(\boldsymbol{y}))^{-1} \, diag(\boldsymbol{m - d^m})   \big)^{-1} \nonumber\\
%%&= Z^m \, (diag(\boldsymbol{y}))^{-1} \, \sum_{k=0}^{+\infty} \big(-(diag(\boldsymbol{y}))^{-1} \, diag(\boldsymbol{m - d^m})\big)^k \nonumber\\
%%&= A^m  \, \sum_{k=0}^{+\infty} \big(-(diag(\boldsymbol{y}))^{-1} \, diag(\boldsymbol{m - d^m})\big)^k
%%\end{align}

\clearpage
\section{Coal in tons in the input-output table}

To estimate the use of coal in tons by all industrial sectors and final consumers, data are used from the South African Coal Sector Report from the SA Department of Energy \citep{coal2013DoE} and the Minerals Council Facts and figures \citepalias{MineralsCouncil2018Facts}. Demand by final consumers is estimated at 2 million tons (mt), by industry other than the power sector (I33 or SIC41) and coal to liquids (I16) is estimated at 16 mt. Coal to liquids is estimated to use 31 mt annually. Coal export in 2014 is 75.8 mt and total sales equals 269.2 mt. From these figures it is estimated that the power sector used 123.3 mt in 2014.

\section{Note on employment content decomposition}

Illustration of the decomposition of the decomposition of the numbers of jobs created within the sector to which demand is addressed and in all the other sectors.\lies{TODO}

\begin{figure}[!ht]
	\centering
	\includegraphics[width=0.95\textwidth]{../graphs/ECdecomposition-avg-PQ-direct_2014.pdf}
	\caption{\label{Decomposition_PQ_within}\footnotesize The net length of the horizontal bars represents for each sector the number of jobs created within the sector to which demand is addressed, compared to the economy's average employment content of 2.24. The number of jobs per sector is decomposed into five factors, compared to the average composition of the employment content over the whole economy, excluding SIC 01/ISIC P Private households with employed persons. The graph is to be read in the same fashion as figure \ref{Decomposition_PQ}. %The factors give an indication of the sources of the differences in employment content. The components of a horizontal bar situated to the left/right of the O mark are smaller/bigger than the average. The interpretation of each of the five components is as follows : 1) when Wages is bigger than the average, this means that wages are smaller in this sector than the average wage throughout the economy, 2) when Labour share is bigger than the average, the sector's labour share is bigger than the average, 3) when Taxes is bigger than the economy's average, the tax burden on the sector is smaller than the average, 4) when Local final demand is smaller than the average, thus on the left of the 0, then local demand + exports are relatively smaller and imports are relatively bigger, 5) when Intermediate imports is situated on the left, then the importation of intermediate inputs is higher than the economy's average.
	}
\end{figure}	

\begin{figure}[!ht]
	\centering
	\includegraphics[width=0.95\textwidth]{../graphs/ECdecomposition-avg-PQ-indirect_2014.pdf}
	\caption{\footnotesize \label{Decomposition_PQ_other}The horizontal bars show for each sector a decomposition of the employment content of each sector, from which is subtracted the number of jobs created within the sector to which demand is addressed. The resulting number of jobs is decomposed into five factors, compared to the average composition of the employment content over the whole economy, excluding SIC 01/ISIC P Private households with employed persons. The graph is to ne read in the same fashion as figure \ref{Decomposition_PQ}. %The factors give an indication of the sources of the differences in employment content. The components of a horizontal bar situated to the left/right of the O mark are smaller/bigger than the average. The interpretation of each of the five components is as follows : 1) when Wages is bigger than the average, this means that wages are smaller in this sector than the average wage throughout the economy, 2) when Labour share is bigger than the average, the sector's labour share is bigger than the average, 3) when Taxes is bigger than the economy's average, the tax burden on the sector is smaller than the average, 4) when Local final demand is smaller than the average, thus on the left of the 0, then local demand + exports are relatively smaller and imports are relatively bigger, 5) when Intermediate imports is situated on the left, then the importation of intermediate inputs is higher than the economy's average.
	}
\end{figure}

%\subsection{Employment content decomposition}\label{ec_appendix}
%
%
%\begin{figure}[!ht]
%	\centering
%	\thispagestyle{empty}
%	\includegraphics[width=\textwidth]{../graphs/ECdecomposition-avg-direct-indirect_2014.pdf}
%	\caption{\label{Decomposition_direct_indirect_avg}}
%\end{figure}	

%\section{Coal Export, graphs and tables}\label{coal export}
%
%\begin{table}[ht]
%	\centering
%	\renewcommand*{\arraystretch}{1.15}
%	\begin{tabular}{lp{20pt}lcccc}
%		\toprule
%		%		\multicolumn{3}{c}{Ori} &\multicolumn{3}{c}{Number of persons employed, 2035} \\ 
%		%\multicolumn{3}{c}{Origin of jobs} &\multicolumn{2}{c}{Export scenarios} &  \\ 
%		\multicolumn{3}{l}{Revenue destination } & \makecell{BAU\&2DS \\2018} & \makecell{BAU\ \  \\ 2035\ \ } & \makecell{\ \ 2DS \\\ \ 2035} & \makecell{Value at risk \\(in \ldots rand)}\\ 
%		\midrule
%		\multicolumn{3}{l}{Value added, total}   & 86.7  & 90.6 & 84.6 & \\ 	
%		& 						 & Wages & 28.9 & 20.5  & 33.4 \\ 
%		& 						 & Gross operating surplus & 55.2  &  67.2  & 48.7 &\\ 
%		& 						 & Taxes & 2.6 &  2.9 & 2.5 & \\
%		\midrule
%		\multicolumn{3}{l}{Value added, coal sector}  & 60.6 & 72.3 & 54.8 \\ 
%		& 						 & Wages & 17.9 &  12.7 & 20.8 \\ 
%		& 						 & Gross operating surplus & 41.5 & 57.6  & 32.9 & \\ 
%		& 						 & Taxes & 1.5 &  2.1 & 1.2 \\
%		\midrule
%		\multicolumn{3}{l}{Value added, rest of economy}  & 25.6  & 18.2 & 29.8 \\ 
%		& 						 & Wages & 10.9 & 7.8  & 12.7 \\ 
%		& 						 & Gross operating surplus & 13.7 &  9.7  & 15.8 & \\ 
%		& 						 & Taxes & 1.1 & 0.8  & 1.3 \\
%		\midrule
%		\multicolumn{3}{l}{Imports, total}  		 & 13.3  & 9.4  & 15.4 & \\ 	
%		& 						 & by coal sector & 8.1 &  5.8 & 9.4 & \\ 
%		& 						 & by rest of economy & 5.1 &  3.6  & 5.9 & \\ 
%		\bottomrule
%	\end{tabular}
%	\caption{\label{ExportRevenue_Decomposition_Diff_table}\small Comparison of the impact of export demand on the number of jobs related to the activity in the coal mining sector (SIC 21). The percent difference in coal exports between the two scenarios is equal to that between BAU and 2DS in 2035 as projected by CPI. The BAU scenario is approximated by the South African economy as in the year 2014.}
%\end{table}
%
%
%\begin{figure}[!h]
%	%	\centering
%	\hspace{-10pt}\includegraphics[width=1.1\textwidth]{../graphs/ValueAdded_Sector_Coal-Export_BAU_2DS.pdf}
%	\caption{\label{ValueAdded_Sector_Coal-Export_BAU_2DS}\small The graph shows the revenue from the export of coal in constant Rand (2014), over the period 2018-2035, under the business as usual scenario (BAU) on the left and under the 2°degrees (2DS) scenario on the right. The export revenue is decomposed into importation costs of intermediate goods (red), taxes, wages and gross operating surplus, for the coal sector (yellow) and for the rest of the economy (blue). Is is assumed that the prices changes of the exported coal impact tax revenue and gross operating surplus. All domestic prices and wages stay constant over the whole period, except  }
%\end{figure}
%
%
%\begin{figure}[!h]
%	%	\centering
%	\hspace{-10pt}\includegraphics[width=1.1\textwidth]{../graphs/NumberEmployees_Sector_Coal-Export_BAU_2DS.pdf}
%	\caption{\label{NumberEmployees_Sector_Coal-Export_BAU_2DS}\small The graph shows the number of jobs sustained by coal export demand per sector, over the period 2018-2035, under the business as usual scenario (BAU) on the left and under the 2°degrees (2DS) scenario on the right.}
%\end{figure}

%\clearpage
%\subsection{Uses and resources}
%
%Figures \ref{Uses_Resources_percent_2014} and \ref{Uses_Resources_abs_2014} give a visual summary of the input-output table. For each of the 50 sectors of the IOT, they provide an overview of how goods are obtained --- imported versus domestically produced --- displayed in the right-hand columns and how they are used in the left-hand column. How imports and production are used is divided up between intermediate consumption, final uses such as consumption (by households and government) and building of the capital stock (accumulation of the fixed stock and inventories) and for export. Left- and right-hand columns necessarily have the same height because of equation \ref{IOT}. Figure \ref{Uses_Resources_percent_2014} shows per sector the share, dissimulating the information on the relative size of the different sectors. Figure \ref{Uses_Resources_abs_2014} retains this last information.% ; however, due to the large size of some service sectors, the distribution of uses and resources is somewhat lost for many.  
%
%Figure \ref{Uses_Resources_abs_2014} shows from what sectors South Africa obtained export revenue in 2014 (the year for which the most recent IOT is available). These are in order of magnitude : Metal ores (SIC 23-24), Transport (SIC 71-74), Trade (SIC 61-63), Basic iron and steel (SIC 351 \& 353), Coal and lignite (SIC 21), Other services nec, Other mining (SIC 25), Motor vehicles (381-387). Three of these are services (Transport, Trade and Other services nec).
%
%
%\begin{landscape}
%	\thispagestyle{empty}
%	\begin{figure}[!ht]
%		%\centering
%		\vspace{-30pt}\hspace{-36pt}\includegraphics[width=1.5\textwidth]{../graphs/Uses_Resources_percent_2014.pdf}
%		\caption{\label{Uses_Resources_percent_2014}For a disaggregation of the economy into 50 sectors, the graph shows how each sector obtains its resources, through domestic production (Output, in light blue) or through Imports (in dark blue) in the right vertical bar. The left bar shows the use made of the resources : Exports (red), Consumption which represents both consumption by households and by government, Capital formation, which includes Fixed capital formation and changes in inventories (very pale blue and light orange) and Intermediatie consumption (orange). The value of each type of use and resource is in million Rand. The figures are for the year 2014. Data source : IO tables from Stats SA.}
%	\end{figure}	
%\end{landscape}
%
%
%\begin{landscape}
%	\thispagestyle{empty}
%	\begin{figure}[!ht]
%		%\centering
%		\vspace{-30pt}\hspace{-36pt}\includegraphics[width=1.5\textwidth]{../graphs/Uses_Resources_abs_2014.pdf}
%		\caption{\label{Uses_Resources_abs_2014}For a disaggregation of the economy into 50 sectors, the graph shows how each sector obtains its resources, through domestic production (Output, in light blue) or through Imports (in dark blue) in the right vertical bar. The left bar shows the use made of the resources : Exports (red), Consumption which represents both consumption by households and by government, Capital formation, which includes Fixed capital formation and changes in inventories (very pale blue and light orange) and Intermediatie consumption (orange). The value of each type of use and resource is in million Rand. The figures are for the year 2014. Data source : IO tables from Stats SA.}
%	\end{figure}	
%\end{landscape}








\end{document}