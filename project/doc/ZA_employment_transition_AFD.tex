\documentclass[12pt,english]{article}
\usepackage[T1]{fontenc}
%\usepackage[latin9]{inputenc}
\usepackage[utf8]{inputenc}
%\usepackage[french]{babel}
%\usepackage{pdfpages}
\usepackage{geometry}
\geometry{verbose,tmargin=2.2cm,bmargin=2.2cm,lmargin=2.2cm,rmargin=2.2cm}
\usepackage[parfill]{parskip} % Activate to begin paragraphs with an empty line rather than an indent
\usepackage{babel}
\usepackage{varioref}
\usepackage{float}
%\usepackage[caption = false]{subfig}
\usepackage{amstext}
\usepackage{amsmath, amsthm} 
\usepackage{mathtools}
\usepackage{amsfonts}
\usepackage{mathrsfs}
\usepackage{amssymb,graphics,psfrag}
%\usepackage{dsfont}
\usepackage{hyperref}
\usepackage{gensymb}
\usepackage{graphicx}
%\usepackage{subfig}
\usepackage{setspace}
%\usepackage{subcaption,siunitx,booktabs}
\usepackage{booktabs}
%\usepackage{caption}
\usepackage{subcaption}
\usepackage{pdflscape}
\usepackage{adjustbox}
\usepackage{esint}
%\onehalfspacing
\usepackage{cite}
\usepackage{natbib}
\usepackage{textcomp}
\usepackage{multirow}
\usepackage{array}
\usepackage{longtable}
\usepackage[dvipsnames,table]{xcolor}
\hypersetup{colorlinks=true,
			linkcolor=black,
			citecolor=Blue,
			urlcolor=Blue}
%\usepackage{tikz}
%\usetikzlibrary{positioning}
\usepackage{makecell}
\usepackage{tabularx}
\usepackage{array}

\newcommand{\mli}[1]{\mathit{#1}}


\PassOptionsToPackage{normalem}{ulem}
\usepackage{ulem}
\usepackage{babel}
% \usepackage[version=3]{mhchem}
 
\usepackage[textsize=small,textwidth=2cm,shadow]{todonotes}
\usepackage[]{todonotes}
\newcommand\lies[2][]{\todo[color=orange!50,#1]{ldf: #2}}
\newcommand\li[1]{\lies[inline]{#1}}
%\newcommand\etienne[2][]{\todo[color=yellow!50,#1]{es: #2}}
%\newcommand\es[1]{\etienne[inline]{#1}}
%\newcommand\antonin[2][]{\todo[color=green!50,#1]{ap: #2}}
%\newcommand\ap[1]{\antonin[inline]{#1}}
%\newcommand\rb[1]{\textcolor{green!80!black}{#1}}
%\newcommand\ru[1]{\textcolor{green!80!black}{\sout{#1}}}


%\title{Effect on employment from a low-carbon transition on South Africa} %On the non-neutrality of climate impacts and policies \ \\ \Large Working paper, not for distribution.

%\author{Liesbeth De Fossé}

\begin{document}
\defcitealias{united2002isic}{ISIC rev. 3.1}
\defcitealias{IOT2014}{StatSA, IOT 2014}

\newcommand{\STAB}[1]{\begin{tabular}{@{}c@{}}#1\end{tabular}}

\tableofcontents

\section{Introduction}

%The aim of the study is to measure the impact on employment of the changing international conditions in which the South African economy operates in the case of a low-carbon transition. %For example, a 2° scenario worldwide implies demand for coal will decrease. The South African coal sector can thus be impacted, both through a changed (lowered) export price as well as through lower export volume. When the demand and the profitability of a sector, in this case the mining sector, are expected to decrease, employment in the sector will follow the sector's trend.


This study seeks to evaluate from a macroeconomic perspective the effect on the South African economy of a global transition to a low-carbon economy. The study focuses on the transition risk --- i.e. the risk arising from the adoption of an economic apparatus on a world-wide scale that would make up a low-carbon economy --- due to external sources, outside of the influence of South African decision-makers. This external risk consists in the fact that the demand on the international market for certain goods exported and imported by South Africa, %and their price % on the international market, 
is impacted by a low-carbon transition and unanticipated at the time investments were made in assets to produce/transform exported/imported goods. 
A low-carbon transition of the South African economy itself is not considered here. In this aspect the present report follows the approach taken by the Climate Policy Initiative (CPI) who conducted an analysis of the external transition risk faced by South Africa taking a micro-economic, sectoral, approach. At times when judged necessary, risk is in CPI's analysis evaluated at the level of individual assets : for example, because for the coal-mining sector the effect of coal-export demand on the profitability of mines depends on the type of contracts with purchasers each mine has already concluded at present, a min-per-mine analysis is made. 

The present report differs from CPI's in that it approaches the issue from a macroeconomic perspective, %the opposite side of the scale at which the economy can be viewed. Conducting a macro analysis %of the same issue 
offering a complementary point of view on the conclusions obtained from the micro-analysis. To allow for maximum comparability between the results obtained from both studies, while the toolboxes used are considerably different, the scenarios analyzed with the toolboxes are as much as possible the same. 

Two scenarios are compared : a business as usual (BAU) scenario where current economic policy choices are assumed to hold for the period covered by the study (2108-2035), %it is not assumed that the world economy will make any specific effort to develop a low-carbon socio-technological system, 
and a 2 degrees (2DS) scenario where on the contrary choices are made resulting in the development of a low-carbon socio-technological system that allows to contain %The specific low-carbon scenario considered is one where the economic apparatus / adopts sociological and technological structure that allows 
global warming with a 50\% chance within the limit of 2°C with respect to pre-industrial times. Each of these scenarios implies distinct prospective export volume and/or price trajectories over the period under study for key sectors/commodities. The hypothesis for trajectories are obtained from CPI's models. Key export commodities are selected based on 1) whether export demand for them is different under BAU and 2DS and 2) whether they are important for the SA economy ; they are thermal coal, platinum and iron ore. A key imported commodity is crude oil, demand for which is satisfied by close to 100\% imports and of which the price on the international market might be lower under a 2DS versus under BAU. 

%This report proposes an analysis of the impacts of the adoption of a low-carbon  world-scale development of the economy on the South African economy from a macroeconomic point of view. This analysis complements the sector-, firm- and asset-level analysis conducted by CPI. The purpose of this complementary analysis is to evaluate whether the conclusions reached via two distinct methods are qualitatively similar. The two methods can in short be juxtaposed as follows : both on the micro and the macro level, the impact of changes to external conditions (export prices, export demand volumes) to which the SA economy is impacted is simulated/evaluated. For the goods exported by SA and identified as undergoing a significant change in demand in case of a low carbon trajectory followed by the world economy, CPI used scenarios for volume and price on the global markets under a 2DS and a BAU scenario. In the case of the micro analysis, the impact of these scenarios is evaluated on the economy on a sector, firm, and/or asset based level. The impacts on the individual units are then aggregated to obtain a figure representing the impact on the whole economy. The macro analysis follows the opposite route : the impact on the South African economy of the alternative demand scenarios for the main SA export goods that would be impacted by a world-scale low-carbon transition is evaluated on the level of the economy as a whole, from the macro economic point of view. From this economy-wide point of view, an effort is then made to pinpoint the observed effects on the macro-level to specific economic units (subsectors, firms). The two approaches are in a certain sense situated on the opposite side of the spectrum of how the economy can be described. 

%The impact on the South African economy is assessed if the world adopts a transformation of the economy compatible with a xx\% chance of keeping the raise of the average global temperature under 2°C. This impact is explored in detail by CPI taking a microeconomic approach. CPI identified the key sectors that would be impacted by a global 2DS scenario and analyzed for each of the main actors/firms making up those sectors the impact an external 2DS would have on them. 

%The development of a low-carbon economy on a worldwide scale, has implications for the quantities of fossil fuels used and, the demand for fossil fuels, specifically for coal, and for the demand for certain metals, etc. Thus, the nations that are providers, through export, of the primary commodities of which demand would be impacted by a low-carbon world-economy, will see there revenues impacted. 

\section{Method}

The specific issues under scrutiny in this study are the ones identified by the CPI study. Under a 2DS South Africa is expected to export less coal than under BAU. Secondly, the market price for crude oil, imported by South Africa, is expected to be lower under 2DS. In addition, the CPI analysis showed that lower revenues from exported coal might result in a higher domestic price. %for coal, which is used for electricity generation. 
%The impact of this second order effect will be looked at from the point of view of the macro analysis. 

%when the coal sector faces different demand for export purposes under both scenarios, the differential impact on the economy is not limited to the coal sector but extends to the sectors that sell intermediate inputs to the coal sector. The macroeconomic picture provided by the IOT allows to systematically evaluate the impact of the external scenarios throughout the economy, taking into account how each sector depends on others for its revenue.

The differential impact of export and import volumes and prices in BAU versus 2DS is not limited to %impact the growth and profitability  
the domestic sector doing the exporting and importing. The effect extends to the rest of the economy because commodities are made in value chains, linking industries through procurement %and supply 
relationships. The first will be referred to as the direct and the latter as the indirect effects. The indirect effects arise from two different mechanisms. The first is how the volume of output of a sector, of which goods for export are a part, determines the volume of inputs bought from others. %throughout the economy is via the inputs it uses, which contribute to total demand addressed to other sectors. 
If output doubles,  %, which includes exports, consumption by households and government and investment, 
the demand for inputs addressed to other sectors will react with the same order of magnitude. This effect can be substantial : in terms of employment, the number of jobs created indirectly, throughout the economy via demand for inputs, can be higher than the number of jobs created in the sector using the inputs. The second mechanism consists of how the price of the output of a sector, used as input by others, has repercussions on the profitability of other sectors and the price of the corresponding commodities. 

%The macroeconomic picture provided by the IOT allows to systematically evaluate the impact of the external scenarios throughout the economy, taking into account how each sector depends on others for its revenue.


The methodology here used for the analysis allows to take into account both direct and indirect consequences on the macro-level. The information needed to perform such an analysis is provided by input output tables (IOT). %These tables provide quantitative information on what inputs each sector uses from the other sectors and on how the output from each of the sectors is used.  It is based on the input-output table (IOT) of the South African economy. 
An IOT is a quantitative representation of the economy as a whole showing the inter-dependencies between sectors formed by the procurement relationships. %These inter-dependencies can be described in two ways, both based on the same fact : the goods produced by every industrial sector are used on the one hand for final uses, such as electrical appliances bought by households, and on the other hand used as inputs by other sectors. Examples of the latter are  This fact is described in the IOT firstly 1) by showing 
The IOT shows how the output of an industrial sector is used throughout the economy : final goods for export, investment, for consumption by households, government, as inputs for a number of other sectors and how much is exported. Inputs can be primary commodities such as coal or metal ores, processed goods such as steel or parts that can be assembled into machinery, etc. Consequently, the IOT also shows what inputs every sector uses produced by the same and other sectors. This latter fact can also be described by saying that the IOT shows how one sectors activity creates demand, and how much of it, to other sectors. Indeed, making cars out of steel generates economic activity and jobs in the steel sector. 

%The accounting equations represented in 
The IOT, in conjunction with sectoral employment data, can be used to estimate the number of jobs directly and indirectly generated by a given sector's activity. The table can then be used as the basis for creating a straightforward model to make estimates of how future developments in a sector's activity impact both GDP and the direct and indirect jobs the sector's activity generates.

%The method here used provides a way to make an estimate, when demand to a sector changes --- for example following the above scenario, due to decreased exports or lower export prices --- of what the impact on the whole economy  will be. Indeed, not only the sector that directly faces the changed conditions is impacted, but also the sectors from which this first sector uses inputs. In their turn, this latter set of sectors use inputs, and thus the changed demand they face impacts the sectors from whom they use inputs, and so forth. The initially changed export demand thus generates a change in the volume of production, and also in the price they face and their profitability, of a whole number of industrial sectors. The impact on employment will thus not be limited to the industrial sector directly concerned, but will spread to a certain extent throughout the economy. For every scenario, both the amount of jobs directly and indirectly generated by an industry's productive activity are estimated.

Also the CPI study looks into the impact on the value chain in which the identified key sectors are situated. For example, the demand for coal does not only impact the corresponding part of the mining sector, but also the actor, Transnet, providing the transport by rail of the coal to power ports, from where the coal is exported. However, the advantage of the present study is that it evaluates these indirect impacts in a systematic manner.

The methodology can be summarized as follows. The same price and volume trajectories for key export goods %impacted by a 2DS, 
evaluated on a micro-level by CPI, are evaluated on the macro level with the IO-model. The IO-model yields the difference in the impact of these export scenarios at the level of the economy as a whole between the BAU and 2DS scenarios. A distinction is made between the direct and the indirect impact. %, the former due to different final (export) demand and the latter being the impact due to the consequently different intermediate demands. 
These impact will be estimated both in terms of loss or gain in GDP or growth, and in terms of loss of jobs. Indeed, a key aspect of evaluating the impact of export scenarios on the economy is not only to evaluate the lost revenue, but also to evaluate which economic actors bear this cost. CPI makes a detailed analysis of how the costs are born between stakeholders (firms, the government), ... from a financial point of view and on jobs via ... In addition, the IO analysis allows for an evaluation of the impact on employment of the lost revenue. 


%The impact on employment is assessed of the external threats, as identified by CPI, facing South Africa under a 2DS, from a macroeconomic standpoint. 
%The effect on employment (and global cost to the economy) assuming the baseline scenario, which is IEA's New Policies Scenario (NPS), is compared with CPI's 2DS scenario. The 2DS is an adaptation of the IEA's SDS. 


\subsection{The input-output table}

An input-output table is a quantitative representation of the economy as a whole showing the inter-dependencies between sectors formed by their mutual procurement relationships. %These inter-dependencies can be described in two ways, both based on the same fact : the goods produced by every industrial sector are used on the one hand for final uses, such as electrical appliances bought by households, and on the other hand used as inputs by other sectors. Examples of the latter are  This fact is described in the IOT firstly 1) by showing 
The precise structure of an input-output table varies, %along with in line ith 
depending on the level of detail of the data collected. The more detailed the data, the heavier the data requirements. %The IOT is more detailed when the economy is disaggregated into more sectors, 
Figure \ref{IOT_basic} represents the structure of the IOT as supplied by the South African statistical agency (Stat SA). The first $n$ rows of the table show the $n$ industrial sectors into which the producing economy is disaggregated in their capacity as producers. The column the most to the right shows the total production or output $x_i$ for each of them. The columns represent the different agents, or institutional units, of the economy in their capacity as buyers. In each column is specified what amount they buy of each of the goods in the $n$ rows. The first $n$ columns are the $n$ industrial sectors, now in their role as buyers. The industries buy goods to use as intermediate inputs in their production process. Thus, this first $n \times n$ block in the IOT shows the industrial sectors in their buyer-seller relationships. The next columns, grouped under "Final demand" show what amount of each type of product is bought by the different agents as final product. These goods and services are bought as is, not as an input for the production of another product. The five types of final expenditure, are Consumption by households $c_i$ and government (Government expenditure) $g_i$, Exports $e_i$, Gross fixed capital formation (GFCF) $i_i$ and Changes in inventories $inv_i$. These first $n+5$ columns constitute total demand, for intermediate and final goods. The last column is "Imports" and specifies what amount $m_i$ of each type of good is imported. Imports are the second source of products in addition to domestic production. Demand is thus supplied to through domestic production and imports. By attaching a minus-sign to imports, the first $n+6$ columns add row-wise up to domestic output. Thus, the rows shows how the total output of each sector is used in the economy, or alternatively, where the demand for its good comes from.

\begin{table}[!b]
	\centering
	\begin{adjustbox}{width=\textwidth}
		\renewcommand*{\arraystretch}{1.15}
		%\footnotesize
		\small
		\begin{tabular}{cr|ccccc|b{30pt}b{30pt}p{30pt}b{30pt}b{30pt}|c|c}
			%\toprule
			%& \STAB{\rotatebox[origin=l]{90}{Sector $1$}} & &\STAB{\rotatebox[origin=r]{90}{Sector $j$}} & & \STAB{\rotatebox[origin=l]{90}{Sector $n$}} & \STAB{\rotatebox[origin=r]{90}{Consumption}} & \STAB{\rotatebox[origin=l]{90}{Government expenditures}} & \STAB{\rotatebox[origin=r]{90}{Exports}} & \STAB{\rotatebox[origin=l]{90}{GFCF}} & \STAB{\rotatebox[origin=r]{90}{Changes in inventories}} &  \STAB{\rotatebox[origin=l]{90}{Imports}} & \STAB{\rotatebox[origin=r]{90}{Output}}  \\ 		
			\multicolumn{2}{c}{\ }		& \multicolumn{5}{c}{Using/ buying sectors} & \multicolumn{5}{c}{Final demand}  &  \multicolumn{2}{c}{\ }	   \\
			%\cmidrule{3-7}
			
			&		& Sector $1$ & \ldots& Sector $j$ &\ldots & Sector $n$ & Cons. & Gov. exp. & Exports & GFCF & Ch. inv. & Imports & Output  \\
			\hline
			\multirow{5}{15pt}{\STAB{\rotatebox[origin=c]{90}{Supplying/}}\STAB{\rotatebox[origin=c]{90}{selling sectors}}}& Sector $1$& $z_{11}$ & \ldots& $z_{1j}$ & \ldots & $z_{1n}$ & $c_1$ & $g_1$ & $e_1$ & $i_i$ & $inv_i$ & $-m_i$ & $x_i$  \\ 
			&\STAB{\rotatebox[origin=l]{90}{\ldots}}\ \ \ \ \ \ &  \STAB{\rotatebox[origin=l]{90}{\ldots}}&  & \STAB{\rotatebox[origin=l]{90}{\ldots}}&  &\STAB{\rotatebox[origin=l]{90}{\ldots}}&  &  & \STAB{\rotatebox[origin=l]{90}{\ldots}} &  &  & \STAB{\rotatebox[origin=l]{90}{\ldots}} &  \STAB{\rotatebox[origin=l]{90}{\ldots}} \\ 
			&Sector $i$& $z_{i1}$ & \ldots & $z_{ij}$ &\ldots & $z_{in}$ & $c_i$ & $g_i$ & $e_i$ & $i_i$ & $inv_i$ & $-m_i$ & $x_i$  \\ 
			&\STAB{\rotatebox[origin=l]{90}{\ldots}} \ \ \ \ \ \ &  \STAB{\rotatebox[origin=l]{90}{\ldots}}&  & \STAB{\rotatebox[origin=l]{90}{\ldots}}&  &\STAB{\rotatebox[origin=l]{90}{\ldots}}&  &  & \STAB{\rotatebox[origin=l]{90}{\ldots}} &  &  & \STAB{\rotatebox[origin=l]{90}{\ldots}} &  \STAB{\rotatebox[origin=l]{90}{\ldots}} \\ 
			&Sector $n$& $z_{n1}$ &\ldots & $z_{nj}$ & \ldots& $z_{nn}$ & $c_n$ & $g_n$ & $e_n$ & $i_n$ & $inv_n$ & $-m_n$ & $x_n$  \\ 
			\hline
			\multicolumn{2}{r|}{Total expenditures (Int./final)} & $z_1$ & & $z_j$ & & $z_n$ & $c$ & $g$ & $e$ & $i$ & $inv$ & $-m$ & $x$  \\  
			\hline
			\multicolumn{2}{r|}{Net taxes on products}& $t_1$ &\ldots & $t_j$ & \ldots& $t_n$ &  \multicolumn{7}{c}{\ }     \\ 
			\cline{1-7} 
			\multirow{3}{15pt}{\STAB{\rotatebox[origin=c]{90}{Value}}\STAB{\rotatebox[origin=c]{90}{added}}}&Wages& $w_1$ &\ldots & $w_j$ &\ldots & $w_n$ &  \multicolumn{7}{c}{\ }  \\ 
			&Other taxes& $ot_1$ &\ldots & $ot_j$ & \ldots& $ot_n$ &   \multicolumn{7}{c}{\ }    \\ 		
			&Gross op. surplus& $gs_1$ & \ldots& $gs_j$ &\ldots & $gs_n$ &  \multicolumn{7}{c}{\ }     \\
			\cline{1-7} 		
			&Output& $x_1$ & & $x_j$ & & $x_n$ &  \multicolumn{7}{c}{\ }  \\ 		
		\end{tabular}
	\end{adjustbox}
	
	\caption{\label{IOT_basic}Structure of a single-country Input-Output table. The IOT supplied by the South African Statistical Agency has this form.	 }
\end{table} 


Column-wise, we can see what different goods each institutional unit is buying. The purchases by the $n$ industries reveal their production structure. For example, sector ... buys a lot of ... etc. 

A downside of the IOT in figure \ref{IOT_basic} is that it does not provide separate information on how domestic production and imports are used. Both are added to form a pool of total resources per type $x_i + m_i$, of which is the shown how much of the resources are allocated to what use ($z_{ij}$,$c_i$,$g_i$,$e_i$,$i_i$, $inv_i$). However, per type of use is not specified how much stems from domestic production and how much from importation. This is important, because the IOT will serve to create a model that will, among other things, respond to the question of how exports, which is final demand for the country in question, generate economic activity throughout the economy, via the procurement of inputs. If these inputs are imported, their production will not generate domestic jobs. It is when the inputs are domestically produced, that demand for them generates jobs and domestic added value. 

%For the intended analysis, the distinction between the uses made of imported versus domestically produced goods and services is important. Indeed, consumption of domestic output creates economic activity, and thus jobs and value, within the domestic economy. In contrast, when for example an industry uses imported goods as intermediate inputs for its activity, this does not give rise to domestic but to foreign jobs. Is is only when domestically produced goods are used as intermediate inputs that jobs are generated and value available domestically is created, indirectly, through the interlocking dependencies between industrial sectors.

Figure \ref{IOT_importsUses} represents an IOT that integrates this information. Each of the uses \{$z_{ij}$, $c_i$, $g_i$, $e_i$, $i_i$,  $inv_i$\} is now split up in a domestic and an imported part, indicated with respective superfixes $^d$ and $^m$. For example, the industries' intermediate inputs $z_{ij}$ of table \ref{IOT_basic} ar split in two parts $z_{ij}^d$ and $z_{ij}^m$. The same equivalent relationship holds for all the other uses. We thus get : 
\begin{align}
z_{ij} &= z_{ij}^d + z_{ij}^m. \label{usesDecompMX} \\
c_i &= c_i^d + c_i^m \nonumber \\
&\ldots \nonumber
\end{align}  Note that the relationship is also assumed for exports $e_i$. Indeed, some of imported goods are re-exported. 	

On the basis of an IOT table, a model can be built making one specific assumption. This assumption is inspired by interpreting the first $n$ elements in the columns of each industry as a production technology or a production function. Specifically, it is assumed that the amount of inputs per unit of output is constant. Additionally, we will also assume the share of value added in output, and the labour, capital and tax share in value added to be constant and the same as in the reference IOT.

\begin{table}[!t]
	\centering
	\begin{adjustbox}{width=\textwidth}
		\renewcommand*{\arraystretch}{1.15}
		%\footnotesize
		\small
		\begin{tabular}{cr|ccccc|b{30pt}b{30pt}p{30pt}b{30pt}b{30pt}|c|c}
			%\toprule
			%& \STAB{\rotatebox[origin=l]{90}{Sector $1$}} & &\STAB{\rotatebox[origin=r]{90}{Sector $j$}} & & \STAB{\rotatebox[origin=l]{90}{Sector $n$}} & \STAB{\rotatebox[origin=r]{90}{Consumption}} & \STAB{\rotatebox[origin=l]{90}{Government expenditures}} & \STAB{\rotatebox[origin=r]{90}{Exports}} & \STAB{\rotatebox[origin=l]{90}{GFCF}} & \STAB{\rotatebox[origin=r]{90}{Changes in inventories}} &  \STAB{\rotatebox[origin=l]{90}{Imports}} & \STAB{\rotatebox[origin=r]{90}{Output}}  \\ 		
			\multicolumn{2}{c}{\ }		& \multicolumn{5}{c}{Using/ buying sectors} & \multicolumn{5}{c}{Final demand}  &  \multicolumn{2}{c}{\ }	   \\
			%\cmidrule{3-7}
			
			&		& Sector $1$ & \ldots& Sector $j$ &\ldots & Sector $n$ & Cons. & Gov. exp. & Exports & GFCF & Ch. inv. & Imports & Output  \\
			\hline
			\multirow{5}{25pt}{\STAB{\rotatebox[origin=c]{90}{Supplying/}}\STAB{\rotatebox[origin=c]{90}{selling sectors}}\STAB{\rotatebox[origin=c]{90}{(Domestic)}}}& Sector $1$& $z^d_{11}$ & \ldots& $z^d_{1j}$ & \ldots & $z^d_{1n}$ & $c^d_1$ & $g^d_1$ & $e^d_1$ & $i^d_i$ & $inv^d_i$ &  & $x_i$  \\ 
						&\STAB{\rotatebox[origin=l]{90}{\ldots}}\ \ \ \ \ \ &  \STAB{\rotatebox[origin=l]{90}{\ldots}}&  & \STAB{\rotatebox[origin=l]{90}{\ldots}}&  &\STAB{\rotatebox[origin=l]{90}{\ldots}}&  &  & \STAB{\rotatebox[origin=l]{90}{\ldots}} &  &  & &  \STAB{\rotatebox[origin=l]{90}{\ldots}} \\ 
			&Sector $i$& $z^d_{i1}$ & \ldots & $z^d_{ij}$ &\ldots & $z^d_{in}$ & $c^d_i$ & $g^d_i$ & $e^d_i$ & $i^d_i$ & $inv^d_i$ &  & $x_i$  \\ 
			&\STAB{\rotatebox[origin=l]{90}{\ldots}} \ \ \ \ \ \ &  \STAB{\rotatebox[origin=l]{90}{\ldots}}&  & \STAB{\rotatebox[origin=l]{90}{\ldots}}&  &\STAB{\rotatebox[origin=l]{90}{\ldots}}&  &  & \STAB{\rotatebox[origin=l]{90}{\ldots}} &  &  &  &  \STAB{\rotatebox[origin=l]{90}{\ldots}} \\ 
			&Sector $n$& $z^d_{n1}$ &\ldots & $z^d_{nj}$ & \ldots& $z^d_{nn}$ & $c^d_n$ & $g^d_n$ & $e^d_n$ & $i^d_n$ & $inv^d_n$ &  & $x_n$  \\ 
			\hline
			\multirow{5}{25pt}{\STAB{\rotatebox[origin=c]{90}{Supplying/}}\STAB{\rotatebox[origin=c]{90}{selling sectors}}\STAB{\rotatebox[origin=c]{90}{(Rest of the world)}}}& Sector $1$& $z^m_{11}$ & \ldots& $z^m_{1j}$ & \ldots & $z^m_{1n}$ & $c^m_1$ & $g^m_1$ & $e^m_1$ & $i^m_i$ & $inv^m_i$ & $-m_i$ &  \\ 
			&\STAB{\rotatebox[origin=l]{90}{\ldots}}\ \ \ \ \ \ &  \STAB{\rotatebox[origin=l]{90}{\ldots}}&  & \STAB{\rotatebox[origin=l]{90}{\ldots}}&  &\STAB{\rotatebox[origin=l]{90}{\ldots}}&  &  & \STAB{\rotatebox[origin=l]{90}{\ldots}} &  &  & \STAB{\rotatebox[origin=l]{90}{\ldots}} &   \\ 
			&Sector $i$& $z^m_{i1}$ & \ldots & $z^m_{ij}$ &\ldots & $z^m_{in}$ & $c^m_i$ & $g^m_i$ & $e^m_i$ & $i^m_i$ & $inv^m_i$ & $-m_i$ &   \\ 
			&\STAB{\rotatebox[origin=l]{90}{\ldots}} \ \ \ \ \ \ &  \STAB{\rotatebox[origin=l]{90}{\ldots}}&  & \STAB{\rotatebox[origin=l]{90}{\ldots}}&  &\STAB{\rotatebox[origin=l]{90}{\ldots}}&  &  & \STAB{\rotatebox[origin=l]{90}{\ldots}} &  &  & \STAB{\rotatebox[origin=l]{90}{\ldots}} &  \\ 
			&Sector $n$& $z^m_{n1}$ &\ldots & $z^m_{nj}$ & \ldots& $z^m_{nn}$ & $c^m_n$ & $g^m_n$ & $e^m_n$ & $i^m_n$ & $inv^m_n$ & $-m_n$ &   \\ 
			\hline
		\multicolumn{2}{r|}{Total expenditures (Int./final)} & $z_1$ & & $z_j$ & & $z_n$ & $c$ & $g$ & $e$ & $i$ & $inv$ & $-m$ & $x$  \\  
		\hline
		\multicolumn{2}{r|}{Net taxes on products}& $t_1$ &\ldots & $t_j$ & \ldots& $t_n$ &  \multicolumn{7}{c}{\ }     \\ 
		\cline{1-7} 
		\multirow{3}{15pt}{\STAB{\rotatebox[origin=c]{90}{Value}}\STAB{\rotatebox[origin=c]{90}{added}}}&Wages& $w_1$ &\ldots & $w_j$ &\ldots & $w_n$ &  \multicolumn{7}{c}{\ }  \\ 
		&Other taxes& $ot_1$ &\ldots & $ot_j$ & \ldots& $ot_n$ &   \multicolumn{7}{c}{\ }    \\ 		
		&Gross op. surplus& $gs_1$ & \ldots& $gs_j$ &\ldots & $gs_n$ &  \multicolumn{7}{c}{\ }     \\
		\cline{1-7} 		
		&Output& $x_1$ & & $x_j$ & & $x_n$ &  \multicolumn{7}{c}{\ }  \\ 		
			
		\end{tabular}
	\end{adjustbox}
	
	\caption{\label{IOT_importsUses}Structure of a single-country Input-Output table where the uses of imports are specified.}
\end{table} 


%\clearpage

\subsection{Input-output demand-pull model}

A much used application of the IOT is the demand-pull model. The demand-pull model allows to calculate for a given scenario of final demand, how much output each of the sectors needs to provide, and how much goods need to be imported, to satisfy that demand. %The model calculates the total production requirements to satisfy demand. 
The total output needed is made up of the sum of the products used in different stages in the production process. The first component of output is of course final demand itself. To that, one has to add the inputs needed to produce these final goods and services. These inputs are called the direct requirements. In addition, for the production of these inputs, an additional set of inputs are needed, the production of which requires in their turn again more inputs, etc. The sum of all these latter input requirements are called the indirect requirements.

In the context of this paper, the comparison of different export scenarios, corresponding to BAU and 2Dg assumptions, will translate into different sets of total output requirements to the $n$ industrial sectors. These different projected production levels, or revenue streams, for each of the sectors translates into different levels of employment, taxes collected and capital income. The latter will affect the valuation of individual companies that make up the most impacted industries. 

In addition, because the model constructed here is based on the IOT in figure \ref{IOT_importsUses} that specifies the use of imports, what share of revenue stemming from the sale of final products is spent on imports.

The crucial assumption that is made when constructing and applying the demand-pull IO-model is that the inputs needed to produce one unit of output are the same at all stages of the production process, where with stages is meant the production of the final products, the direct and the indirect requirements.

%Denoting %$a_{ij}=z_{ij}/y_i$ the amount of inputs from sector $j$, sector $i$ uses to produce 1 unit of output, 
%$a_{ij}^d=z_{ij}^d/x_i$ the amount of domestically produced inputs of good $j$ used by sector $i$ to produce one unit of output, %and $a_{ij}^m=z_{ij}^m/x_i$ the amount of imported inputs of good $j$ used per unit of output from sector $i$, we get :


Thus, the demand-pull model is based on the central tenet that for each sector $j$, the ratio of each type of input $z_{ij}^k$, $i$ $\in$ \{1,\ldots, n\}, $k$ $\in$ \{d, m\} to the sector's output $x_j$ is constant. This is to be thought of as a recipe : if making a loaf of bread demands $400g$ of flour, then to make two breads one needs $800g$ of flour. We will denote these ratios $a_{ij}^k$. They represent the output needed of a certain type to make one unit of output : 
\begin{equation}
a_{ij}^k = z_{ij}^k/x_j. \label{sharesIntermediate}
\end{equation}
Secondly, we make the hypothesis of constant shares of domestically produced final goods for each type of final demand to the %, consumption, government expenditure, exports, capital formation, of each type of good $i$, to the 
total (imported plus domestically produced) final demand of that type, is constant. For example for consumption, we suppose the constant share $s^d_{ci}$ (eqn. \ref{sharesFinal}). Similar hypothesis are made for government expenditure, exports and capital formation of all the $n$ types of goods.
\begin{align}
s^d_{ci}&=c_i^d/c_i, \ \ \  i=1:n \label{sharesFinal}\\
s^d_{gi}&=g_i^d/g_i, \ \ \  i=1:n \nonumber\\
&\ldots \nonumber
\end{align} 

The combination of the hypothesis embodied in equations \ref{sharesIntermediate} and \ref{sharesFinal} and the accounting equations embodied in the rows of the IOT (figure \ref{IOT_importsUses}) yield the model. The accounting equations hold that (a) for every sector $i$ ($i \in \{1,\ldots,n\}$), its total output $x_i$ is given by the sum of total intermediate inputs plus final demand minus imports of that good, (b) resources stemming from importation are used for final and intermediate consumption and (c) the same holds for domestic production. Using eqns.\ref{usesDecompMX} and \ref{sharesIntermediate}, we get : %: $x_i = \sum_j z_{ij} + c_i + g_i + i_i + inv_i + e_i - m_i$, where $z_{ij}$ stands for the intermediate inputs used by sector $j$ and provided by sector $i$, $c_i$ and $g_i$ stand for consumption by household and government, $i_i$ is investment, $e_i$ exports and $m_i$ is imports. This equation can be rewritten as equation \ref{IOT} to show, on the left-hand side, how the total amount of goods available to be used by the domestic economy is obtained, and on the right-hand side how the goods are used :  
\begin{align}
x_i + m_i &= \sum_j a_{ij} \, x_j + c_i + g_i + i_i + inv_i + e_i \label{useAll} \\
m_i &= \sum_j a_{ij}^m \, x_j+ c_i^m + g_i^m + i_i^m + inv_i^m + e_i^m \label{useImports} \\
x_i &= \sum_j a_{ij}^d \, x_j + c_i^d + g_i^d + i_i^d + inv_i^d + e_i^d \label{useOutput} 
%x_i + m_i &= \sum_j (z_{ij}^d + z_{ij}^m) + c^d_i + g_i^d + i_i^d + inv_i^d + e_i^d +  c_i^m + g_i^m + i_i^m + inv_i^m + e_i^m \label{allUSes}
\end{align}

%The goods obtained through domestic production and imports are consumed domestically, for intermediary ($z_{ij}$) or final  purposes ($c_i$, $g_i$, $i_i$, $inv_i$) and the remaining share is exported and a source of revenue. However, the equation does not provide information on what share of consumption by households of good $i$ stems from domestic production $c_i^d$ versus from imports $c_i^m = c_i - c_i^d$. The same is true for the other uses $u_i$, $u_i \in \{z_{ij}, c_i, g_i, i_i, inv_i, e_i\}$ on the right-hand side of eqn. \ref{useAll}.

%(We suppose that the imported goods are not exported. This assumption is not neutral and thus not necessarily hold for all sectors, given the aggregation level of the SA IO table. For example, for the sector Motor vehicles ($I32$), a substantial share, $32\%$ of the value of the sector's inputs are from within the same sector. Additionally, $25\%$ of total output is exported and $76\%$ of total input is imported (+ imports are larger than total intermediate inputs). This suggests that there is both imports of finished vehicles, since import is larger than intermediate use in this category. Additionally, given the high value of imports, it suggests that there is import of parts which get assembled in SA.)

%To take into account that the demand for imported goods does not directly create value or give rise to jobs domestically (although the imported goods might be necessary for a particular production process), separate IOTs for domestic output and imported goods and services are needed\footnote{The OECD provides such tables at a level of detail of 35 sectors, the most recent one for the year 2011. The South African statistical agency provides a table with a level of detail of 50 sectors, the most recent for the year 2014, but that does not distinguish the use of imports.}. Each of these specifies accounting equations \ref{useOutput} and \ref{useImports} that permit to decompose eqn. \ref{IOT} in eqn. \ref{allUSes}:

We can now use the two sets of equations \ref{useOutput} and \ref{useImports} to calculate, for a given final demand scenario $y_i= c_i + g_i + i_i + inv_i + e_i$, the output required of all sectors and importation of all types of goods, assuming the $a_{ij}^k$, $i$ $\in$ \{1,\ldots, n\}, $k$ $\in$ \{d, m\} are known.
%To study the impact of changes in export or domestic price changes of specific goods and services on the domestic economy, only the set of equations \ref{useOutput} is needed.
The set of equations \ref{useOutput} yield domestic production which in turn allows the calculation of imports with the equations \ref{useImports}. 

For easier notation, we write these two sets of equations in matrix format. Output, imports and the five types of final demand for the $n$ types of goods are each grouped in $n \times 1$ column vectors, which are denoted in bold small letters. Capital letters indicate $n \times n$ matrices. The total, domestic and imports direct requirements matrices, $A$, $A^d$ and $A^m$, have as elements on row $i$ and column $j$ the technical coefficients $a_{ij}$, $a^d_{ij}$ and $a^m_{ij}$. Five matrices are introduced that contain the shares of domestic/imported final demand in total final demand (given by equation \ref{sharesFinal}) on the diagonal and zero's elsewhere  : $S^k_{c}$, $S^k_{g}$, $S^k_{e}$, $S^k_{i}$ and $S^k_{inv}$, $k$ $\in$ \{d, m\}. Of course, $S^d_{c} + S^m_{c} = \mathbb{I}$, where $\mathbb{I}$ is the $n \times n$ unitary matrix.

\begin{align}
\boldsymbol{x} &= A^d\,  \boldsymbol{x} + S^d_{c} \, \boldsymbol{c} + S^d_{g} \, \boldsymbol{g} + S^d_{i} \, \boldsymbol{i} + S^d_{inv} \, \boldsymbol{inv} + S^d_{e}\,  \boldsymbol{e} \label{useOutputsDom} \\
\boldsymbol{m} &= A^m \, \boldsymbol{x} + S^m_{c} \,\boldsymbol{c} + S^m_{g}\, \boldsymbol{g} + S^m_{i}\, \boldsymbol{i} + S^m_{inv}\, \boldsymbol{inv} + S^m_{e}\, \boldsymbol{e} \label{useOutputsImport} 
\end{align}

Rearranging equation \ref{useOutputsDom}, we get :
\begin{align}
\boldsymbol{y}^d &= S^d_{c} \, \boldsymbol{c} + S^d_{g} \, \boldsymbol{g} + S^d_{i} \, \boldsymbol{i} + S^d_{inv} \, \boldsymbol{inv} + S^d_{e} \, \boldsymbol{e} \nonumber \\
\boldsymbol{y}^m &= S^m_{c} \, \boldsymbol{c} + S^m_{g} \, \boldsymbol{g} + S^m_{i} \, \boldsymbol{i} + S^m_{inv} \, \boldsymbol{inv} + S^m_{e} \, \boldsymbol{e} \nonumber  \\
\boldsymbol{x} &= \big(\mathbb{I} - A^d \big)^{-1} \, \boldsymbol{y}^d \label{outputVector} \\
\boldsymbol{m} &= A^m \, \boldsymbol{x} + \boldsymbol{y}^m \label{importVector} 
%\boldsymbol{m} &= A^m  \boldsymbol{x} + \boldsymbol{c}^m + \boldsymbol{g}^m + \boldsymbol{i}^m + \boldsymbol{inv}^m + \boldsymbol{e}^m \label{useOutputsImport} 
\end{align}
Matrix $\big(\mathbb{I} - A^d \big)^{-1}$ is the domestic Leontief inverse which we note $B^d$. The elements $x_i$ of $\boldsymbol{x}$ are equal to a sum of $n$ terms : $x_i = \sum_j b_{ij}y_j$. Each of these terms are interpreted as follows : $b_{ij}y_j$ represents the output of sector $i$ stemming from the final demand addressed to sector $j$. The total $x_i$ is the total output needed from sector $i$ to satisfy the demand scenario $\boldsymbol{y}$. Because we are interested in each of these terms, the diagonal matrices $C$, $G$, $E$, $I$, $INV$ and $Y$ are introduced with the elements of the respective vectors (denoted by the same letter in lower case and bold) on the diagonal and zeros off-diagonal. Then, eqn. \ref{outputVector} becomes $X = \big(\mathbb{I} - A^d \big)^{-1} \, Y^d$, where $X$ has elements $x_{ij} = b_{ij}y_{jj}$. The interpretation of these elements is the same as for $b_{ij}y_j$ given above, since $y_j = y_{jj}$.

Similarly, we are interested in the terms that make up the vector $\boldsymbol{m}$ as given by eqn. \ref{importVector} or of the matrix $M$ given by $M = A^m \, X + Y^m$. Element $m_{ik}$ of $M$ is given by $m_{ik} = \sum_j a_{ij}x_{jk}$ which represents the the import of good $i$ by all sectors in response to demand of sector $k$. Element $m_i$ of $\boldsymbol{m}$ is then the import of good $i$ in response to demand from all sectors. However, for our analysis, we would like to know the import of all $n$ types of goods, done by each sector, in response to demand from sector $k$. Thus, we would like to know the sum $\sum_i a_{ij}x_{jk}$, which is the import expense of sector $j$ in response to demand from sector $k$. 

To this end we introduce the matrices $X^k$, which are diagonal matrices with the $k$th column of $X$ on the diagonal and zero elsewhere. The demand-pull IO-model, which allows to calculate the output and import by all the sectors needed to satisfy given final demand $\boldsymbol{y}$ (and if relevant its components, eg. consumption,\ldots), then becomes :
\begin{align}
Y^d &= S^d_{c} \, C + S^d_{g} \, G + S^d_{i} \, I + S^d_{inv} \, INV + S^d_{e} \, E \label{IOmodel_Yd} \\
Y^m &= S^m_{c} \,C + S^m_{g}\, G + S^m_{i}\, I + S^m_{inv}\, INV + S^m_{e}\, E 		\label{IOmodel_Ym} \\
X &= \big(\mathbb{I} - A^d \big)^{-1} \, Y^d  										\label{IOmodel_X} \\
%M^k &= A^m \, X^k + Y^m, \ \ \  k \in \{1,\ldots,n\} \label{Mk}
M^k &= A^m \, X^k, \ \ \  k \in \{1,\ldots,n\} 							\label{IOmodel_Mk}
%\boldsymbol{m} &= A^m  \boldsymbol{x} + \boldsymbol{c}^m + \boldsymbol{g}^m + \boldsymbol{i}^m + \boldsymbol{inv}^m + \boldsymbol{e}^m \label{useOutputsImport} 
\end{align}
In summary, the above model allows to calculate the output of all sectors and import by all sectors needed to satisfy a given final demand scenario. The matrices with parameters of the model are the direct requirement matrices $A^d$ and $A^m$, and the share matrices $S^d_{c}$, $S^d_{g}$, $S^d_{e}$, $S^d_{i}$ and $S^d_{inv}$. Final demand for domestic production is given by $Y^d$ which is zero off-diagonal ; its $i$th diagonal element is final demand for domestically produced goods addressed to sector $i$. Final demand for imported goods is given by $Y^m$, of which the $i$th diagonal element represents the final demand for imported goods of sector $i$ ; again, the off-diagonal elements are zero. The results of the model are to be found in the output matrix $X$ and, the $n$ import matrices $M^k$ and the matrix with imports of final goods $Y^m$. The elements $x_{ij}$ of $X$ give the output sector $j$ needs from sector $i$ to satisfy the final demand scenario. The elements $m^k_{ij}$ of $M^k$ represent the import of good $i$ done by sector $j$ to satisfy demand (in domestically produced goods) from sector $k$. Thus, the columns sums of $M^k$ are the imported goods, intermediate and final, of all types, sector $j$ needs to satisfy demand from sector $k$.


%In the absence of seperate IOTs for domestic production and imports, the fast solution is to suppose that for every type of use $u_i$, the share $s_{i}^d$ of domestic output within the total use is equal over all the types of uses made :
%\begin{align}
%s_i^d &= \frac{y_i}{y_i + m_i} \\
%u_i^d &= s_i^d \, u_i \label{shareM} \\
%u_i^m &= (1-s_i^d) \, u_i
%\end{align}
%where $u_i^d$ and $u_i^m$ stand for all the possible uses made of respectively domestic output and imported products : $u_i^m \in \{z_{ij}^m, c_i^m, g_i^m, i_i^m, inv_i^m, x_i^m\}$ and $u_i^d \in \{z_{ij}^d, c_i^d, g_i^d, i_i^d, inv_i^d, x_i^d\}$.

%Going to back to matrix notation and denoting $S^d$ the diagonal matrix of the shares $s_i^d$ and 0's off-diagonal, then :
%\begin{align}
%A^d &= S^d \, A \\
%C^d &= S^d \, C \\
%G^d &= S^d \, G \\
%I^d &= S^d \, I \\
%INV^d &= S^d \, INV \\
%E^d &= S^d \, E 
%\end{align}

%Combining the above with eqn. \ref{useOutputsDom} and denoting $\mathbb{I}$ the identity matrix, we obtain the demand-driven input-output model that allows to determine for a given scenario of final demand $D = (C, G, I, INV, X )$, the output per sector needed :
%\begin{equation}
%%Y  = S^d \, A \, Y + S^d \, (C + G + I + INV + X)  \label{IO1} \\
%X = (\mathbb{I} - S^d \, A)^{-1} \,  S^d \, (C + G + I + INV + E)  \label{IOdemand-driven}
%\end{equation}

\subsection{Decomposition of final demand-revenue} \label{Section_decomposition}

The revenue coming into the country from selling goods to other countries is distributed, following the logic of economic activity, to different participants. Part of the export revenue pays workers, part goes to the state via taxes, and the remaining part represents the remuneration of capital, which is in practice the money used to pay back debt, pay shareholders, etc. However, not all the revenue coming into the country via exports of say sector $i$ stays within the country. Part is used to import intermediate goods used in the production of the exported goods and thus leaves the country again. The part of the export revenue that is used by sector $i$ to buy intermediate goods for its production (of exported goods) is a cost to the sector but not a loss for the domestic economy : the cost to sector $i$ is revenue for supplying sectors, who after paying in turn their supplying sector, use the remainder, value added, to pay wages, taxes and remunerate capital. This suggests that the export revenue can be traced back into a part leaving the country, and a part distributed as value added --- wages, taxes and capital remuneration --- throughout the economy. 

First we show that the revenue coming in via final demand, in our case we are most interested in exports, can be decomposed entirely in value added created in the different sector of the economy. (To simplify, we assume away for now "Taxes on production (?)".)

\ldots

We thus see that the total value of final demand has spread throughout the economy creating value in different sectors. The matrix ... traces back that value to where the demand for the products generated. However, not all the value originates in the domestic economy. 

The cost of imported inputs represent export revenue for the rest of the world (RoW) and thus represent foreign value added, which is embedded in gross exports, part of final demand y. 

%There are two separate issues : one is that the demand for imported intermediate goods does not generate jobs within the economy, the second is how the demand generated for domestic goods, whether it be final or the indirectly generated demand for intermediates, canbe decomposed as value added over the whole economy. 

Once the output matrix $X$ has been determined, with elements $x_{ij}$ representing the output from sector $i$ resulting from final demand addressed to sector $j$, %detailing the output needed from all sectors in response to final demand addressed to all sectors, 
final demand revenue can be decomposed into imports, value added and net taxes on products. To show this, we first decompose every element $x_{ij}$ into domestically produced $z^{tot,d}_{ij}$ and imported intermediate inputs $z^{tot,m}_{ij}$, taxes on products $t_{ij}$ and the constituent parts of value added, wages $w_{ij}$, other taxes $ot_{ij}$ and gross operating surplus $gs_{ij}$ :   
\begin{equation}
x_{ij} = z^{tot,d}_{ij} + z^{tot,m}_{ij} + t_{ij} + w_{ij} + ot_{ij} + \mli{gs}_{ij} \label{decompositionOutput1} \\
\end{equation}
Each of the elements on the right hand side is determined based on the assumption 
%The assumption that underlies this decomposition is 
that for each type of product $i$, its production technology (represented by the $i$th column of $A$)%, denoted as $\boldsymbol{a}_j$)
, the shares per output of wages, taxes and gross operating surplus, are only dependent on the type $i$ and, importantly, independent of what use is made of the output of sector $i$ (as inputs by any of the sectors $j$ or as final demand). For example, for all sectors $i$, the share of wages per output of good $i$ is independent of the origin of demand for the good $i$ : $w_{ij}/x_{ij} = w_{ik}/x_{ik}$, $\forall j,k \in \{i, \dots n\}$. The analogous assumption is made for all the other variables on the right-hand side of equation \ref{decompositionOutput1}, %Otherwise put, the share of taxes, wages, each of the intermediate input costs (imported and domestically produced), etc. is, for every $i$ the same for all $x_{ij}$, $i$ $\in$ $\{1,\ldots, n\}$. Thus, the elements of $X$ can be decomposed as :
which allows to determine them as : 
\begin{align}
%x_{ij} &= z^{tot,d}_{ij} + z^{tot,m}_{ij} + t_{ij} + w_{ij} + ot_{ij} + gs_{ij} \label{decompositionOutput1} \\
z^{tot,d}_{ij} &= x_{ij} \, z^d_j/x_j \\
z^{tot,m}_{ij} &= x_{ij} \, z^m_j/x_j \label{totImports} \\
t_{ij} &= x_{ij} \, \, t_j/x_j \\
w_{ij} &= x_{ij} \, w_j/x_j \\
ot_{ij} &= x_{ij} \, ot_j/x_j \\
gs_{ij} &= x_{ij} \, gs_j/x_j
\end{align}
In the above equations, for example $z^{tot,d}_{ij}$ represents the total cost of domestically produced intermediate inputs that went into the production of $x_{ij}$. %The interpretation of the other variables on the right-hand side of equation \ref{decompositionOutput1} is similar. 

Summing equation \ref{decompositionOutput1} over $j$, we get :
\begin{equation}
x_i - \sum_j z^{tot,d}_{ij}  = \sum_j z^{tot,m}_{ij} + \sum_j \big(t_{ij} + w_{ij} + ot_{ij} + gs_{ij}\big) \label{decompositionOutput2}
\end{equation}
The left-hand side of the above equation is equal to final demand of good $i$, domestically produced $y^d_i$. The first term on the right-hand side is the importation of product $i$ to be used as intermediate inputs. %$m^{inp}_i$. 
Summing equation \ref{decompositionOutput2} over $i$ and reversing the order of the sums over $i$ and $j$ results in :
\begin{equation}
\sum_i y^d_i  = \sum_{i}\sum_{j} z^{tot,m}_{ij} + \sum_j \big( \sum_i t_{ij} + \sum_i (w_{ij} + ot_{ij} + gs_{ij})\big) \label{decompositionOutput3}
\end{equation}
The two terms within the second sum over $j$ on the right-hand side of the above equation, $\sum_i t_{ij}$ and $\sum_i (w_{ij} + ot_{ij} + gs_{ij})$ represent respectively net taxes on products of sector $j$ and value added created by the sector $j$, decomposed into wages, other taxes and gross operating surplus. Thus, final domestic demand decomposes into imports, net taxes on products and value added. Adding final demand satisfied through imports $y^m_i$ for products $i$ we can write :
\begin{equation}
\sum_i y_i =  \sum_i (\sum_{j} z^{tot,m}_{ij} + y^m_i) + \sum_j \big( \sum_i t_{ij} + \sum_i (w_{ij} + ot_{ij} + gs_{ij})\big) \label{decompositionOutput4}
\end{equation}
The above equation shows how the revenue gained from final demand decomposes into importation costs and into value added created throughout the economy. Finally, we note the link between the imports stemming from final demand to sector $k$ as decomposed in matrix $M^k$ (equation \ref{IOmodel_Mk}) and as given by the above decomposition of output (equation \ref{totImports}). The total cost of intermediate inputs imported by sector $j$ resulting from demand addressed to sector $k$ can be calculated in two equivalent ways : $z^{tot,m}_{jk} = \sum_i m_{ij}^k$. %how intermediate and final demand for domestically produced goods, and not for imported goods, create value added throughout the economy. Imports represent an outgoing flux for the domestic economy. (This comment is meant as an illustration of the functioning of the economy, not as an argument against importing goods.) 


\subsection{Employment content}

% First explain the ECT matrix. Say that every element is a mix of 2 pieces of information, 1 sector specific row-wise, the second sector-sepcific column-wise. Say that this information will be split, because of effect of averaging. Say that employment content is columns sum, the general form of the decomposition is ... thus summing terms will average out...

Given a final demand scenario $Y$ (or its constituent components \{$C$, $G$, $I$, $\mli{INV}$, $E$\}), equations \ref{IOmodel_Yd} - \ref{IOmodel_Mk} allow to calculate the necessary output and imports by each of the sectors to satisfy demand. In addition, the IO-model can also be used to estimate employment created throughout the economy by a given demand scenario. To capture both the jobs created by final demand $y_i$ %for good $i$ 
within the sector $i$ as well as the jobs created in other sectors, the concept of the employment content (ec) is used. The employment content $ec_i$ of sector $i$ is defined as the number of direct and indirect jobs generated by one unit of final demand addressed to %the production of one unit of output of 
sector $i$. To calculate the employment content of all sectors, define the $n \times n$ matrix $EC$ where $ec_{ij}$  is the number of jobs in sector $j$ created by demand addressed to sector $i$. Define $E$ the $n\times n$ matrix with zeros off-diagonal and with on the diagonal the number of people employed in sector $i$ per unit of the sector's output, $e_{ii}=ne_i/x_i$, where $ne_i$ is employment, in the year of the IOT, in sector $i$. Given the final demand matrices $Y^d$ and $Y^m$ as defined in \ref{IOmodel_Yd} and \ref{IOmodel_Ym}, the employment content matrix $EC$ is calculated as :
\begin{align}
EC	 &= E\,  \, B^d \, S^d \label{EC} \\ 
B^d &=  (\mathbb{I}-A^d)^{-1} \\
S^d	 &=  Y^d \, (Y^d+Y^m)^{-1}  
\end{align}
%Direct employment in sector $j$ is then defined as $ec_{jj}$ and indirect employment as $\sum_{i, i\neq j} ec_{ij}$. 
The total employment content $ec_j$ of sector $j$ is given by $\sum_i ec_{ij}$. The diagonal matrix $S^d$ (with zeros off-diagonal) captures that the more final goods are imported, the less domestic economic activity demand for those goods creates.
%The employment content is then defined as the number of jobs created through a single unit of final demand, thus by setting $C + G + I + INV + X=\mathbb{I}$, where $\mathbb{I}$ is the identity matrix. 
%\begin{equation}
%	EC	= E\, LI^d \, S^d \label{EC} \\ 
%\end{equation}
%where $S^d = \mathbb{I} - S^m$ contains the shares of domestic output in total supply of goods, i.e. supply through imports and through domestic production, available for domestic use and for exports. In the above equation, $E$ and $S^d$ are diagonal matrices with 0's off-diagonal.

The elements $ec_{ij}$ of EC are composed of three factors, 
\begin{equation}ec_{ij} = e_{ii} \, b^d_{ij} \, s^d_{jj}, \label{EC_decomp}\end{equation}
that capture different aspects of the (terms of the) employment content $ec_j$. The number of jobs created in sector $i$ due to a unit of final demand addressed to sector $j$ can be high because a) sector $i$ employs many people per unit of output ($e_{ii}$ is high), b) because production of final product $j$ demands many domestically produced direct and indirect inputs from sector $i$ ($b^d_{ij}$ is high), c) final good of type $j$ is primarily sourced through domestic production ($s^d_{jj}$ is high). To further understand the reasons behind the differences in the employment content between sectors, we apply the decomposition of the ec as proposed in \citep{perrier2017transition}. The approach further decomposes firstly the factor $e_{ii}$ to understand some reasons why employment per unit of output is different between sectors, and secondly $b^d_{ij}$ to separate out the effects of the input-intensity of final product $j$ and whether inputs are sourced domestically or imported.

The decomposition of $e_{ii}$ is based on the accounting equations that are present in the columns of the IOT (see figure \ref{IOT_importsUses}) :
\begin{equation}
x_i = z^d_i + z^m_i + t_i + ot_i + w_i + gs_i
\end{equation}
Since total employment $e_i$ is equal to compensation of employees $w_i$ divided by the average wage $\bar{w}_i$, we get :
\begin{equation}
 e_i = (x_i - gs_i - z^d_i - z^m_i - t_i - ot_i )/\bar{w}_i \label{empl_accounting}
\end{equation}
From the above equation can be deduced that employment is high when 1) wages are low, 2) taxes are low, 3) compensation of employees (equal to the nominator) is high, 4) imports of inputs are low and 5) inputs in general are low\footnote{However, high inputs means low employment in the sector in case, but high demand, and thus possibly large employment in sectors supplying to sector $i$.}. To capture this observation, employment per unit of output $e_{ii}$ is decomposed as :
\begin{align}
%E &= \frac{\mli{VA^d}}{Y} \, \frac{\mli{VA}^{HT}}{\mli{VA}^d} \, \frac{\mli{COMP}}{\mli{VA^{HT}}} \, \frac{\mli{NPE}}{\mli{COMP}}   \label{decompE} \\ 
e_{ii} = \frac{ne_i}{x_i}&= \frac{va_i}{x_i} \ \frac{va^{HT}_i}{va_i}  \ \frac{w_i}{va^{HT}_i}  \ \frac{ne_i}{w_i} \nonumber \\
&= \frac{va_i}{x_i} \ t_i \, ls_i \, n_i \label{ECdecomp_e} \\
t_i &= va^{HT}_i/va_i \nonumber  \\
ls_i &= w_i\, /va^{HT}_i \nonumber \\
n_i &= \mli{ne}_i/w_i \  (\, =1/\bar{w}_i)	  \nonumber \\
e_{ij, i\neq j} &= 0		  \nonumber 
\end{align}
%where $\mli{VA^d}$ is, $\mli{VA}^{HT}$, $\mli{COMP}$,$T$, $L$, $N$ and are all diagonal matrices with 0's off-diagonal.
In the above, $va_i$ and $va_i^{HT}$ denote respectively value added including all taxes, $va_i= w_i + gs_i + t_i + ot_i$, and excluding taxes $va^{HT}_i= w_i + gs_i$.

The factor $b^d_{ij}$ is decomposed by defining the ratios $mi_{ij}$ of the total requirements coefficients, domestically produced, to total requirements, imported and domestic :
\begin{align}
b^d_{ij} &= mi_{ij} \ b_{ij} \label{ECdecomp_bd} \\
mi_{ij} &= b^d_{ij} / b_{ij} \nonumber
\end{align}
 
Combining equations \ref{EC_decomp}, \ref{ECdecomp_e} and \ref{ECdecomp_bd}, the desired factorial decomposition is, almost, obtained :
\begin{equation}
ec_{ij} = \frac{va_i}{x_i} \ t_i \ ls_i \ n_i \ mi_{ij} \ b_{ij} \ s_{jj}
\end{equation}
The above equation contains two distinct parts. The last three factors % %The factors $va_i\ b_{ij}/x_i$ and $s_{jj}$ 
have an influence on employment through a scale effect : the more output is produced by sector $i$, the more people it will employ\footnote{Given that we assume that employment per unit of output is constant per sector.}. The three factors $mi_{ij} \ b_{ij} \ s_{jj}$ are primarily characteristics of sector $j$ : they capture whether final goods of type $j$ are imported or not, what types of inputs the sector uses --- i.e; inputs of what type of goods and imported or not. Indeed, inputs might be required that are not available on the domestic market. The first four factors have an influence on employment because of characteristics specific to the producing sector $i$. To finalize the decomposition, the factors representing value added created in sector $i$ because of demand addressed to sector $j$, if all intermediate goods were produced domestically, $\frac{va_i}{x_i} b_{ij}$, is used as a ventilation factor :
\begin{align}
ec_{ij} &= (v_{ij}^{1/4} \ t_i) \ (v_{ij}^{1/4}ls_i) \ (v_{ij}^{1/4}n_i) \ (v_{ij}^{1/4}mi_{ij})  \ s_{jj}  \\
ec_{ij} &=  \ t^v_{ij} \ ls^v_{ij} \ n^v_{ij} \ mi^v_{ij}  \ s_{jj} \label{ECdecomp_ventil} \\
v_{ij} &= b_{ij} \frac{va_i}{x_i} 
\end{align}
Next, to compare the employment content of each sector and their decompositions to the average ec and its decomposition, the average $\bar{ec}$ is defined by, not by averaging over the $ec_j$'s, but by defining means of the decomposition factors on the right-hand side of 
\ref{ECdecomp_ventil}. The definitions of the means are given in appendix \ref{ECmean}. The matrix of which the columns sums are the mean employment contents of the economy (all column sums are thus equal) $\bar{EC}$ has elements $\bar{ec}_{ij}$ defined as : 
\begin{align}
%ec_{ij} &= (v_{ij}^{1/4} \ t_i) \ (v_{ij}^{1/4}ls_i) \ (v_{ij}^{1/4}n_i) \ (v_{ij}^{1/4}mi_{ij})  \ s_{jj} \label{ECdecomp_ventil} \\
\bar{ec}_{ij} &=  \ \bar{t}^v_{ij} \ \bar{ls}^v_{ij} \ \bar{n}^v_{ij} \ \bar{mi}^v_{ij}  \ \bar{s}_{jj} \label{ECdecomp_ventil_mean} \\
%v_{ij} &= b_{ij} \frac{va_i}{x_i} 
\end{align}
The diagonal and the non-diagonal elements of $\bar{EC}$ are different, and all the diagonal elements are equal and all the non-diagonal elements are equal to one another. The diagonal elements represent the average number of jobs created by demand addressed to sector $j$ within the same sector. The non-diagonal elements represent the the average number of jobs in a sector by demand coming from a different sector. 

To transform the factorial decomposition in an additive decomposition which allows for easy visual apprehension of the data, the logarithmic mean divisia index is used as detailed in \citep{ang2005lmdi}. Taking the logarithm of the ratio of \ref{ECdecomp_ventil} and \ref{ECdecomp_ventil_mean} results in a sum of terms where each term is, for small deviations of the mean, the percent change of $ec_{ij}$ from the mean $\bar{ec}_{ij}$. The same interpretation holds for each of the decomposition factors :
\begin{align}
%ec_{ij} &= (v_{ij}^{1/4} \ t_i) \ (v_{ij}^{1/4}ls_i) \ (v_{ij}^{1/4}n_i) \ (v_{ij}^{1/4}mi_{ij})  \ s_{jj} \label{ECdecomp_ventil} \\
log\bigg(\frac{ec_{ij}}{\bar{ec}_{ij}}\bigg) &=  \ log\bigg(\frac{t^v_{ij}}{\bar{t}^v_{ij}}\bigg) +\ log\bigg(\frac{ls^v_{ij}}{\bar{ls}^v_{ij}}\bigg) +\ log\bigg(\frac{n^v_{ij}}{\bar{n}^v_{ij}} \bigg) +\ log\bigg(\frac{mi^v_{ij}}{\bar{mi}^v_{ij}}\bigg)  +\ log\bigg(\frac{s_{jj}}{\bar{s}_{jj}}\bigg)  %\label{ECdecomp_ventil_final}
%v_{ij} &= b_{ij} \frac{va_i}{x_i} 
\end{align}
Scaling the above additive composition leads to the final decomposition equation that sheds a light on why all $ec_{ij}$'s are different, by comparing them to the means $\bar{ec}_{ij, i\neq j}$ and $\bar{ec}_{jj}$. The decomposition of the deviation of the total employment content $ec_j$ to the economy's mean $\bar{ec}$ is then given by summing over $i$. 
\begin{align}
%ec_{ij} &= (v_{ij}^{1/4} \ t_i) \ (v_{ij}^{1/4}ls_i) \ (v_{ij}^{1/4}n_i) \ (v_{ij}^{1/4}mi_{ij})  \ s_{jj} \label{ECdecomp_ventil} \\
ec_j - \bar{ec} &= \sum_i \big( ec_{ij} - \bar{ec}_{ij} \big) \nonumber \\
&= \sum_i \frac{ec_{ij} - \bar{ec}_{ij}}{log(\frac{ec_{ij}}{\bar{ec}_{ij}})}\ \Bigg( log\bigg(\frac{t^v_{ij}}{\bar{t}^v_{ij}}\bigg) +\ log\bigg(\frac{ls^v_{ij}}{\bar{ls}^v_{ij}}\bigg) +\ log\bigg(\frac{n^v_{ij}}{\bar{n}^v_{ij}} \bigg) +\ log\bigg(\frac{mi^v_{ij}}{\bar{mi}^v_{ij}}\bigg)  +\ log\bigg(\frac{s_{jj}}{\bar{s}_{jj}}\bigg) \Bigg)  \label{ECdecomp_ventil_final}
%v_{ij} &= b_{ij} \frac{va_i}{x_i} 
\end{align}
To obtain detailed information on the type of employment created by final demand addressed to a specific sector $j$, one can both look at the decomposition of the difference between the total employment $ec_j$ and the mean $\bar{ec}$ as well as to the decomposition of the individual terms of equation \ref{ECdecomp_ventil_final}, i.e. before summing over $i$. There are several reasons for this. A first reason is that the total requirements coefficients on the diagonal $b_{jj}$ are typically much larger than off-diagonal $b_{ij, i\neq j}$. Thus, in the sum over $i$, the diagonal elements will tend to dominate the off-diagonal elements. In addition, the sum over $i$ off the off-diagonal elements will do exactly what an average does, it is average out. For example when, on average, the labour share of the jobs created throughout the economy by final demand for product $j$ is close to the mean, this can be due to the fact that this is true for all the sectors in which jobs are sustained, or it can be due to a situation where part of the jobs are in sectors with an above average labour share and part with a below average labour share. This effect will further contribute to the dominance of the diagonal over the off-diagonal elements. %Hence, summing provides a correction to the masks the type of jobs created by demand for product $j$ throughout the economy by the jobs sustained in sector $j$ itself.  
To characterize what types of jobs are created throughout the economy, when applying the above decomposition to the South African economy, we will provide separately the decomposition of firstly the jobs created within the sector to which final demand is addressed, $ec_{jj} - \bar{ec}_{jj}$ and secondly and thirdly of all the jobs in sectors with above and below average $ec_{ij}$, i.e. for $i\neq j$, $\sum_i (ec_{ij} - \bar{ec}_{ij})$, with $ec_{ij} - \bar{ec}_{ij} >0$ and with $ec_{ij} - \bar{ec}_{ij} <0$.

%To show these effects, the method proposes a factorial decomposition of the employment content of each sector. The employment content of sector $i$ is the sum over all sectors of the employment (number of jobs) that is created in each one by a unit of final demand addressed to sector $i$. However, the reasons for high/low employment as evoked by equation \ref{empl_accounting} are sector-specific. Thus, when we sum over the whole economy, in some sectors wages will be above the mean and in others below ; thus, summing will tend to average out deviations from the mean. Thus, when the wages component is close to average of the employment content, this may be because is some sectors wages are above and in others below average. Or, because indeed, demand from sector $i$ generated activity in sectors with mostly average wages.
%
%For this reason, we will rather decompose employment in each sector, as inspired by \label{empl_accounting}, and 
%
%
%decomposes the difference between each sectors total employment and the economy's mean EC content into a sum of five terms. Each of the terms sheds a light on whether the employment content is above or below average because 1) wages are low/high, the labour share is high/low, taxes are 
%The approach we take here is slightly different from 
%
%Employment per unit of output is decomposed as :
%\begin{align}
%	E &= \frac{\mli{VA^d}}{Y} \, \frac{\mli{VA}^{HT}}{\mli{VA}^d} \, \frac{\mli{COMP}}{\mli{VA^{HT}}} \, \frac{\mli{NPE}}{\mli{COMP}}   \label{decompE} \\ 
%	&= \frac{\mli{VA^d}}{Y} \, T \, L \, N \\
%	T &= \\
%	L &= \\
%	N &= 
%\end{align}
%where $\mli{VA^d}$ is, $\mli{VA}^{HT}$, $\mli{COMP}$,$T$, $L$, $N$ and are all diagonal matrices with 0's off-diagonal.
%
%The domestic leontief inverse is decomposed into the total leontief inverse and factors that reflect the import rates of intermediate inputs : 
%\begin{equation}
%	LI^d = \mli{LI} \odot \mli{MI} \label{decompLId}
%\end{equation}
%where $\odot$ designates the element wise Hadamard multiplication of matrices.
%
%Combining \ref{EC}, \ref{decompE} and \ref{decompLId}, brings us close to the final decomposition that will be used :
%
%\begin{equation}
%EC	= \frac{\mli{VA}^d}{Y} \, T \, L \, N \, \mli{LI} \odot \mli{MI} \, S^d,   \label{decompEC1} \\ 
%\end{equation}
%Again, note that the matrices with possibly non-zero elements off-diagonal are $EC$, $LI$ and $MI$. Next QQ sets :
%
%$$ V = \frac{\mli{VA}^d}{Y} \mli{LI}$$ 
%
%such that :
%\begin{equation}
%EC	= T \, L \, N \, V \odot \mli{MI} \, S^d,   \label{decompEC1} \\ 
%\end{equation}
%which can equivalently be written as :
%
%\begin{equation}
%EC	= \bar{T} \odot \bar{L} \odot \bar{N} \odot V \odot \mli{MI} \odot \bar{\bar{S^d}},   \label{decompEC2} \\ 
%\end{equation}
%
%where the operation denoted by the single bar $\bar{\ }$ is right multiplication with $\mathbb{J}$ the $n\times n$ matrix with all ones, $\bar{T} = T \, \mathbb{J}$, and the operation denoted by the double bar $\bar{\bar{\ }}$ is left multiplication with $\mathbb{J}$, $\bar{\bar{S^d}} = \mathbb{J} \, S^d$. 

\subsection{Cost-push model}

The input-output table can also be used to determine, when the price of an input $i$ changes with respect to a base scenario, what the impact would be on the prices of all the other sectors, assuming that the increased cost is fully transferred to buyers. This implies that the profitability of each sector is not impacted and that they still use the same quantities of all inputs. 

Given a reference IOT measured in value, then for the base scenario, prices $p_i$ for all sectors $i$ can be fixed at 1, such that that the IOT in quantities has the same numerical values as the one in value. For the base scenario, the vector of all prices is written as $\boldsymbol{p}$ (and is thus a vector of all 1's).

If $Z$ is the inter-industry matrix (in quantity) of size $n\times n$ with all sectors, then we denote $Z^{-i}$ the inter-industry matrix of size $n-1 \times n-1$ where the row of inputs from sector $i$ to other sectors and the columns of uses of $i$ are omitted. Denoting the $1 \times n-1$ row of inputs of sector $i$ to other sectors except for to itself as $(inp^{-i}_i)^T$, the accounting equations that follow from the first $n$ columns of the IOT then read :

\begin{equation}
(p^{-i})^T \, Y^{-i} = (p^{-i})^T Z^{-i} + (inp^{-i}_i)^T + (gva^{-i})^T + (t^{-i})^T \label{cost-push1}
%(y^{-i})^T =  Z^{-i} + (inp^{-i}_i)^T + (gva^{-i})^T + (t^{-i})^T
\end{equation}

where $Y^{-1}$ is the diagonal $n-1 \times n-1$ matrix of output, measured in quantities, and $(p^{-1})^T$, $(gva^{-1})^T$  and $(t^{-1})^T$ are the $1 \times n-1$ row-vectors of respectively prices, gross value added a net taxes on products. Right-multiplying equation \ref{cost-push1} with the diagonal matrix with the inverse of total output measured in quantities on the diagonal and 0's elsewhere, $\underline{Y}^{-i}$,we get :
\begin{equation}
(p^{-i})^T  = (p^{-i})^T A^{-i} + \Big((inp^{-i}_i)^T + (gva^{-i})^T + (t^{-i})^T \Big)\, \underline{Y}^{-i} \label{cost-push2}
\end{equation}
and equivalently :
\begin{equation}
(p^{-i})^T  =  \Big((inp^{-i}_i)^T + (gva^{-i})^T + (t^{-i})^T \Big)\, \underline{Y}^{-i} \, (I - A^{-i})^{-1} \label{cost-push3}
\end{equation}
where the vector $p^{-i}$ contains all 1's.

Consider an alternative scenario $A$ where the price of input $i$ changes by $\lambda\%$ for all sectors with respect to the base scenario, then the value of inputs $inp^{A^{-i}}_{i} = (1+\lambda/100)\, inp^{-i}_i$. The resulting prices in all the other sectors are given by :
\begin{equation}
(p^{A^{-i}})^T  =  \Big((inp^{A^{-i}}_i)^T + (gva^{-i})^T + (t^{-i})^T \Big)\, \underline{Y}^{-i} \, (I - A^{-i})^{-1} \label{cost-push-scen}
\end{equation}

This price increase can be interpreted as a measure for how the total cost of the price change is distributed over all sectors. In addition, the cost to final consumers can be determined. 

\subsection{Demand scenarios}



\section{Data}

The data used for the present study are obtained from Statistics South Africa (Stat SA), the South African national statistical agency, and from the OECD. The latter are in itself based on the data collected by Stat SA. The latter offer additional information because of the treatments data underwent (aggregation, seasonal adjustment). %and because the data for South Africa have been put in relation to data  are however useful in some instances because they offer access to different types of pre-treatments . 

\subsection{Datasources}

\subsubsection{IOT}	

To take into account that the demand for imported goods does not directly create value or give rise to jobs domestically through theire production (although the imported goods might be necessary for a particular production process), separate IOTs for domestic output and imported goods and services are needed. The most recent input output tables available from Stat SA are for the years 2013 and 2014. The level of detail is at 50 sectors. However, no table is available that specifies uses of imports versus uses of domestic production. Such a table is provided by the OECD, at a detail of 35 sectors ; the most recent one is for 2011 at the time of writing.

To construct a separate IOT for domestic production and for imports, the 2011 OECD tables were used to estimate such tables at the 50-sector level for the year 2014. This was done using an optimization procedure, detailed in appendix \ref{IOTopti}.

%\footnote{The OECD provides such tables at a level of detail of 35 sectors, the most recent one for the year 2011. The South African statistical agency provides a table with a level of detail of 50 sectors, the most recent for the year 2014, but that does not distinguish the use of imports.}. 

\subsubsection{Employment data}
For the present study employment data are needed of the same sectoral aggregation as the 50-sector IOT. These are not readily available but are constructed on the basis of two types of employment data and in addition more detailed data for certain sectors, all collected by and available from Stat SA.
 
Data on employment are gathered by the South African statistics agency through two types of channels. The Quarterly Employment Survey is conducted with enterprises \citep{QES2018}. The Quarterly Labour Force Survey is a survey taken from households from individuals aged 15 years and over. Contrary to the former, the data thus obtained cover also the agricultural sector, self-employed workers whose businesses are unincorporated, unpaid family workers, and private household workers among the employed \citep{QLFS2018}.

The QES offers employment numbers for the all sectors, excluding agriculture and private households, at a level of detail of 91 sectors. The QLFS offers data at a disaggregation of the economy of 10 sectors. Thus, while the QLFS covers more types of workers, the QES is more detailed.

The IOT, although on average less detailed than the employment data sourced from the QES, offers for a few sectors a different or more detailed disaggregation than the 91 sectors used by the former. This is the case for "Agriculture" (SIC 1), the "Mining and quarrying" sector (SIC 2), and for "Coke oven products; petroleum refineries; processing of nuclear fuel" (SIC 33). To obtain estimates compatible with the IOT, assumptions and additional data are used. Also, to obtain final numbers, the choice is made to include all types of workers, self-employed workers whose businesses are unincorporated, unpaid family workers, and private household workers. %i.e. to suppose total employment as given by the QLFS.
 
The last assumption implies that the final data set results in the same numbers when aggregated to the 10 sectors of QLFS. Thus, to obtain data for agriculture, the QLFS estimate is used, disaggregated to Agriculture (SIC 11), Forestry (SIC 12), Fishing (SIC 13) according to their shares in output for the year 2014. 

The employment data for the mining sector are disaggregated into gold and non-gold in the QES. To get numbers for the same disaggregation as the IOT table (SIC 21 coal, 23-24 gold and metals, 25 other), data are used from the 2015 report from Stat SA on the Mining industry \citep{mining2015}. In the report, both direct employees of the mining industry are included, as employees through labour brokers, of subcontractors and capital employees\lies{what are those?}, for xx subsectors of the mining industry. %The total thus obtained in the report is 490146, versus 459271 in the quarterly employment series 2009-2018. Also, the proportion working for the gold mines SIC 23 is different. In the quarterly series, 21\% of people in the mining sector work for gold, versus 25\% according to the survey.
To obtain an estimate for the number of people working in SIC 21, 23+24 and 25, the non-gold share of the quarterly series is disaggregated according to the shares of the mining report. 

To obtain the disaggregation of sector 33 as in the IOT, it is assumed that shares of workers employed in each subsector are equal to the shares of output.

Finally, after the QES data are aggregated into 50 sectors, volume adjustments are made based on the QLFS numbers.

%The conversion between SIC and ISIC rev.3 https://www.statssa.gov.za/additional_services/sic/descrip6.htm
%https://www.statssa.gov.za/additional_services/sic/preface.htm

Employment data are also available from the OECD at a level of 7 sectors. These allow to make a final check of the obtained data (figure \ref{ComparisonOECD-STATSA-Empldata_2014}). The main difference found is between sectors ISIC rev.3 G-I and L-P. G-I covers Wholesale and Retail trade and L-P service activities among which Private Households. 

\begin{figure}[!h]
	\centering
	\includegraphics[width=0.8\textwidth]{../graphs/ComparisonOECD-STATSA-Empldata_2014.pdf}
	\caption{\label{ComparisonOECD-STATSA-Empldata_2014}Comparison of data sources on the number of persons employed per sector in South Africa, 2014.}
\end{figure}


%\subsubsection{Scenarios}

\

%\section{Results}
%\section{Descriptive statistics of the data used / macro description of the South African economy}

\section{Descriptive statistics}

\subsection{Uses and resources}

Figures \ref{Uses_Resources_percent_2014} and \ref{Uses_Resources_abs_2014} give a visual summary of the input-output table. For each of the 50 sectors of the IOT, they provide an overview of how goods are obtained --- imported versus domestically produced --- displayed in the right-hand columns and how they are used in the left-hand column. How imports and production are used is divided up between intermediate consumption, final uses such as consumption (by households and government) and building of the capital stock (accumulation of the fixed stock and inventories) and for export. Left- and right-hand columns necessarily have the same height because of equation \ref{IOT}. Figure \ref{Uses_Resources_percent_2014} shows per sector the share, dissimulating the information on the relative size of the different sectors. Figure \ref{Uses_Resources_abs_2014} retains this last information ; however, due to the large size of some service sectors, the distribution of uses and resources is somewhat lost for many.  

Figure \ref{Uses_Resources_abs_2014} shows from what sectors South Africa obtained export revenue in 2014 (the year for which the most recent IOT is available). These are in order of magnitude : Metal ores (SIC 23-24), Transport (SIC 71-74), Trade (SIC 61-63), Basic iron and steel (SIC 351 \& 353), Coal and lignite (SIC 21), Other services nec, Other mining (SIC 25), Motor vehicles (381-387). Three of these are services (Transport, Trade and Other services nec). %The non-service output in this list is made up of Metal ores, Basic iron and steel (SIC 351 \& 353), Coal and lignite (SIC 21) and Motor vehicles. 
As pointed out by the CPI, the export goods for which global demand is (very) likely or possibly different between a BAU and a 2DS scenario are some metal ores such as ... and coal. Export of metal ores provides about xx \% of GDP and of coal about $1\%$\li{small tables}.

Judging from the dark blue rectangles of the right-hand columns of figure \ref{Uses_Resources_abs_2014}, the most imported goods are mostly manufactured products, with at the third place Other mining, which includes crude oil. For crude oil the price on the international market is expected to be lower in a 2DS versus BAU. 

Transport costs higher ? Impact on place in GVC ? OECD : between 15-20 of export is imported (check).


\begin{landscape}
	\thispagestyle{empty}
	\begin{figure}[!ht]
		%\centering
		\vspace{-30pt}\hspace{-36pt}\includegraphics[width=1.5\textwidth]{../graphs/Uses_Resources_percent_2014.pdf}
		\caption{\label{Uses_Resources_percent_2014}For a disaggregation of the economy into 50 sectors, the graph shows how each sector obtains its resources, through domestic production (Output, in light blue) or through Imports (in dark blue) in the right vertical bar. The left bar shows the use made of the resources : Exports (red), Consumption which represents both consumption by households and by government, Capital formation, which includes Fixed capital formation and changes in inventories (very pale blue and light orange) and Intermediatie consumption (orange). The value of each type of use and resource is in million Rand. The figures are for the year 2014. Data source : IO tables from Stats SA.}
	\end{figure}	
\end{landscape}


\begin{landscape}
	\thispagestyle{empty}
	\begin{figure}[!ht]
		%\centering
		\vspace{-30pt}\hspace{-36pt}\includegraphics[width=1.5\textwidth]{../graphs/Uses_Resources_abs_2014.pdf}
		\caption{\label{Uses_Resources_abs_2014}For a disaggregation of the economy into 50 sectors, the graph shows how each sector obtains its resources, through domestic production (Output, in light blue) or through Imports (in dark blue) in the right vertical bar. The left bar shows the use made of the resources : Exports (red), Consumption which represents both consumption by households and by government, Capital formation, which includes Fixed capital formation and changes in inventories (very pale blue and light orange) and Intermediatie consumption (orange). The value of each type of use and resource is in million Rand. The figures are for the year 2014. Data source : IO tables from Stats SA.}
	\end{figure}	
\end{landscape}

%\subsubsection{Employment}

\subsection{Decomposition employment content}
Figures \ref{Decomposition_direct_indirect_avg} and \ref{Decomposition_direct_indirect_absolute} shows the total employment content --- the number of persons employed to generate 1 million rand of final output --- for each of the 50 sectors of the IOT, decomposed into jobs that are directly generated within the concerned sector ("Direct" in red), and jobs that are situated in other sectors, due to demand stemming from the concerned sector for intermediate products ("Indirect" in blue).

In figures \ref{Decomposition_PQ} and \ref{Decomposition_direct_indirect_avg}, the difference between the employment content (EC) for each sector and the average employment content over the economy is visualized. In figure \ref{Decomposition_direct_indirect_avg},  the EC is decomposed into its direct and indirect components, and how those relate to the average number of jobs directly and indirectly generated. In figure \ref{Decomposition_PQ}, total employment content is decomposed according to eqn. \ref{}, specifying whether high/low employment in a particular sector is mainly due to low/high wages, high/low domestic demand, etc.

Overall, it can be seen that when the employment content of a sector is high, it is generally mainly due to low wages in the given sector. 

The employment content in the mining sector is lower than average, because of a low labour share and higher wages than the economy's average. The factor "local final demand" contributes above average ; this is due to low import and high export. Figure \ref{Decomposition_direct_indirect_absolute} shows that about half the jobs created by final demand for coal, gold etc. is created indirectly, by the production of intermediate inputs needed for mining. This means that for every job created or destroyed in the mining sector because of changes in final demand, another job is created or destroyed throughout the economy, in other sectors.

%The mining industry generates less jobs per unit of final output than the economy's average. Both the mining of coal and lignite and of metal ores generate as many jobs through directly as indirectly per unit of final demand. The number of indirect jobs created is close to the average. It is the amount of directly generated jobs that is well below the average. Figure \ref{ECdecomposition-avg-PQ_2014.pdf} shows the reasons behind the lower-than average employment content. A low labour share and high wages are the main reason ; there is a positive contribution from	the factor designated by "local final demand" which contributes positively to the labour share when imports are small or when exports are large. Both are the case for the two mining sectors. In Other mining (25), which includes the import of crude oil, there is a similar contribution to a low employment content from a low labour share and higher-than-average wages ; additionally, due to the large share of imports in this sector, the employment content sinks even further below the economy's average. 


\begin{figure}[!ht]
	\centering
	\includegraphics[width=0.95\textwidth]{../graphs/ECdecomposition-avg-PQ_2014.pdf}
	\caption{\footnotesize \label{Decomposition_PQ}The horizontal bars show for each sector a decomposition of the employment content into five factors, compared to the average composition of the employment content over the whole economy, excluding SIC 01/ISIC P Private households with employed persons. The factors give an indication of the sources of the differences in employment content. The components of a horizontal bar situated to the left/right of the O mark are smaller/bigger than the average. The interpretation of each of the five components is as follows : 1) when Wages is bigger than the average, this means that wages are smaller in this sector than the average wage throughout the economy, 2) when Labour share is bigger than the average, the sector's labour share is bigger than the average, 3) when Taxes is bigger than the economy's average, the tax burden on the sector is smaller than the average, 4) when Local final demand is smaller than the average, thus on the left of the 0, then local demand is relatively smaller and imports are relatively bigger, 5) when Intermediate imports is situated on the left, then the importation of intermediate inputs is higher than the economy's average.}
\end{figure}	


\begin{figure}[!ht]
	\centering
	\includegraphics[width=0.95\textwidth]{../graphs/ECdecomposition-avg-PQ-direct_2014.pdf}
	\caption{\footnotesize \label{Decomposition_PQ_direct}The horizontal bars show for each sector a decomposition of the direct employment content into five factors, compared to the average composition of the employment content over the whole economy, excluding SIC 01/ISIC P Private households with employed persons. The graph is to ne read in the same fashion as figure \label{Decomposition_PQ}. %The factors give an indication of the sources of the differences in employment content. The components of a horizontal bar situated to the left/right of the O mark are smaller/bigger than the average. The interpretation of each of the five components is as follows : 1) when Wages is bigger than the average, this means that wages are smaller in this sector than the average wage throughout the economy, 2) when Labour share is bigger than the average, the sector's labour share is bigger than the average, 3) when Taxes is bigger than the economy's average, the tax burden on the sector is smaller than the average, 4) when Local final demand is smaller than the average, thus on the left of the 0, then local demand + exports are relatively smaller and imports are relatively bigger, 5) when Intermediate imports is situated on the left, then the importation of intermediate inputs is higher than the economy's average.
	}
\end{figure}	

\begin{figure}[!ht]
	\centering
	\includegraphics[width=0.95\textwidth]{../graphs/ECdecomposition-avg-PQ-indirect_2014.pdf}
	\caption{\footnotesize \label{Decomposition_PQ_indirect}The horizontal bars show for each sector a decomposition of the indirect employment content into five factors, compared to the average composition of the employment content over the whole economy, excluding SIC 01/ISIC P Private households with employed persons. The graph is to ne read in the same fashion as figure \label{Decomposition_PQ}. %The factors give an indication of the sources of the differences in employment content. The components of a horizontal bar situated to the left/right of the O mark are smaller/bigger than the average. The interpretation of each of the five components is as follows : 1) when Wages is bigger than the average, this means that wages are smaller in this sector than the average wage throughout the economy, 2) when Labour share is bigger than the average, the sector's labour share is bigger than the average, 3) when Taxes is bigger than the economy's average, the tax burden on the sector is smaller than the average, 4) when Local final demand is smaller than the average, thus on the left of the 0, then local demand + exports are relatively smaller and imports are relatively bigger, 5) when Intermediate imports is situated on the left, then the importation of intermediate inputs is higher than the economy's average.
	}
\end{figure}


\begin{figure}[!ht]
	\centering
	\thispagestyle{empty}
		\includegraphics[width=\textwidth]{../graphs/ECdecomposition-avg-direct-indirect_2014.pdf}
	\caption{\label{Decomposition_direct_indirect_avg}}
\end{figure}	



\begin{figure}[!ht]
	\centering
	\thispagestyle{empty}
		\includegraphics[width=\textwidth]{../graphs/ECdecomposition-direct-indirect_2014.pdf}
	\caption{\label{Decomposition_direct_indirect_absolute} The length of the horizontal bars shows the employment content in absolute terms for the South african economy decomposed into 91 sectors. The employment content is measured as the number of jobs created, throughout the economy, per million rand of final demand. Total employment content is decomposed into direct jobs, created within each sector, and indirectly generated jobs, due to demand for intermediate input to other sectors. The dotted blue and gray lines indicate the average indirect and total employment content.}
\end{figure}	

\clearpage

\subsection{Key sectors}
The difference in direct and indirect effects generated by the 2DS versus the BAU scenarios will be estimated with the the aid of the most recent input-output table (2014). Throughout the analysis, it will be assumed that the inter-dependencies between the different sectors in the economy --- the value of input bought by sector $i$ from $j$ to produce 1 mR of output by $i$ --- are assumed to be as in the year 2014. %In a few (1 ?) cases,  

In reality, this assumption does not hold. The 2014 IOT used is measured in value and its numbers might change both because of changes in the relative price of commodities and because over time sector's systems of production change, as reflected in and defined by the amount of inputs used per unit of output. For example, the cost of coal use for 1 $kWh$ of electricity can change on the one hand because the domestic price of coal changes, without changing the $kg$s used, or because solar panels replace coal.

The scenario comparisons in section \ref{scenarioAnalysis} cover the period 2018-2035, over which both relative prices and systems of production will have changed, both under BAU as under 2DS assumptions. This study will not attempt to estimate how the IOT might change over the modeling period ; the assumption is that the 2014 IOT is a good approximation. In order to get picture of the information the 2014 IOT provides, the following subsections describe key sectors for transition scenarios, their place within the economy and how they are linked to other sectors. This will give a first idea of the dominant risks, via indirect effects, of the external transition scenarios. Information on the two types of transmission channels is set forth : how changes in one sector impact others because they use the former's commodity as input and secondly because it uses inputs from others. 


\subsubsection{Coal}

\ref{ChamberofMines2018Strategy}
\ref{MineralsCouncil2018Facts}
\ref{burton2018coal}

At the beginning of the period 2010-2017, the mining sector (SIC 2) represents about 8\% of GDP, contributing about half a percentage point less to GDP towards the end of the period \citep{P0441StatSA2018Q2} (measured in constant 2010 prices). 

Within the mining sector, coal mining (SIC 21, Coal and lignite) contributed 24.4\% in 2012 and 28.1\% in 2015 to the sector's total income \citep{mining2015}. The primary source of income of the coal mining industry is exports, which represent 55.5\% of production. Most of the remaining share is sold to industries who use coal as inputs (41.3\%), notably for electricity production. Products of the coal mining industry are almost basically obtained through domestic production, only 0.4\% of the coal used in the economy is imported (table \ref{Mining2104_UR} and figure \ref{Uses_Resources_percent_2014}).

\begin{table}[!h]
	\centering
	\begin{tabular}{rc|cl}
		\hline
		& Uses & Resources  \\ 
		\hline
		Exports & 55.5\% & &\\ 
		Intermediary inputs &  41.3\%  & 99.6\% &  Domestic production\\ 
		Inventories &  2.4\% & 0.4\% & Imports    \\ 
		Consumption & 0.8\% &  &\\ 
		\hline
	\end{tabular}
	\caption{\label{Mining2104_UR}Sector \emph{Coal and lignite}, SIC 21. The right hand column, Resources, shows how the total amount of the sector's commodity available to the South-African economy is obtained. The left hand column, Uses, shows how the sector's commodity is used domestically and how much is exported. The data are for the year 2014. Source, \cite{IOT2014}.}
\end{table} 

Hence, the export of coal directly contributed in 2014 about 1.2\% to GDP, with similar shares for more recent years (which are ?). This implies that movements in coal exports impact GDP significantly. 

Furthermore, CPI analysis hypothesizes that the domestic price of coal might increase due to the loss of cross-subsidies from exports. This price increase represents a cost that can spread throughout the whole economy if the increase in the domestic coal-price is transferred to the price of electricity. Indeed, coal inputs represent a cost of roughly 0,9\% of GDP to the South African economy in the form of intermediary inputs, of which more than half (57 \%) is used for electricity production. %Thus, movements in the domestic coal-price can get transferred to the whole economy via the electricity price.
Figure \ref{ElecPiechart} shows that all sectors, including households, spend rather similar shares of expenses on electricity.% in a rather evenly distributed manner. 
Electricity costs on average xx \% $\pm$ xx\% of output (of costs / ?). %Thus, in the case, as CPI analysis shows, the domestic price of coal increases due to the loss of cross-subsidies from exports, the 
A possible increase of the price of electricity will thus impact the entire industry. Households spend about 2.5\% of total expenditures on electricity\footnote{All mentioned figures are for the year 2014, obtained from the 2014 input-output table \citep{IOT2014}}. However, % depending on how the increase of the cost of electricity propagates throughout the economy --- either the profitability of industries is impacted and/or the cost increase is transferred in the price of commodities --- 
final consumers and notably households might bear more of the increased cost of coal if the cost is transferred to final consumption goods then what is suggested by their direct electricity consumption.

% latex table generated in R 3.4.4 by xtable 1.8-3 package
% Thu Sep 27 16:41:51 2018
\begin{table}[ht]
\centering
\begin{tabular}{rrr}
  \hline
 & \% intermediate inputs & \%, accumulated \\ 
  \hline
Electricity, gas and water & 57.2 & 57.2 \\ 
  Coke oven manufacture & 18.0 & 75.2 \\ 
  Metal ores & 6.2 & 81.4 \\ 
  Paper & 4.4 & 85.8 \\ 
  Basic iron and steel & 4.3 & 90.1 \\ 
  Distribution of water & 2.5 & 92.6 \\ 
  Non-metallic minerals & 2.0 & 94.6 \\ 
  Other mining & 1.1 & 95.7 \\ 
  Other community activities & 1.0 & 96.7 \\ 
  Food & 0.9 & 97.6 \\ 
   \hline
\end{tabular}
\end{table}


% latex table generated in R 3.4.4 by xtable 1.8-3 package
% Thu Sep 27 16:41:51 2018
\begin{table}[ht]
\centering
\begin{tabular}{rrr}
  \hline
 & \% intermediate inputs & \%, accumulated \\ 
  \hline
Metal ores & 19.8 & 19.8 \\ 
  Nuclear fuel & 8.4 & 28.3 \\ 
  Electricity, gas and water & 8.1 & 36.4 \\ 
  Real estate activities & 7.7 & 44.0 \\ 
  Basic iron and steel & 6.7 & 50.7 \\ 
  Computer activities & 5.2 & 55.9 \\ 
  Trade & 4.3 & 60.2 \\ 
  Other chemicals & 4.2 & 64.4 \\ 
  Agriculture & 3.6 & 68.0 \\ 
  Transport & 3.2 & 71.2 \\ 
   \hline
\end{tabular}
\end{table}


The above describes how different coal export demand scenarios can have an effect on the economy as a whole because coal is used as an input, directly by some industries and indirectly by many through electricity use. The second channel through which different export scenarios impact the economy is due to the demand addressed to other sectors by the coal sector. Table \ref{} evaluates this effect in terms of the sector's employment content, defined as the number of jobs created --- throughout the economy --- by 1 million Rand of final demand addressed to the coal industry. For the coal sector, final demand is for 94.5\% made up of exports\footnote{Figure obtained from table \ref{Mining2104_UR}.}. 

%\input{../tables/Ind4_E_VA_2014.tex}
% latex table generated in R 3.4.4 by xtable 1.8-3 package
% Tue Sep 25 16:58:17 2018
\begin{table}[ht]
\centering
\begin{tabular}{rllrrrr}
  \hline
 & Industry & Code & Ec & E, 2014 & VAc & VA, 2014 \\ 
  \hline
1 & Coal and lignite & 21 & 0.72 & 25581.44 & 0.60 & 21293.75 \\ 
  2 & Transport & 71-74 & 0.28 & 9847.72 & 0.09 & 3360.13 \\ 
  3 & Trade & 61-63 & 0.20 & 6963.92 & 0.03 & 933.20 \\ 
  4 & Computer activities & 86\_88 & 0.11 & 3801.96 &  &  \\ 
  5 & Financial intermediation & 81 & 0.03 & 931.21 & 0.01 & 505.48 \\ 
  6 & General machinery & 356-9 & 0.02 & 877.14 &  &  \\ 
  7 & Structural metal & 354-5 & 0.02 & 843.29 &  &  \\ 
  8 & Other services nec & 95\_96\_99 &  &  & 0.02 & 748.47 \\ 
  9 & Auxiliary financial & 83 &  &  & 0.02 & 630.31 \\ 
  10 & Electricity, gas and water & 41 &  &  & 0.01 & 496.87 \\ 
   \hline
\end{tabular}
\end{table}


The total employment content of the mining sector is lower than the economy's average, as can be seen in figure \ref{Decomposition_direct_indirect_avg}. This means that for 1 Rand of final demand addressed to the mining industry, less jobs are generated throughout the economy than when the same value of demand is distributed evenly over all sectors. On average in the economy, 2.48 jobs are created for 1 mR of final demand (local demand + exports) whereas in coal mining, 1.58 jobs are sustained throughout the economy. %by 1 mR of final demand addressed to the sector, by the sector's activity, of which. 
The direct employment content, which is the jobs generated within the coal sector, equals 0.72. The remaining share is the indirect employment content and equals 0.86. This is the number of jobs generated in other sectors by the export of coal. Table \ref{} shows that 68\% of these indirectly created jobs are situated in three sectors : Transport (33\%), Trade (23\%) and Computer activities (13\%). 

The Transport sector is thus, after coal mining itself, the first sector impacted by different coal export scenarios. 

% Look at what percentage of jobs comes from coal.

%The economic activity generated by the demand for final goods addressed to the coal mining sector, which is primarily export goods, generated by the sector's use of intermediate inputs, is situated in descending order in terms of the number of jobs created, in the Transport sector (SIC 74-71) with 0.27 jobs (per 1 mR final demand addressed to the mining sector), 0,19 to the Trade sector, 0,11 to Computer activities; 0,026 of Financial intermediation and 0,05 jobs divided equally over sectors 354-359, Structural metal and General Machinery.

\begin{table}[ht]
	\centering
	\renewcommand*{\arraystretch}{1.15}
	\begin{tabular}{lp{20pt}lcc | c}
		\toprule
		%		\multicolumn{3}{l}{Inputs} &\multicolumn{3}{c}{} \\ 
		%\multicolumn{3}{c}{Origin of jobs} &\multicolumn{2}{c}{Export scenarios} &  \\ 
		\multicolumn{3}{c}{Inputs} & 2014 & 2017 & 2014, direct \& indirect\\ 
		\midrule
		\multicolumn{3}{l}{Wages (\% of output)}  		 & 17.8 & 17.2 &  \\ 	
		\multicolumn{3}{l}{Intermediate inputs (\% of output)}  & 39.4  & 46.8  & 44.0 \\ 
		%& 						 & Local & 31.2 &  37.0 &  35.4\\ 
		%& 						 & Imports & 8.1 &  9.8  & 8.6 \\
		& 						 & Local (\% of total input costs) & 79.4 &  79.0 &  80.5\\ 
		& 						 & Imports (\% of total input costs) & 20.6 &  21.0  & 19.5 \\
		\bottomrule
	\end{tabular}
	\caption{\label{costShares_2014_2017}\small Input and wage costs as a share of output in the Coal sector in 2014 and in 2017. The first to columns on the left, "2014" and "2017" provide direct input costs. The column on the right "2014, direct \& indirect" takes into account the total output, direct and indirect, required of all sectors to satisfy coal export demand. The input costs are the input costs summed over all sectors. Source : \cite{IOT2014} and \cite{MineralsCouncil2018Facts}.}
\end{table}


\subsubsection{Electricity}

\begin{figure}[!h]
	\centering
	\includegraphics[width=0.5\textwidth]{../../biblio/Eskom/EskomPE.png}
	\caption{\label{Eskom2017PE}Breakdown of primary energy cost of electricity generation, 2017. Between parentheses, the share of generated volume per technology ; the numbers are total cost per technology in million Rand. Source : Eskom Integrated report 2017 \citep{Eskom2017AR}.}
\end{figure}

Electricity is in South Africa provided by state-owned Eskom. In the first eight months of 2018, 91 \% of electricity was generated by Eskom \citep{P4141_201808}. The remaining share was generated by independent power producers (IPPs), which are mostly (all?) renewables. IPPs sell generated power to Eskom who is the sole transmitter and provides 60\% of distribution ; 40\% is distributed by municipalities.

In 2017, 85 \% of electricity was generated through coal-plants \citep{Eskom2017AR}. The remaining 15\% stems from the country's sole nuclear plant, gas-turbines, 4\% from renewables produced by IPPs and a small share is imported (figure \ref{Eskom2017PE}). Coal is historically the primary energy source of choice in South Africa, because easily mined domestically and thus cheap. Even though coal-prices have been rising, in 2018, the 85\% of coal-generated electricity represented 63 \% of the total primary energy cost for Eskom. Notably, the primary energy cost per $MWh$ of coal is 6-7 times lower than the price Eskom pays for electricity generated by IPPs \citep{Eskom2017AR}.  

\begin{figure}[!t]
	\centering
	\includegraphics[width=0.7\textwidth]{../../biblio/electricity/Eskom-electricity-prices.jpg}
	\caption{\label{Eskom-electricity-prices} The evolution of the real and nominal price of electricity in South Africa from 1974-2019. Source : \href{https://mybroadband.co.za/news/business/68724-how-lecturers-tore-eskoms-tariff-case-apart-in-25-minutes.html}{Eberhard, A. and Logan, C.}.}
\end{figure}


\begin{figure}[!t]
	\centering
	\includegraphics[width=0.9\textwidth]{../../biblio/StatSAreports/Electricity/ElecSA2010-2018_P4141_201808.png}
	\caption{\label{Elecvolume_2010_2018}The volume of electricity generated from 2010-2018. Source : \citep{P4141_201808}.}
\end{figure}

Since 2008, real electricity prices have risen drastically in South Africa (figure \ref{Eskom-electricity-prices}). Part of the price increases are due to the rising cost of coal and part to the necessity to raise cash for investment in new capacity. The steady rise of the price is seen as a factor in the decline of electricity demand over the period 2011-2015 (figure \ref{Elecvolume_2010_2018}). Several studies have investigated the effect of the price hikes on demand, on an aggregate and on a sectoral level, for industry and for consumers, see \cite{goliger2018electricity} for an overview.

%It is against this backdrop that the additional source of increase in the generation cost of electricity under a 2DS --- the loss of cross-subsidies for the domestic coal-price because of reduced export --- has to be evaluated. 

It is against this backdrop that the possible higher domestic coal-price under 2DS has to evaluated. Indeed, the loss of cross-subsidies for coal consumed domestically due to lower export-volumes under 2DS constitute an additional source of increase in the generation cost of electricity under a 2DS versus a BAU scenario.

%\citep{inglesi2010forecasting}

\paragraph{Household electricity demand}



\subsubsection{Platinum metal group}

The mining of metal ores generates 1.89 jobs throughout the economy per mR of final demand. 

\subsubsection{Oil}

The importation of crude oil falls under mining sector SIC 22, which is classified in the IOT under Other mining. The import figure of this sector corresponds to the import of crude oil. 

Refinement of crude oil falls under SIC 332, which is in the IOT grouped with 331 (coke oven products) 

\subsubsection{Global value chains}

Under a 2DS scenario, supply chains are necessarily "greener". 

How is South Africa's economy possibly exposed to the greening of global supply chains ? Is it risk for the economy ?
- the most obvious factor is the risk posed to coal exports
- what share of the SA economy is dependent on its inscription in global supply chains ? Are the economic activity and the jobs hence created under risk ? 
- the importance generally of global supply chains might be affected if transportation cost increases because of the necessity to reduce emissions
- companies might start to take into account the carbon content of supplies/intermediate inputs


\citep{von2016energy}

\clearpage
\section{Prospective analysis}

\subsection{Coal}

\subsubsection{Coal export}


\begin{figure}[!h]
	\centering
	\includegraphics[width=\textwidth]{../graphs/Coal-Export_CPI_VolPrice_BAU_2Deg.pdf}
	\caption{\label{Coal-Export_CPI_VolPrice_BAU_2Deg}\small Projections for the period 2018-2035 of the volume of coal exported by South Africa in million tons coal equivalent (mtce), the export price in USD/mtce and the total value of exports in USD under CPI's business as usual (BAU) scenario and 2 degrees (2Deg) scenario. Source : CPI model, based on data from Wood MacKenzie.}
\end{figure}

In a world economy developing in a way that is compatible with average global warming limited to 2° degrees, global demand for coal will be drastically lower. Over the period 2018-2035, South Africa is estimated by CPI's models to export about half the volume of coal under the 2DS versus a under the BAU scenario (left graph of figure \ref{Coal-Export_CPI_VolPrice_BAU_2Deg}). %CPI's models estimate that the volume of coal exported by South Africa over the period 2018-2035 under a 2DS is about half the volume under BAU, with considerably lower export demand 
On the long term, towards the end of the modeling period, the volume of exported coal drops below 10 million tons coal equivalent (mtce) in the 2 DS scenario which is about 85\% lower than under BAU. Moreover, the price on the global market is expected to be lower under 2DS, resulting in the lower profitability of the sector on average and in pressure %  first instance in a lower profitability of the sector in general and in particular in the 
on the financial sustainability of operating mines. % and their possible closing \citep{CPI2019SA}. 
The projected export trajectories under both scenarios are presented in figure \ref{Coal-Export_CPI_VolPrice_BAU_2Deg}. The area between the two lines of the graph on the right represents the revenue from coal exports at risk under the 2DS scenario. The low price fetched for coal on the export market exacerbates the effect of the revenue loss from low demand under 2DS. 

This loss in revenue impacts firstly the coal sector itself and secondly, to varying degrees, the rest of the economy. As set out previously, this is because part of the revenue obtained from coal exports is spent on domestically produced inputs, necessary for mining coal, and becomes revenue for the suppliers. Thus, the risk posed under a 2DS by the lower revenue obtained from coal exports is not only a risk for the sector itself, but also for other parts of the economy. Furthermore, because the coal sector's suppliers themselves spent part of their sales revenue (obtained from the coal sector who obtained its revenue from export) on inputs, the coal export revenue spreads throughout the economy as value added of (almost) all sectors. Value added is here defined in a broad sense, i.e. as output, which is taken to be the same as total sales revenue from domestic final goods, minus the cost of inputs. It thus includes, referring to the IOT in table \ref{IOT_importsUses}, net taxes on products, other taxes, wages and gross operating surplus. In addition, part of the revenue is spent on imports. The value added of each sector thus flows to government via taxes, to workers in the form of wages and a part of gross operating surplus remunerates investors in the form of debt payments or dividends. 
%the value created throughout the economy by the export of coal benefits different actors in the economy. For each sector, what is left over from sales revenue after paying suppliers splits into three parts : payment to workers, payment of taxes to different layers of government and a remaining share labeled gross operating surplus. This last part is used for investments and exceptional maintenance costs of the production infrastructure and is distributed to its investors, who are the de factor owners of the production infrastructure, in the form of debt payments, payments of dividends to shareholders. %This is the part that is mostly distributed to the owners of the fixed capital of the sector, i.e. to the economic actors that invested in the sector's production infrastructure and expect a return on their investment. 
The input-output methodology does not provide information on who are the holders of the fixed capital of each sector. This information is provided by the finance-based, sector and asset-level study performed by CPI \citep{CPI2019SA}. 

\begin{figure}[!b]
	%	\centering
	\hspace{-10pt}\includegraphics[width=1.1\textwidth]{../graphs/ExportRevenue_Decomposition_BAU_2Deg.pdf}
	\caption{\label{ExportRevenue_Decomposition_Diff}\small The graph shows the revenue from the export of coal in constant Rand (2014), over the period 2018-2035, under the business as usual scenario (BAU) on the left and under the 2°degrees (2Deg) scenario on the right. The export revenue is decomposed into importation costs of intermediate goods (red), taxes, wages and gross operating surplus, for the coal sector (yellow) and for the rest of the economy (blue). Is is assumed that the prices changes of the exported coal impact tax revenue and gross operating surplus. All domestic prices and wages stay constant over the whole period, except  }
\end{figure}

\begin{table}[ht]
	\centering
	\renewcommand*{\arraystretch}{1.15}
	\begin{tabular}{lp{20pt}lcccc}
		\toprule
%		\multicolumn{3}{c}{Ori} &\multicolumn{3}{c}{Number of persons employed, 2035} \\ 
		%\multicolumn{3}{c}{Origin of jobs} &\multicolumn{2}{c}{Export scenarios} &  \\ 
		\multicolumn{3}{l}{Revenue destination } & \makecell{BAU\&2Deg \\2018} & \makecell{BAU\ \  \\ 2035\ \ } & \makecell{\ \ 2Deg \\\ \ 2035} & \makecell{Value at risk \\(in \ldots rand)}\\ 
		\midrule
		\multicolumn{3}{l}{Value added, total}   & 86.7  & 90.6 & 84.6 & \\ 	
		& 						 & Wages & 28.9 & 20.5  & 33.4 \\ 
		& 						 & Gross operating surplus & 55.2  &  67.2  & 48.7 &\\ 
		& 						 & Taxes & 2.6 &  2.9 & 2.5 & \\
		\midrule
		\multicolumn{3}{l}{Value added, coal sector}  & 60.6 & 72.3 & 54.8 \\ 
		& 						 & Wages & 17.9 &  12.7 & 20.8 \\ 
		& 						 & Gross operating surplus & 41.5 & 57.6  & 32.9 & \\ 
		& 						 & Taxes & 1.5 &  2.1 & 1.2 \\
		\midrule
		\multicolumn{3}{l}{Value added, rest of economy}  & 25.6  & 18.2 & 29.8 \\ 
		& 						 & Wages & 10.9 & 7.8  & 12.7 \\ 
		& 						 & Gross operating surplus & 13.7 &  9.7  & 15.8 & \\ 
		& 						 & Taxes & 1.1 & 0.8  & 1.3 \\
		\midrule
		\multicolumn{3}{l}{Imports, total}  		 & 13.3  & 9.4  & 15.4 & \\ 	
		& 						 & by coal sector & 8.1 &  5.8 & 9.4 & \\ 
		& 						 & by rest of economy & 5.1 &  3.6  & 5.9 & \\ 
		\bottomrule
	\end{tabular}
	\caption{\label{ExportRevenue_Decomposition_Diff_table}\small Comparison of the impact of export demand on the number of jobs related to the activity in the coal mining sector (SIC 21). The percent difference in coal exports between the two scenarios is equal to that between BAU and 2DS in 2035 as projected by CPI. The BAU scenario is approximated by the South African economy as in the year 2014.}
\end{table}


As detailed in section \ref{Section_decomposition}, the input-output methodology allows to decompose export revenue in imports and in value added over all sectors of the economy. Figure \ref{ExportRevenue_Decomposition_Diff} shows the decomposition of the export revenue trajectories over the period 2018-2035, for both scenarios, in imports (in red), taxes and value added (in yellow for the coal sector and blue for the other sectors). % throughout the economy. 
These estimates are based on simplifying assumptions, such that the structure of the economy remains very close throughout the modeling period as it is in 2014. It is assumed that all the economy's real prices stay constant, including the unitary wage, except for the export price of coal.
%Additionally, low prices further compromise the ability of the sector to pay of its debts. We assume here, in a first approximation, that 
The coal export price affects only taxes and gross operating surplus (of the coal mining sector)%(the capital share) 
such that the sum of value added and imports equals sales revenue. This assumption is maintained both under BAU with sustained prices as in the 2DS scenario with low prices. 



At the beginning of the period, 13.2\% of export revenue is spent on imported inputs, 60.6\% becomes value added (including all taxes) %as a direct result of economic activity 
in the coal sector, and 25.6\% becomes value added %distributed over participating actors to the activity in 
of the other sectors of the economy. Table \ref{ExportRevenue_Decomposition_Diff_table} summarizes these figures and specifies what percent of revenue goes to government in the form of taxes, to workers in the form of wages paid, and to the holders of capital. 
Towards the end of the period, the situation is very different under both scenarios. %Whereas under BAU, export volumes and prices are strong, under 2DS, low volumes reduce employment and wages, taxes and gross operating surplus. Indeed, 
Under BAU, the exported volume is on average of the same order of magnitude as at the beginning of the period in 2018. However, the price on the international market is projected to be by up to 30\% higher in the 2030's, making the sector much more profitable than today. Value added in the coal sector rises from 60.6\% of the total export revenue in the beginning of the period to 72.3\% towards 2035. On the contrary, under the 2Deg scenario, the exported volume is in 2035 a tenth of today's. This implies that few mines will be sustained by export demand. Additionally, low prices mean that exporting mines will generate less profit \textit{via} exports. Value added in the mining sector is only 54.8\% in 2035 under 2Deg. In Mines that supply both the domestic and international market \ldots
These estimates are based on simplifying and stringent assumptions. It is assumed that only the export price of coal changes and that 
%Additionally, low prices further compromise the ability of the sector to pay of its debts. We assume here, in a first approximation, that 
all the other prices in the economy, including the unitary wage, are constant throughout the period in real terms. The coal export price is assumed to influence only taxes and gross operating surplus %(the capital share) 
, both under BAU with sustained prices as in the 2DS scenario with low prices. The number of employees per ton of coal mined is also assumed constant throughout the period.
% what kind of changes could be expected in reality ?

%In terms of value added, most of the risk is concentrated in the coal sector. 

Figure \ref{ValueAdded_Employment_Sector_Coal-Export_BAU_2Deg} %and table \ref{VA_Sector_Coal-Export_BAU_2Deg_table} 
compares both value added and employment throughout all sectors in 2035 under BAU and 2Deg with the (current) situation in 2018. Whereas in terms of value added, two thirds is concentrated in the coal sector itself and 10\% in the Transport sector. In terms of employment, about half the jobs sustained by coal export are situated within the coal sector and the other half in the other sectors of the economy. Notably, the Transport sector counts for 17.8\% of the total number of jobs sustained by coal export, Trade for 12.7\% and Computer activities for 5.8\%. The number of workers per ton of coal mined and wages are by hypothesis constant throughout the period. Table \ref{NE_Sector_Coal-Export_BAU_2Deg_table}
lists how many jobs are sustained, in each of the different sectors, by the export of coal and the number of jobs at risk under a 2Deg scenario.

The reason the risk posed by a reduction of coal export is spread differently throughout the economy in terms of value added versus in terms of employment, can be deduced from figure \ref{ECdecomposition-coal_2014}. The net length of the horizontal bars of the graph compare the number of jobs created by a unit of final demand in the coal sector, i.e. the employment content of the coal sector, to the average number of jobs created in the economy by a unit of final demand. The net length of the bars is decomposed into different terms that shed light on the reasons the employment content is higher or lower than the economy's average.

The figures for 2014 are assumed to hold throughout the period. On average, in 2014, 2.24 jobs are created throughout the economy for one million rand of final demand, of which on average, 1.30 are situated within the sector to which final demand is addressed and 0.94 in the rest of the economy. The employment content of the coal sector is only 1.52, of which about half is within the coal sector -- 0.72 -- with 0.80 spread throughout the economy. Thus, the graph is to be read as follows : the sum of the net length of all the bars is equal to difference between the coal sector's ec and the average ec, $-0.72$. The net length of the horizontal bar corresponding to the Coal and lignite sector represents the difference between the average number of jobs created in the economy within the sector to which 1 million Rand of final demand is addressed, and the number within the coal sector sustained by a unit of final demand within the coal sector : $-0.58$. The employment content is a sum of the number of jobs over all sectors of the economy. Since we are looking at the deviation of the employment content of a sector and the economy's average, this is equivalent to summing, over all sectors, of deviations between employment created within each sector (by a unit of final demand of a particular sector) and the average employment created within a sector by a unit of final demand. For some sectors this deviation will be positive, for other negative. This means that demand from the coal sector created jobs in the economy in some sectors where employment per unit of output is higher than on average in the economy, and in some lower. The top/middle horizontal bar of graph \ref{ECdecomposition-coal_2014} groups the sectors with above/below average employment per unit of final demand. x\% of the jobs created by coal export is thus in sectors with higher employment than the economy's average. These sectors are Transport, Trade, ... . The decomposition of the bar shows why employment in these sectors is relatively high : inputs are sourced more often locally than on average, taxes are relatively low, the labour share is higher than the economy's average and wages are lower. These characteristics contrast with the xx\% jobs created in the Manufacturing, .. sectors where more inputs are imported than on average in the economy, taxes are higher, the labour share lower and wages higher. The mining sector itself employs less people per unit of output mainly because of a low wage share and secondly because of high wages. These effects dominate the fact that taxes are lower and less inputs are sourced more often sourced locally.

Whereas the total ec of the coal sector is below the economy's average, these jobs are created in sectors with quite different characteristics. 

%total value added created in the domestic economy according to industries.%al sectors where the value is created. In 2018, of the value added created in the domestic economy, 60.6\% is created within the coal sector, 9.2\% by the Transport sector, due to the transport of coal by rail and truck, 2.6\% in the Trade sector. %and x\% in the rest of the economy. 

Graph 

A reduction of coal export thus poses a risk to these other sectors if the loss of revenue stemming from a reduction of demand for inputs by the coal export is concentrated in specific actors of the South-African economy, such as in companies and/or geographical areas.


%Looking at the issue from the angle of employment, the picture changes. Figure \ref{NumberEmployees_Sector_Coal-Export_BAU_2Deg} and table \ref{NE_Sector_Coal-Export_BAU_2Deg_table} show how many jobs are sustained, in each of the different sectors, by the export of coal. More than half the jobs sustained by coal export are jobs throughout the economy, not in the coal sector. 


%Thus, in terms of value added, most of the risk falls on the coal sector itself, and a part of the transport sector that is very directly linked to the coal sector. 

\begin{table}[ht]
	\centering
	\renewcommand*{\arraystretch}{1.15}
	\begin{tabular}{lp{20pt}lccc}
		\toprule
		\multicolumn{3}{c}{Origin of jobs} &\multicolumn{3}{c}{Number of persons employed, 2035} \\ 
		%\multicolumn{3}{c}{Origin of jobs} &\multicolumn{2}{c}{Export scenarios} &  \\ 
		\multicolumn{3}{c}{} & BAU & 2Deg & Jobs at risk\\ 
		\midrule
		\multicolumn{3}{l}{Jobs sustained by coal exports, total}  		 & 94966 & 13139 & 81827 \\ 	
		\multicolumn{3}{l}{Jobs sustained by coal exports, coal sector}  & 45098 & \ 6240 & \\ 
		\multicolumn{3}{l}{Jobs sustained by coal exports, rest of economy}  & 49868  & \ 6899 & \\ 
		& 						 & Transport & 16868 &  \  2334 & \\ 
		& 						 & Trade & 12022 &  \ 1663  & \\ 
		& 						 & Computer activities & \, 5540 & \, \, 766  & \\
		& 						 & General machinery & \ \,1780 &  \, \, 246
		  & \\  
		\bottomrule
	\end{tabular}
	\caption{\label{coalExport_BAUvs2DS_2035}\small Comparison of the impact of export demand on the number of jobs related to the activity in the coal mining sector (SIC 21). The percent difference in coal exports between the two scenarios is equal to that between BAU and 2DS in 2035 as projected by CPI. The BAU scenario is approximated by the South African economy as in the year 2014.}
\end{table}

\begin{figure}[!h]
	\centering
	\hspace{-10pt}\includegraphics[width=\textwidth]{../graphs/ValueAdded_Employment_Sector_Coal-Export_BAU_2Deg.pdf}
	\caption{\label{ValueAdded_Employment_Sector_Coal-Export_BAU_2Deg}\small The graph shows the revenue from the export of coal in constant Rand (2014), over the period 2018-2035, under the business as usual scenario (BAU) on the left and under the 2°degrees (2Deg) scenario on the right. The export revenue is decomposed into importation costs of intermediate goods (red), taxes, wages and gross operating surplus, for the coal sector (yellow) and for the rest of the economy (blue). Is is assumed that the prices changes of the exported coal impact tax revenue and gross operating surplus. All domestic prices and wages stay constant over the whole period, except  }
\end{figure}




%\begin{figure}[!h]
%	%	\centering
%	\hspace{-10pt}\includegraphics[width=1.1\textwidth]{../graphs/ValueAdded_Sector_Coal-Export_BAU_2Deg.pdf}
%	\caption{\label{ValueAdded_Sector_Coal-Export_BAU_2Deg}\small The graph shows the revenue from the export of coal in constant Rand (2014), over the period 2018-2035, under the business as usual scenario (BAU) on the left and under the 2°degrees (2Deg) scenario on the right. The export revenue is decomposed into importation costs of intermediate goods (red), taxes, wages and gross operating surplus, for the coal sector (yellow) and for the rest of the economy (blue). Is is assumed that the prices changes of the exported coal impact tax revenue and gross operating surplus. All domestic prices and wages stay constant over the whole period, except  }
%\end{figure}
%
%
%\begin{figure}[!h]
%	%	\centering
%	\hspace{-10pt}\includegraphics[width=1.1\textwidth]{../graphs/NumberEmployees_Sector_Coal-Export_BAU_2Deg.pdf}
%	\caption{\label{NumberEmployees_Sector_Coal-Export_BAU_2Deg}\small The graph shows the number of jobs sustained by coal export demand per sector, over the period 2018-2035, under the business as usual scenario (BAU) on the left and under the 2°degrees (2Deg) scenario on the right.}
%\end{figure}





\begin{figure}[!h]
	\centering
	\hspace{-16pt}\includegraphics[width=0.9\textwidth]{../graphs/ECdecomposition-coal_2014.pdf}
	\caption{\label{ECdecomposition-coal_2014}\small Comparison of the number of jobs created throughout the economy by 1 million rand of final demand for coal, to the average number of jobs created by 1 million rand of final demand. When the net length of the horizontal bars is positive/negative, job creation in the related industries is above/below the economy's average. The decomposition indicates why. For example, for the top group of sectors (Transport, Transport, \ldots), local demand for intermediate and final products is above average, whereas for the second group (Manufacturing, \ldots), local demand for inputs is below the economy's average. "1/Wages" has to be read inversely : for "Coal and lignite", wages are higher than the average in the economy which contributes to a lower than average number of jobs created by one unit of final demand. }
\end{figure}





%CPI's projections for coal export revenues towards 2035, i.e. towards the end of the period under study, are 90\% lower under 2DS than under BAU. The lower revenue is due both to lower export volumes and lower prices. Moreover, they project that the coal sector, thanks to sustained local demand mainly from electricity production, can support about 30 000 jobs on the long term.

To make an estimate from a macroeconomic point of view of the effect on the economy of the difference in coal export revenue in 2035 under 2DS versus BAU, the demand-pull IO-model is used. % provided by eqn. \ref{IOdemand-driven}. %using the macroeconomic picture of the economy provided by the IO-model, of the effect on employment of the difference in coal export demand in 2035 under 2DS versus BAU, a comparison is made between the South African economy in the year 2014 and the same economy with the sole difference that coal exports are diminished by 90\%, following CPI's projections. 
%Both the direct and the indirect impacts in terms of revenue and jobs will be discussed. In a second instance the impact is traced to specific subsectors to evaluate what share of the activity of the concerned economic units is impacted.
The volume and the price change under both scenarios are treated in separate instances. The volume impact is taken into account by determining output and resulting employment per sector with the demand-pull model as in eqn. \ref{IOdemand-driven}, which assumes that prices are constant. Thus, the model is applied to the values of exports calculated as if the price is the BAU price, for both scenarios. Then, to take into account the price-effect, total revenue and gross operating surplus are recalculated assuming the actual price under 2DS. However, %the cost structure of the coal sector as provided by the 2014 IOT is left unchanged. This is not in accordance with the fact that 
the prospective price and volume scenarios of coal export under a 2DS provided by CPI, and used here as input, already include the impact on the profitability of the sector of the lower coal-price on the international market. A number of unprofitable mines are assumed closed in the 2DS scenario. This means that for the 2DS trajectories of price and volume provided, the corresponding cost structure of the coal sector (column of the IOT) is assumed different than the one in the BAU scenario. We will here not attempt to make an estimate of the more efficient technology of the coal-mining sector on the aggregate level that results from the closing of unprofitable coal-mines under 2DS. The matrix of technical coefficients is assumed the same under both scenarios. This means that the obtained measures of profitability under 2DS will be a low estimate, since too high costs are assumed.

%on output and employment per sector throughout the economy, the export volumes under both scenarios are evaluated under the BAU price. 
The actual demand scenarios used to estimate the effect of different export values in 2035 are as follows. For the BAU demand scenario, demand as in the South-african economy in the year 2014, as provided by the IOT, is used. As demand scenario for 2DS, the same as BAU is used with the sole difference that the volume of coal exports is determined such that the ratio of the BAU versus the 2DS volume is equal to the ratio of the volumes specified by CPI's scenarios.

%Denote $p_{E,coal}^{k,2035}$ and $q_{E,coal}^{k,2035}$, $k \in \{\mli{BAU}, \mli{2DS}\}$ the export prices and volumes of coal in the year 2035 for both scenarios. CPI projects export values of coal under respectively BAU and 2DS of $p_{E,coal}^{BAU,2035}\, q_{E,coal}^{BAU,2035}$ and $p_{E,coal}^{2DS,2035}\, q_{E,coal}^{2DS,2035}$. The values fed into the IO-model to determine the difference in output and employment induced per sector are however determined by the BAU price : $p_{E,coal}^{BAU,2035}\, q_{E,coal}^{BAU,2035}$ and $p_{E,coal}^{BAU,2035}\, q_{E,coal}^{2DS,2035}$. The remaining parts of the demand scenarios are as in 2014, as given by the 2014 IOT. 

The impact on employment between the two scenarios (with hypothetical inputs) can be found in table \ref{coalExport_BAUvs2DS_2035}. The top part of the table shows that the number of jobs in the coal sector under 2DS is about half that under BAU. The IO-model projects about 39 600 jobs in the coal sector under 2DS, which is in the same ballpark as CPI's estimate. Of this total number of jobs in coal mining, 15\% is generated by export and some consumption by households. The biggest share, 85\%, is generated by demand from other sectors. 

In addition, the reduction of the coal-mining sector also has an impact on economic activity in other sectors. %the IO analysis allows to make an estimate of the impact on other sectors due to . 
The estimate of this impact, as provided by the IO-model, is shown in the bottom part of table \ref{coalExport_BAUvs2DS_2035}. Over 40 000
jobs are lost throughout the economy, in addition to the job losses in coal-mining itself. Indeed, the amount of jobs lost in coul-mining is about the same as the amount lost throughout the economy, as could be deduced from the employment content analysis in figure \ref{Decomposition_direct_indirect_absolute}. The Transport sector is the first impacted because of the lost activity from transporting coal by rail. 

%The reason behind it is that the labour share is lower than on average in the economy and wages are higher than the average. Local demand for the output of the mining sectors 21, 23-24 is higher than the economy's average. 

Stat SA documents on the transport sector : \citep{LandTransp2014}, \citep{P7000_2013}.

\subsubsection{The domestic coal-market}

% latex table generated in R 3.4.4 by xtable 1.8-3 package
% Tue Sep 25 17:12:36 2018
\begin{table}[ht]
	\centering
	\begin{tabular}{lp{20pt}lcc}
		%\toprule
		\multicolumn{3}{c}{Origin of jobs} &\multicolumn{2}{c}{Demand scenarios}   \\ 
		\multicolumn{3}{c}{} & 2014 & 2014 with 90\% less export \\ 
		\midrule
		Jobs in the coal industry,	& \multicolumn{2}{l}{total}  & 82146 & 39601 \\ 
		& \multicolumn{2}{l}{due to final demand}  & 48490 & \, 5945 \\ 
		
		& \multicolumn{2}{l}{due to other sectors}  & 33656 & 33656 \\ 
		& 	 & Electricity, gas and water & \,\ 7580  & \,\ 7580 \\ 
		&    & Metal ores& \,\ 4540  & \,\ 4540 \\ 
		&    & Construction& \,\ 2456  & \,\ 2456  \\ 														
		&    & Coke oven manufacture& \,\ 2360  & \,\ 2360   \\ 
		%				&    & Other community activities& \,\ 1560  & \,\ 1560   \\ 
		%				&    & Basic iron and steel & \,\ 1530  &  \,\ \,\ 769  \\
		%				&    & Trade & \,\ 1530  & \,\ 1094  \\ 
		%				&    & Food & \,\ 1276  & \,\ 1229 \\  
		%				&    & Real estate activities & \,\ 1227  & \,\ 1219 \\
		%				&    & Transport & \,\ 1165  & \,\ \, 930 \\  
		\midrule
		\multicolumn{3}{l}{Jobs in other sectors, due to the coal sector}  & 57542 &  13000 \\ 
		& 						 & Transport & 18667 &  \ \,2289 \\ 
		& 						 & Trade & 13200 &  \  1618 \\ 
		\bottomrule
	\end{tabular}
	\caption{\label{coalExport_BAUvs2DS_2035}Comparison of the impact of export demand on the number of jobs related to the activity in the coal mining sector (SIC 21). The percent difference in coal exports between the two scenarios is equal to that between BAU and 2DS in 2035 as projected by CPI. The BAU scenario is approximated by the South African economy as in the year 2014.}
\end{table}
%table


%6/11 need to construct output figure for 2018 for the mining sector, in m Rand, with share export and share domestic | Base on CPI, with price unit conversions + base on (take care about volume + price / ) would also need employment for the same year in mining...


%Due to the loss of cross-subsidies from coal-exports, which impacts the profitability of coal-mining, the domestic price of coal might increase to compensate. 
The domestic coal market might be impacted by the volume of coal exported and the price obtained. When prices on the export market are high, then the profit resulting from exported coal can sustain relatively lower prices on the domestic market. Domestically, coal is almost only purchased by industry as inputs (see figure \ref{Uses_Resources_percent_2014}). The lion's share is used for the generation of electricity (table \ref{4_Coal-and-lignite_IIusedBy}). It is expected that cost increases for the electricity provider would be transferred to the rest of the economy by increasing the it's price. Assuming that all sectors pass on the cost increase to their price, the cost-push IO-model (eqn. \ref{cost-push-scen}) allows to estimate how an increase in the domestic coal price impacts the prices of the output of all sectors. 

Figure \ref{} represents the volume and price trajectories for the domestic coal market. These have been obtained using a bottom-up approach based on the number of 
Prospective trajectories have been built for the domestic coal market. 


\begin{figure}[!h]
%	\centering
	\hspace{-10pt}\includegraphics[width=1.1\textwidth]{../graphs/costPush_I4_10.pdf}
	\caption{\label{costPush_I4_10} The above graph shows the increase in the price of all sectors after a 10\% increase in the price of coal, assuming that the cost increase is entirely transferred, without affecting profitability of any of the sectors.}
\end{figure}


Figure \ref{costPush_I4_10} shows the increase in the price of all sectors after a 10\% increase in the price of coal. The resulting increase of the price of electricity is about 1,8\%.


\subsection{Electricity}

The electricity sector can be impacted in two ways by differences between a 2DS and BAU scenario. The increased cost of coal can be a source of a raise of electricity-tariffs. Secondly, the share of coal in the generation-mix might decrease.

\subsubsection{Reduced demand due to increase in electricity-price}

An increase in the price of electricity can imply reduced demand. Motivated by the last (two) decades or price evolution, several studies estimate aggregate and sectoral price-elasticities (\cite{inglesi2010forecasting} and \cite{goliger2018electricity}, references therein). 


\subsubsection{Change in the electricity generation-mix}

%To evaluate the effect of an increase of the price of coal on the price of electricity, 

%\subsubsection{Adapting IOT for different weight of coal in generation mix}
To evaluate the effect of a lower share of coal in the generation mix in 2DS versus BAU, a corresponding Leontief matrix needs to be constructed. 

To determine how the technical coefficients of sector SIC 41 evolve with different shares of coal, keeping the total volume of electricity generated constant, the coefficients are divided in three groups : the ones that are necessary for coal production, the ones that are necessary for one of the other technologies and the ones that are associated with both of the former groups. %The coefficients that are not in table \ref{GenerationMixcoefficients} are implicitly assumed to be independent of what technology generates 1 unit of electricity. In other words, the amount spent by sector SIC 41, as specified in the 2014 IOT table, for example on computer activities (SIC 86\&88), is determined by the total amount of $GWh$ of electricity produced in 2014, and not by what share was produced with coal. 
Table \label{GenerationMixcoefficients} specifies the dominant inputs of the electricity sector and to which of the three groups they are associated. The technical coefficients of the third group are divided in two shares, such that, calibrating all prices in the 2014 IOT to 1, the cost share of coal in te IOT is equal to the cost-share of coal as stated in the 2014 Integrated report from Eskom of the breakdown of the primary energy cost according to generation technology (68\%). Knowing that coal provided in 2014 90\% of $GWh$, then when the share of coal diminishes by a factor $1 - \lambda$, all the coefficients (and shares) associated with coal diminish with ($1-\lambda$) and the others with $1 + 90/10 \, \lambda$.

Because in 2014 coal is cheaper than the average of the other technologies, lower share of coal result in a higher generation cost.


\li{Construct elec technology with GTAP data}
%Future price evolutions of renewables ?


%in conjunction with knowledge of the amount of $GWh$ produced per technology over the year, a breakdown of the technical coefficients of electricity generation in the 2014 IOT can be obtained.  



%To evaluate the effect of a lower share of coal in the generation mix in 2DS versus BAU, it is necessary to know what the relative cost is of generating electricity from coal versus from other inputs. This can be estimated using the decomposition of the primary energy cost as given by Eskom's 2014 annual report \citep{Eskom2014AR}, given in table \ref{EskomPrimaryEnergy}. 


%To construct the corresponding matrix, first the technical coefficients that reflect the generation mix need to be identified. Secondly, a quantitative estimate has to be made on how the coefficients evolve to reflect different generation mixes.
%
%The technical coefficients that catch the generation mix are found in the column of sector SIC 41, Electricity, gas and water of the matrix, which contains the inputs necessary to generate electricity. As shown in table \ref{SIC41composition}, sector SIC 41 contains also the manufacture of gas and steam ad hot water supply ; however electricity generation is the biggest share of activity (xx\% ?). The relevant entries are listed in table \ref{GenerationMixcoefficients}. The main aspect under investigation of the generation mix is a reduced share of coal. Thus, if less coal is used to generate 1 unit of electricity, a hypothesis has to be made on what other production technology would replace coal. %In order to avoid making heuristic assumptions on this issue, 
%It is assumed that all the other production technologies take up the slack in equal parts, proportional to the share they provide as reflected in the IOT for the year 2014. Thus, the technical coefficients of sector SIC 41 that reflect the generation mix are divided in three groups : the ones that are necessary for coal production and the ones that are necessary for one of the other technologies. The coefficients that are not in table \ref{GenerationMixcoefficients} are implicitly assumed to be independent of through what generation technology 1 unit of electricity is produced. In other words, the amount spent by sector SIC 41, as specified in the 2014 IOT table, for example on computer activities (SIC 86\&88), is determined by the total amount of $MWh$ of electricity produced in 2014, and not by what share was produced with coal. The input of the Electricity, gas and water supply to itself contains subsectors --- listed in table \ref{SIC41composition} --- who's shares in total cost are assumed independent of technology, such as the distribution of electricity (SIC 41111), and shares that are assumed proportional to the primary energy inputs used for generation through non-coal technologies, such as Gas and Generation. The latter subsector reflects both electricity produced by IPP's as well as by coal-fired plants as by the Koeberg nuclear station or any other of Eskom's generation plants. For simplicity, we will assume the input of SIC 41 to itself independent of generation mix. 

\begin{table}[ht]
	\centering
	\begin{tabular}{lclc}
		\toprule
		%\midrule
		\multicolumn{2}{l}{Input, SIC code}	& Technology used for  &  \% cost (2014) \\ 
		\midrule
		Coal and lignite & 21 &  Coal &  42.2    \\
	 	Electricity, gas and water & 41 &  & 13.2     \\
	%	Generation &41111 & Other &     \\
		Gas &412 & Other &     \\
		Transport &71 \& 74 &  Coal  &   6.1   \\
		Coke oven manufacture & 331 \& 332 & Coal &  4.0 \\
		Nuclear fuel & 333 \& 334 & Other & 0.3   \\
%		Other mining &  25 & Other &   0.2 \\
%		 &    &    \\
		\bottomrule
	\end{tabular}
	\caption{\label{GenerationMixcoefficients}The technical coefficients of the Leontief matrix that capture the generation mix. From left to right, the columns specify from what sectors inputs are used by sector 41 (Description and SIC code), whether they are used for generating electricity by coal plants are any other generation technology, and what share of total output the input represents. As a reference, intermediate inputs represent 39\% and employee compensation 17\% of output. Source : \cite{IOT2014}.}
\end{table}

%Given Z the IO matrix of inter-industry sales in current prices of the SA economy. The technical coefficients for the electricity sector reflect the particular generation mix of the year for which the matrix is made. To evaluate the effect on the economy of a slightly different generation mix, with the same production of electricity in $MWh$, with possibly different prices, we need to know which of the IO entries will be affected and determine for the relevant coefficients their volume / price ratio. We need to know, when 10\% less coal is used for electricity generation, what coefficients need to increase, and by what percentage, to generate the same final $MWh$. Identifying the relevant coefficients is a question of identifying the place and role of each of the inputs in the production process. Once this is established, to determine by what percentage they need to increase, for the coefficients that represent primary energy inputs that can be interpreted to correspond to the data given in table \ref{EskomPrimaryEnergy}, an estimate is made as follows : .
%
%we want to know, the relative price of generating electricity with coal versus all the other sources. To this end, to  
%
%Denoting Z the $n\times n$ IO matrix of inter-industry sales in current (2014) prices $p^{2014}_i$, with elements $z^{2014}_{ij} = p_i \, q_{ij}$ and $x$ the vector of final output per sector with elements $p_i^{2104} \, x_i$, where $x_i$ is the output, in quantity of goods of sector $i$.  
%
%we want to estimate a new matrix 
%
%
%
% This can be estimated by identifying the cost of the primary energy inputs in the IOT table, combined with 
%
%Assuming that electricity will be produced with less coal and more other inputs translates itself in the IOT model in a change in the relevant technical coefficients of the Leontief matrix. 



\begin{table}[ht]
	\centering
	\begin{tabular}{rlrrr}
		\toprule
		%\midrule
		SIC code	& Description  & Output$^*$ & Revenue$^{**}$ & \%   \\ 
		\midrule
		4 & Electricity, gas and water supply & 221 633 & & 100 \\ 
		41 & Electricity, gas, steam and hot water supply &163 081& & 73.6 \\ 
		411 & Production, collection and distribution of electricity &  &  &  \\ 
		 & Eskom &  & 147 691 & 66.6 \\ 
		 41111 & Generation 	  	  	&& & \\
 		41112 & Distribution of purchased electric energy only 	&&  	&  	  	\\
		41113 & Generation and/or distribution for own use & 8 454& & 3.8\\
		412 & Manufacture of gas; distribution of gaseous fuels & & & \\ 
		413 & Steam and hot water supply &  & &\\ 
		42 & Collection, purification and distribution of water & 58 552 & &26.4 \\ 
		\bottomrule
	\end{tabular}
	\caption{\label{SIC41composition}The composition of sector SIC 41, Electricity, gas, steam and hot water supply. Output and revenue are in million Rand. The estimates are based on 2014 data. The estimate for the value for subsector 41113 is based on how much sector 41 uses from itself, as stated in the IOT. Sources : \href{http://www.statssa.gov.za/additional_services/sic/mdvdvmg4.htm}{Stat SA website, SIC 41 description}, (*) 2014 IOT in current prices \citep{IOT2014}, (**) \href{http://www.eskom.co.za/IR2015/Documents/EskomIR2015single.pdf}{Integrated report}, \citep{Eskom2014AR}.}
\end{table}


\begin{table}[ht]
	\centering
	\begin{tabular}{lrrlr}
		\toprule
		%\midrule
		Primary energy source	& Cost, mR  & Cost, \% & IOT  & IOT\\ 
		\midrule
		Coal and gas & 51 554   &  61.8   & Input I4 to I33 &\\
		Nuclear  fuel &  530  &   0.6  & Input I17 to I33& \\
		OGCT fuel & 9 546   &   11.4   & Input I33 to I33  & \\
		Electricity  imports & 3 679   &  4.4  & I33 Import & \\
		IPPs &  9 453  &  11.3  & Input I33 to I33 & \\
		Environmental levy & 8353 &   10.0   & I33 Taxes & \\
		Other &  310  &  0.4  & Input I4 to I33(?) 	& \\
		\bottomrule
	\end{tabular}
	\caption{\label{EskomPrimaryEnergy}Eskom's Primary energy cost breakdown 2014, million Rand. Source : \href{http://www.eskom.co.za/IR2015/Documents/EskomIR2015single.pdf}{Integrated report}, \citep{Eskom2014AR}.}
\end{table}




\subsection{Oil}

Global demand for crude oil in 2035 would be $24\%$ lower than in BAU and the price $20\%$ lower. 


%\subsection{Platinum group metals}


\clearpage

% SA.bib
\newpage
\bibliography{SA}
\bibliographystyle{apalike}
\clearpage

\appendix

\section{Mean employment content decomposition} \label{ECmean}

%\section{What is in the data ?}

\section{Correspondence tables}

\clearpage
\begin{table}[ht]
	\centering
	%\small
	\vspace{-36pt}\hspace{-10pt}\caption{\label{correspondance_OECD_ZA}Correspondence between the 50-sector-classification of the South African IOTs in national SIC classification and the 34-sector-classification in OECD IOTs following ISIC rev.3.1. Sources : \href{http://www.oecd.org/sti/ind/IOT_Industries_Items.pdf}{OECD code to ISIC rev. 3}, \href{https://www.statssa.gov.za/additional_services/sic/descrip6.htm}{SIC to ISIC}, \cite{IOT2014} and \citetalias{united2002isic}.}
	\vspace{8pt}%\hspace{-10pt}
	\begin{adjustbox}{width=\textwidth}
	%\scriptsize
	\normalsize
		\begin{tabular}{lp{500pt}lll}
		\toprule
		Code & Industry Description & SIC & ISIC rev.3.1 & OECD code \\ \arrayrulecolor{black!30}\midrule
		I1 & Agriculture, hunting and related services & 11 & 01 & C01T05 \\ \midrule
		I2 & Forestry, logging and related services & 12 & 02 & C01T05 \\ \midrule
		I3 & Fishing, operation of fish hatcheries and fish farms & 13 & 05 & C01T05 \\ \midrule
		I4 & Mining of coal and lignite & 21 & 10 & C10T14 \\ \midrule
		I5 & Mining of gold, uranium and metal ores  & 23-24 & 12-13 & C10T14 \\ \midrule
		I6 & Other mining and quarrying & 25 & 14 & C10T14 \\ \midrule
		I7 & Manufacture of food products & 301-4 & 151-154 & C15T16 \\ \midrule
		I8 & Manufacture of beverage and tobacco products & 305-6 & 155\_16 & C15T16 \\ \midrule
		I9 & Spinning, weaving, finishing of textiles and manufacture of other textiles & 311-2 & 17 & C17T19 \\ \midrule
		I10 & Manufacture of knitted, crocheted fabrics, wearing apparel, fur articles and dying of fur & 313-5 & 18 & C17T19 \\ \midrule
		I11 & Tanning and dressing of leather, manufacture of luggage, handbags, saddler and harness & 316 & 191 & C17T19 \\ \midrule
		I12 & Manufacture of footwear & 317 & 192 & C17T19 \\ \midrule
		I13 & Manufacture of products of wood, cork, straw and plaiting materials%; and sawmilling and planning of wood 
		& 321-2 & 20 & C20 \\ \midrule
		I14 & Manufacture of paper and paper products & 323 & 21 & C21T22 \\ \midrule
		I15 & Publishing, printing and service activities related to printing; reproduction of recorded media & 324-6 & 22 & C21T22 \\ \midrule
		I16 & Manufacture of coke oven products; and petroleum refineries/ synthesisers & 331-2 & 231-232 & C23 \\ \midrule
		I17 & Processing of nuclear fuel; and basic chemicals & 333-4 & 233 & C23 \\ \midrule
		I18 & Manufacture of other chemical products, and man-made fibres & 335-6 & 24 & C24 \\ \midrule
		I19 & Manufacture of rubber products & 337 & 251 & C25 \\ \midrule
		I20 & Manufacture of plastic products & 338 & 252 & C25 \\ \midrule
		I21 & Manufacture of glass and glass products & 341 & 261 & C26 \\ \midrule
		I22 & Manufacture of non-metallic mineral products (nec) & 342 & 269 & C26 \\ \midrule
		I23 & Manufacture of furniture & 391 & 36 & C36T37 \\ \midrule
		I24 & Manufacture nec and recycling nec & 392\_395 & 37 & C36T37 \\ \midrule
		I25 & Manufacture of basic iron and steel and casting of metals & 351\_353 & 271\_273 & C27 \\ \midrule
		I26 & Manufacture of precious metals and non-ferrous metals & 352 & 272 & C27 \\ \midrule
		I27 & Manufacture of structural metal products, tanks, %reservoirs 
		and steam generators, and other %fabricated metal products, metalwork service activities 
		& 354-5 & 28 & C28 \\ \midrule
		I28 & Manufacture of general/special purpose machinery, %special purpose machinery, 
		household appliances and office %, accounting and computing 
		machinery & 356-9 & 29-30 & C29, C30T33X \\ \midrule
		I29 & Manufacture of electrical machinery and apparatus nec & 36 & 31 & C31 \\ \midrule
		I30 & Manufacture of %electronic valves, tubes, other 
		electric components; television, %and 
		radio, %transmitters, apparatus for line 
		telephony, %and line telegraphy; and television and radio receivers, sound or 
		video %recording or reproducing 
		apparatus %and associated goods 
		& 371-3 & 32 & C30T33X \\ \midrule
		I31 & Manufacture of medical and measuring appliances, %and instruments and appliances for measuring, checking, testing, navigating and other purposes; manufacture 
		of optical instruments %and photographic equipment
		; %and watches and 
		clocks & 374-6 & 33 & C30T33X \\ \midrule
		I32 & Manufacture of motor vehicles and parts, %bodies (coachwork) for motor vehicles, 
		locomotives, aircraft, spacecraft, trailers and semi-trailers; %motor vehicle and engine parts and accessories 
		; building and repairing ships and boats%, and manufacture of transport equipment not elsewhere classified. 
		& 381-387 & 34-35 & C34, C35 \\ \midrule
		I33 & Electricity, gas steam and hot water supply & 41 & 40 & C40T41 \\ \midrule
		I34 & Collection, purification and distribution of water & 42 & 41 & C40T41 \\ \midrule
		I35 & Construction & 5 & 45 & C45 \\ \midrule
		I36 & Wholesale and commission trade, retail trade; and sale, maintenance and repair of motor vehicles and motor cycles, retail trade in automotive fuel & 61-63 & 51\_52\_50 & C50T52 \\ \midrule
		I37 & Hotels and restaurants & 64 & 55 & C55 \\ \midrule
		I38 & Transport (land, water, air) ; %and supporting and auxiliary transport activities, 
		activities of travel agencies & 71-74 & 60-63 & C60T63 \\ \midrule
		I39 & Post and telecommunications & 75 & 64 & C64 \\ \midrule
		I40 & Financial intermediation, except insurance and pension funding & 81 & 65 & C65T67 \\ \midrule
		I41 & Insurance and pension funding, except compulsory social security  & 82 & 66 & C65T67 \\ \midrule
		I42 & Activities auxiliary to financial intermediation & 83 & 67 & C65T67 \\ \midrule
		I43 & Real estate activities & 84 & 70 & C70 \\ \midrule
		I44 & Renting of machinery and equipment, without operator, and of personal and household goods & 85 & 71 & C71 \\ \midrule
		I45 & Research and development  & 87 & 73 & C73T74 \\ \midrule
		I46 & Computer and related activities; and other business activities & 86\_88 & 72\_74 & C72, C73T74 \\ \midrule
		I47 & Other community, social and personal service activities (including government) & 91\_94 & 75\_90 & C75,  \\ \midrule
		I48 & Education & 92 & 80 & C80 \\ \midrule
		I49 & Health and social work & 93 & 85 & C85 \\ \midrule
		I50 & Other service activities nec & 95\_96\_99 & 91\_92\_93 & C90T93 \\
		\arrayrulecolor{black}\bottomrule	
	\end{tabular}
	\end{adjustbox}
	\thispagestyle{empty}
\end{table}

\clearpage
\section{Estimation of separate domestic and import IOTs}\label{IOTopti}


%\section{Decomposition}
%
%Equation \label{allUSes} can be rewritten as :
%\begin{equation}
%%y_i + m_i - d_i^m= \sum_j \Bigg(\frac{z_{ij}^d}{y_i + m_i-d_i} + \frac{z_{ij}^m}{y_i + m_i-d_i} \Bigg) (y_i + m_i) + d_i^d
%y_i + m_i - d_i^m= \sum_j \big(z_{ij}^d + z_{ij}^m \big) + d_i^d, 
%\end{equation}
%where $d^r = c^r_i + g_i^r + i_i^r + inv_i^r + x_i^r$ , $r\in\{d,m\}$ denote final demand for domestically produced goods and services and for imported goods. The aggregates $m_i - d_i^m$ are the imports used as intermediate inputs, or differently put, the importation needed for domestic production.
%
%We are interested in, given a final demand vector for domestically produced goods $\boldsymbol{d}^d$, not only how much domestic production this entails, but also how much importation this generates because of the use of imported goods as intermediate inputs. To this end, we define modified matrices of technical coefficients $\tilde{A}^r$ as the matrices with elements $\tilde{a}^r_{ij}$ : 
%\begin{align}
%\tilde{a}^r_{ij} = \frac{z_{ij}^r}{y_i + m_i - d_i^m} \label{defAtilde}
%\end{align}
%These matrices tell us how much intermediary inputs, respectively of domestic and of imported origin, are needed per unit of domestic output and imported intermediate goods. Additionally, we define the technical coefficients matrix $\tilde{A}= \tilde{A}^d + \tilde{A}^m$ that states the total intermediary inputs, of any origin, per unit of imports + domestic output.
%
%%We then define the matrices $Y$ and $M_{int}$ such that :
%
%Given $D^d$ the matrix with demand $d^d$ on the diagonal and 0's elsewher, we then define the matrix $\mli{YM}$ as :
%\begin{equation}
%\mli{YM} = (\mathbb{I} - \tilde{A})^{-1} \, D^d \label{YM}
%\end{equation}
%The element $ym_{ij}$ of $\mli{YM}$ is the sum of domestic production and imported intermediate goods of sector $i$ needed to satisfy final demand by sector j $d_j^d$.
%
%\li{The column sums of $M_{int}$ are the same as the column sums of $Z^m$, but are the two matrices the same ?}
%
%We now want to split matrix $\mli{YM}$ up into domestic output and intermediate inputs that were imported.
%
%For this we will decompose the factor that is analogous to the Leontief inverse :
%
%\begin{align}
%(\mathbb{I} - \tilde{A})^{-1} &= (\mathbb{I} - \tilde{A}^d - \tilde{A}^m)^{-1} \nonumber \\
%&= (\mathbb{I} - \tilde{A}^d)^{-1} + (\mathbb{I} - \tilde{A}^m)^{-1} - \mathbb{I} + rest \label{decomp1}
%\end{align}
%
%The first term on the right hand side measures domestic output necessary to satisfy demand $D^d$. We take a closer look at the second and third terms at the right-hand side of  \ref{decomp1} by developing the power series : 
%\begin{equation}
%(\mathbb{I} - \tilde{A}^m)^{-1} - \mathbb{I} = \sum_{k=1}^{+\infty}  (\tilde{A}^m)^k  \label{importedInputsneeded}
%\end{equation}
%Whereas the significance of $\tilde{A}^m D^d$ is quite clear --- it determines the imported intermediate inputs needed to satisfy domestically produced final demand, the meaning of the following terms $(\tilde{A}^m)^k D^d$ with $k>1$ seems nebulous when interpreted along the same line. This line of interpretation looks at the matrix $\tilde{A}^m D^d$ row-wise : we seek an answer to the question how much output from each sector is needed to satisfy the demanded production from other sectors. In this case, the "output from each sector needed" are imported goods. The term $\tilde{A}^m \tilde{A}^m D^d$ does not make sense in this interpretation : what are the import needed to generate the imports ? Rather, one can think of $\tilde{A}^m D^d$ column-wise and interpret each element $a^m_{ij}\,d^d_j$as a contribution to the cost of producing a good $j$, or a contribution to the value of the good $j$ by using the intermediate input $i$, obtained through importation. Then, $\tilde{A}^m \tilde{A}^m D^d$ also takes on a clear significance : it represents the value generated indirectly by using imported intermediate inputs to %generate value by 
%produce output that was produced with imported inputs. 
%%Instead of thinking in terms of the output of sector $i$ needed to satisfy demand from sector $j$, think of it vice-versa : it is the value of output of sector $j$ originating from using input $i$. Of course, a single input cannot produce a good. This is why the former sentence contains "value of". Using input $i$ contributes to the cost of production or to the value of good $j$. 
%
%Why would one determine how much imported goods are needed to import goods ? 
%
%To able to create the value generated by using imported inputs, a certain amount of imported inputs are needed, etc.
%Beware, is is using the imported inputs that creates value domestically ; producing obviously doesn't.
%
%It is now clear that the terms \ref{importedInputsneeded} contribute to the amount of imported goods needed to satisfy domestically produced final demand. The terms in the $rest$ term contain both $\tilde{A}^d$ and $\tilde{A}^m$ factors. For example $\tilde{A}^m \tilde{A}^d D^d$ represents the value generated by using imported goods to produce the intermediate inputs necessary for the production of the intermediate inputs $\tilde{A}^d D^d$, where this last term can be worded as the value created by using locally produced inputs to produce demand $D^d$.
%
%The rest term is known, given by $$ rest =(\mathbb{I} - \tilde{A})^{-1} - (\mathbb{I} - \tilde{A}^d)^{-1} - (\mathbb{I} - \tilde{A}^m)^{-1} + \mathbb{I}.$$
%It contains both value created by using imported and domestically produced inputs and will her be roughly split up according to the last step in the production process :
%$$
%rest = \tilde{A}^d (\tilde{A}^m + T) + \tilde{A}^m(\tilde{A}^d + T)
%$$
%The first term on the right-hand side is counted as domestic production and the second term as imports. (The second term are all the imports needed/used to produce the intermediary inputs --- direct inputs and the chain of indirect inputs --- that have been produced using both imported and domestic inputs.)
%
%The unknown tail $T$ is easily determined : $$ T = ( \tilde{A}^d +  \tilde{A}^m)^{-1}\, ( \tilde{A}^m  \tilde{A}^d +  \tilde{A}^d \tilde{A}^m)$$
%Matrix $\mli{LM}$ as given in eqn. \ref{YM} is then split up as :
%\begin{align}
%\tilde{Y} &= (\mathbb{I} - \tilde{A}^d)^{-1} \, D^d + \tilde{A}^d \big(\tilde{A}^m +  ( \tilde{A}^d +  \tilde{A}^m)^{-1}\, ( \tilde{A}^m  \tilde{A}^d +  \tilde{A}^d \tilde{A}^m)\big) \, D^d \\
%\tilde{M} &= (\mathbb{I} - \tilde{A}^m)^{-1} \, D^d - \mathbb{I} + \tilde{A}^m \big(\tilde{A}^d +  ( \tilde{A}^d +  \tilde{A}^m)^{-1}\, ( \tilde{A}^m  \tilde{A}^d +  \tilde{A}^d \tilde{A}^m)\big) \, D^d
%\end{align}
%
%
%%The value created by producing a product is the value of the product. The value of using the product as intermediate input is 
%
%%To resolve this issue, we determine the relationship between $\tilde{A}^m$ and $A^m$, where the latter is the technical coefficients matrix for imported goods, defined as usual. Starting from \ref{defAtilde} in matrix notation :
%
%%\begin{align}
%%\tilde{A}^m &= Z^m \, (diag(\boldsymbol{y + m - d^m}) )^{-1} \nonumber\\
%%&=  Z^m \, (diag(\boldsymbol{y}))^{-1} \, \big( \mathbb{I} +  (diag(\boldsymbol{y}))^{-1} \, diag(\boldsymbol{m - d^m})   \big)^{-1} \nonumber\\
%%&= Z^m \, (diag(\boldsymbol{y}))^{-1} \, \sum_{k=0}^{+\infty} \big(-(diag(\boldsymbol{y}))^{-1} \, diag(\boldsymbol{m - d^m})\big)^k \nonumber\\
%%&= A^m  \, \sum_{k=0}^{+\infty} \big(-(diag(\boldsymbol{y}))^{-1} \, diag(\boldsymbol{m - d^m})\big)^k
%%\end{align}


\section{Extra graphs}

\subsection{Coal Export}

\begin{figure}[!h]
	%	\centering
	\hspace{-10pt}\includegraphics[width=1.1\textwidth]{../graphs/ValueAdded_Sector_Coal-Export_BAU_2Deg.pdf}
	\caption{\label{ValueAdded_Sector_Coal-Export_BAU_2Deg}\small The graph shows the revenue from the export of coal in constant Rand (2014), over the period 2018-2035, under the business as usual scenario (BAU) on the left and under the 2°degrees (2Deg) scenario on the right. The export revenue is decomposed into importation costs of intermediate goods (red), taxes, wages and gross operating surplus, for the coal sector (yellow) and for the rest of the economy (blue). Is is assumed that the prices changes of the exported coal impact tax revenue and gross operating surplus. All domestic prices and wages stay constant over the whole period, except  }
\end{figure}


\begin{figure}[!h]
	%	\centering
	\hspace{-10pt}\includegraphics[width=1.1\textwidth]{../graphs/NumberEmployees_Sector_Coal-Export_BAU_2Deg.pdf}
	\caption{\label{NumberEmployees_Sector_Coal-Export_BAU_2Deg}\small The graph shows the number of jobs sustained by coal export demand per sector, over the period 2018-2035, under the business as usual scenario (BAU) on the left and under the 2°degrees (2Deg) scenario on the right.}
\end{figure}









\end{document}